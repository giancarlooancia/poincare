% ******************************************************************************
% *************************** INSIEMISTICA *************************************
\newcommand{\numberset}{\mathbb}
\newcommand{\N}{\numberset{N}}
\newcommand{\Z}{\numberset{Z}}
\newcommand{\R}{\numberset{R}}
\newcommand{\C}{\numberset{C}}
\newcommand{\1}{\mathds{1}}
% ******************************************************************************
% ****************************** OPERATORS **************************************
%\DeclarePairedDelimiter{\abs}{\lvert}{\rvert}
\DeclarePairedDelimiter{\mynorm}{\lVert}{\rVert}
\DeclarePairedDelimiter{\inner}{\langle}{\rangle}
\DeclareMathOperator{\sgn}{sgn}
\DeclareMathOperator{\Realpart}{Re} % ridefinisco parte reale
\DeclareMathOperator{\Impart}{Im}
\renewcommand{\Re}{\Realpart} % ridefinisco parte reale (altrimenti dà simbolo in gotico)
\renewcommand{\Im}{\Impart} 
\DeclareMathOperator*{\argmax}{arg\,max}
%\DeclareMathOperator{\Tr}{Tr}
%\DeclareMathOperator{\Res}{Res}


% ******************************************************************************
% *************************** VECTOR CALCULUS **********************************
\newcommand{\bcdot}{\boldsymbol{\cdot}} % così \bcdot è prodotto scalare in grassetto
\renewcommand{\vec}{\boldsymbol}
\newcommand{\del}{\vec{\nabla}}


% ******************************************************************************
% *************************** DIFFERENTIATION **********************************
\newcommand{\ud}{\mathop{}\!\mathrm{d}}
\newcommand{\udd}{{\ud}^2}
\newcommand{\udt}{{\ud}^3}
\newcommand{\udq}{{\ud}^4}
\newcommand{\bb}[1]{\mathbb{#1}}
\newcommand{\de}{\partial}


% ******************************************************************************
% ******************************* THEOREMS *************************************
\theoremstyle{plain}
\newtheorem{theorem}{Theorem}

\theoremstyle{plain}
\newtheorem*{principle}{Principle}

\theoremstyle{plain}
\newtheorem{lemma}{Lemma}

\theoremstyle{definition}
\newtheorem{definition}{Definition}

\theoremstyle{remark}
\newtheorem*{remark}{Remark}

\renewcommand{\thelemma}{L.\arabic{lemma}}
\renewcommand{\thetheorem}{T.\arabic{theorem}}


% ******************************************************************************
% ******************************* SHORTCUTS ************************************
\newcommand{\invgamma}{\sqrt{1- \frac{v^2}{c^2}}}
\renewcommand{\L}{\mathcal{L}}
\newcommand{\g}{\mathfrak{g}}
\newcommand{\h}{\mathfrak{h}}
\newcommand{\M}{\mathcal{M}}
\newcommand{\D}{\mathcal{D}}
\newcommand{\qi}{q^{(m)}}
\newcommand{\qf}{q^{(g)}}
\newcommand{\bdelta}{\bar{\delta}}
\newcommand{\deltat}{{\delta}^3}
\newcommand{\deltaq}{{\delta}^4}
\renewcommand{\epsilon}{\varepsilon}
\newcommand{\phis}{{\phi}^*}
\newcommand{\hbarq}{{\hbar}^2}
\renewcommand{\H}{\mathcal{H}}
\newcommand{\q}{\hat{\vec{q}}}
\newcommand{\Tau}{\mathcal{T}}
\newcommand{\ray}{\mathcal{R}}

% ******************************************************************************
% ************************ SHORTCUTS WITH ARGUMENTS ****************************
\newcommand{\bravec}[1]{\bra{\vec{#1}}} 
\newcommand{\ketvec}[1]{\ket{\vec{#1}}} 
\renewcommand{\op}[1]{\hat{#1}}
\newcommand{\opvec}[1]{\op{\vec{#1}}}
\newcommand{\dual}[1]{\widetilde{#1}}
\DeclareMathOperator{\Aut}{Aut}
\DeclareMathOperator{\id}{id}

% ******************************************************************************
% ******************************** GRAPHICS ************************************
\renewcommand\qedsymbol{$\blacksquare$}

% ******************************************************************************
% ******************************** matrices and scalar prodict ************************
\newcommand{\irow}[1]{% inline row vector
  \begin{smallmatrix}(\,#1\,)\end{smallmatrix}%
}

\newcommand{\icol}[1]{% inline column vector
  \left(\begin{smallmatrix}#1\end{smallmatrix}\right)%
}

\newcommand{\scalar}[2]{\langle #1, #2 \rangle}
\renewcommand{\norm}[1]{\scalar{#1}{#1}}