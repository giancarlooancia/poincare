\appendix
\section{Levi-Civita Symbol}\label{app:levi-civita}
In this appendix we fix the conventions for the \emph{Levi-Civita} symbol. We mainly refer to the Wikipedia page~\cite{wikipedia}. In these notes we're interested only in the Levi-Civita symbol in three dimensions, therefore we'll stick to it. We define the Levi-Civita symbol
\begin{equation}
    \epsilon_{ijk} = 
    \begin{cases}
        +1, \textup{if $(i,j,k)$ is an even permutation of $(1,2,3)$}, \\
        -1, \textup{if $(i,j,k)$ is an odd permutation of $(1,2,3)$},\\
        0, \textup{otherwise} ,
    \end{cases}
\end{equation}
and adopt the convention
\begin{equation*}
    \epsilon^{ijk} = \epsilon_{ijk}.
\end{equation*}
Pay attention to what convention the sources you're referring use. Indeed, this is not always the most convenient choice, in particular when dealing with the Levi-Civita tensor. Without dealing these details, in our case, just recall that when to contiguous indices are permuted, we get a minus sign.

Using Einstein's summation convention, some useful relations are
\begin{subequations}
\begin{align}
    \epsilon_{ijk} \epsilon^{pqk} &= \delta^p_i \delta^q_j - \delta^q_i \delta^p_j \label{appeq:levi-civita-delta} \\
    \epsilon_{jmn} \epsilon^{imn} &= 2 \delta^i_j \\
    \epsilon_{ijk} \epsilon^{ijk} &= 6 .
\end{align}
\end{subequations}

Let's then show that
\begin{equation*}
    J^i = \frac{1}{2} \epsilon^{ijk} M^{jk} \iff M^{ij} = \epsilon^{ijk} J^k, \quad \textup{if $M^{ij} = -M^{ji}$.}
\end{equation*}
\begin{proof}
    Multiplying both sides by $\epsilon_{iln}$ and using eq.~\eqref{appeq:levi-civita-delta} and the antisymmetry:
    \begin{equation*}
        \epsilon_{iln} J^i = \frac{1}{2} \epsilon_{iln} \epsilon^{ijk} \omega^{jk} = \frac{1}{2} ( \delta^j_l \delta^k_n - \delta^k_l \delta^j_n ) \omega^{jk} = \frac{1}{2} (\omega^{ln} - \omega^{nl}) = \omega^{ln} .
    \end{equation*}
    Relabelling dummy indices and using $\epsilon_{ijk} = \epsilon_{kij}$, we finally obtain
    \begin{equation*}
        \omega^{ij} = \epsilon_{ijk} \theta^k = \epsilon^{ijk} \theta^k \qedhere
    \end{equation*}

\end{proof}