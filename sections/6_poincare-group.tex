%%%%%%%%%%%%%%% POINCARE GROUP %%%%%%%%%%%%%%%%%
\section{Poincaré Group}
The Lorentz transformations defined in sec.~\ref{sec:lorentz-group} guarantee that the norm $x^2 \coloneq x_\mu x^\mu$ of a four-vector is invariant under a transformation. However, the principles of special relativity require that the line element $(\ud x)^2 = \eta_{\mu\nu} \ud x^\mu \ud x^\nu = c^2 (\ud t)^2 - (\ud \vec{x})^2$ is invariant. This guarantees that the speed of light is the same in each inertial frame. 

In order to satisfy the above condition, we can enlarge the Lorentz transformations, adding constant translations too. We call the resulting group \emph{Poincaré group} or \emph{inhomogeneous Lorentz group}. For simplicity, we only take the connected component of the Lorentz group, which allows us to consider the connected component of the Poincaré group, given by
\begin{equation}
    ISO(1,3)^+ = \{ (a,\Lambda) : a \in \R^4, \Lambda \in SO(1,3)^+ \},
\end{equation}
with group multiplication defined by
\begin{equation}
    (\Lambda_1, a_1) (\Lambda_2, a_2) = (\Lambda_1 \Lambda_2, a_1 + \Lambda_1 a_2),
\end{equation}
as we'll show soon. In the \emph{defining representation}, the group elements are matrices acting on the Minkowski vector space $\R^{1,3}$. Not to be too pedant with the notation, we continue to denote group elements in this particular representation with $(\Lambda,a)$.

\begin{remark}
    Pay attention to the fact that $\R^{1,3}$ is \emph{not} the flat Minkowski spacetime $\M$. Rather, it can be thought as its tangent space at a point. Then, it's a case that we are able to identify this tangent space with the spacetime itself, if the latter is flat.
\end{remark}

Then, the action of $(\Lambda,a)$ on $R^{1,3}$ is given by
\begin{equation}\label{eq:poincare-definition}
    x^\mu \to {x'}^\mu = (\Lambda, a) x = \tensor{\Lambda}{^\mu_\nu} x^\nu + a^\mu ,
\end{equation}
where $\tensor{\Lambda}{^\mu_\nu}$ is a Lorentz transformation in the defining representation, as defined in sec.~\ref{sec:lorentz-group}, and $a^\mu \in \R^(1,3)$ is an arbitrary constant four-vector. Basically, we're adding four new parameters, having in total a $10$ parameter group, containing translations, rotations, and boosts.

We can check again the group axioms~\ref{def:group-axioms}, considering that by means of eq.~\eqref{eq:representation-property}, a representation inherits the composition rule from the group. We obtain
\begin{subequations}
\begin{align}
    (\Lambda_1, a_1) (\Lambda_2, a_2) &= (\Lambda_1 \Lambda_2, a_1 + \Lambda_1 a_2), \label{eq:poincare-group-multiplication} \\
    \left[(\Lambda_1,a_1)(\Lambda_2,a_2)\right] (\Lambda_3,a_3) &= (\Lambda_1, a_1) \left[ (\Lambda_2,a_2)(\Lambda_3,a_3) \right], \label{eq:poincare-associative} \\
    \1 &= (\1, 0) ,\label{eq:poincare-unit-element} \\
    (\Lambda, a)^{-1} &= (\Lambda^{-1}, -\Lambda^{-1}a). \label{eq:poincare-inverse-element}
\end{align}
\end{subequations}
\begin{proof}
    After two consecutive Poincaré transformations, using eq.~\eqref{eq:poincare-definition},
    \begin{equation*}
        x \to x' = (\Lambda_2,a_2)x = \Lambda_2 x + a_2 \to (\Lambda_1,a_1) x' = \Lambda_1 (\Lambda_2 x + a_2) + a_1 = (\Lambda_1 \Lambda_2) x + (\Lambda_1 a_2 + a_1) ,
    \end{equation*}
    showing that the composition of two Poincaré transformations gives another transformation, as appointed by eq.~\eqref{eq:poincare-group-multiplication}.

    Further, taking three transformations, $(\Lambda_1,a_1)$, $(\Lambda_2,a_2)$, $(\Lambda_3,a_3)$, and composing them using eq.~\eqref{eq:poincare-group-multiplication}, we get
    \begin{gather*}
        \left[ (\Lambda_1,a_1)(\Lambda_2,a_2) \right] (\Lambda_3,a_3) = (\Lambda_1\Lambda_2, a_1 + \Lambda_1 a_2) (\Lambda_3,a_3) = (\Lambda_1\Lambda_2\Lambda_3, a_1 + \Lambda_1 a_2 + (\Lambda_1 \Lambda_2)a_3) , \\
        (\Lambda_1,a_1)\left[(\Lambda_2,a_2)T(\Lambda_3,a_3)\right] = (\Lambda_1,a_1) (\Lambda_2\Lambda_3, a_2 + \Lambda_2 a_3) = (\Lambda_1\Lambda_2\Lambda_3, a_1+\Lambda_1(a_2 + \Lambda_2 a_3)) .
    \end{gather*}
 
    Since they lead to the same result, eq.~\eqref{eq:poincare-associative} is proved.

    The unit element is trivial, since by means of eq.~\eqref{eq:poincare-definition}, taking $(\1,0)$,
    \begin{equation*}
        x \to (\1,0) x = x.
    \end{equation*}

    Finally, to show the inverse element is given by eq.~\eqref{eq:poincare-inverse-element}, let's use eq.~\eqref{eq:poincare-group-multiplication} to compute
    \begin{equation*}
       (\Lambda,a) (\Lambda^{-1}, -\Lambda^{-1}a) = (\Lambda^{-1}\Lambda, a - \Lambda(\Lambda^{-1}a)) = (\1,0),
    \end{equation*}
    which is the group unit element, as expected.
\end{proof}

Basically, while $SO(1,3)$ relates orthonormal frames with coincident origins, elements of $ISO(1,3)$ also allow for the shift of origin.



%%%%%%%%%%%%%%% POINCARE ALGEBRA %%%%%%%%%%%%%%%%%
\section{Poincaré Algebra}
To work out the \emph{Poincaré algebra}, let's follow the approach of sec.~\ref{sec:lorentz-algebra}. Then, we consider the abstract group $ISO(1,3)^+$, which is now characterised by $10$ parameters. Without translations, we must recover Lorentz group. Therefore, we continue to denote the six generators referring to Lorentz transformations by $M^{\mu\nu} = -M^{\nu\mu}$, and their parameters by $\omega_{\mu\nu}=-\omega_{\nu\mu}$. Further, we call the four generators related to translations $P^\mu$, and their parameters $\epsilon_\mu$.

Here, $(\Lambda,a) \in ISO(1,3)^+$ will denote an element of the abstract Poincaré group, while $\Lambda \in SO(1,3)^+$ is a proper orthochronous Lorentz transformation and $a \in \R^4$ a translation.

Using the exponential map\footnote{\color{red}problem here, is it surjective?\color{black}}, any element $(\Lambda,a) \in ISO(1,3)^+$ can be written as
\begin{equation}
    (\Lambda,a) = e^{-\frac{i}{2} \omega_{\mu\nu} M^{\mu\nu} - i \epsilon_\sigma P^\sigma},
\end{equation}
with a conventional choice of constants.

Let's denote a generic representation of the Poincaré group with $U(\Lambda, a)$. The following properties are not necessarily referred to unitary representations, but at the end we'll be interested on them; from this the choice of notation. In order for it to satisfy the definition~\ref{def:representation}, it must inherit the transformation properties of the group, in particular eq.~\eqref{eq:poincare-group-multiplication} and eq.~\eqref{eq:poincare-inverse-element},
\begin{subequations}
\begin{align}
    U(\Lambda, a) U(\Lambda',a') &= U(\Lambda\Lambda', a + \Lambda a') , \label{eq:representation-poincare-group-multiplication} \\
    U^{-1} (\Lambda, a) &= U(\Lambda^{-1},-\Lambda^{-1}a) . \label{eq:representation-poincare-inverse-element}
\end{align}
\end{subequations}

Then, for infinitesimal transformations\footnote{See chapter $2$ of Weinberg~\cite{weinberg} to convince yourself this is possible. Here, we're assuming the exponential map is surjective. As discussed in sec.~\ref{sec:lie-groups-algebras} after eq.~\eqref{eq:exp-map}, it is a non-trivial fact to prove the surjectivity. However, we're not interested in such details, so we'll just ignore it. \color{red} Exponential vs product of exponentials, still to be solved \color{black}},
\begin{equation}\label{eq:infinitesimal-poicare-transformation}
    U(\Lambda, a) = e^{-\frac{i}{2} \omega_{\mu\nu} M^{\mu\nu} - i \epsilon_\sigma P^\sigma} \simeq \1 - \frac{i}{2} \omega_{\mu\nu} M^{\mu\nu} - i \epsilon_\sigma P^\sigma,
\end{equation}
where the explicit forms of $U(\Lambda,a)$ and the generators $M^{\mu\nu}$ and $P^\mu$ depend on the representation. Here, we're denoting the abstract generators and their representation with the same symbols, $M^{\mu\nu}$ and $P^\mu$, without further indices or specifications. From the context it should be straightforward to identify what we're talking about. At the end of the day, we'll be interested in representations, and further recall that the commutation relations defining the Lie algebra are representation independent.

The \emph{Poincaré algebra} is defined by
\begin{subequations}\label{eq:poincare-comm-relations}
    \begin{align}
        \comm{M^{\mu\nu}}{M^{\rho\sigma}} &= i ( \eta^{\nu\rho}M^{\mu\sigma} - \eta^{\mu\rho}M^{\nu\sigma} - \eta^{\mu\sigma}M^{\rho\nu} + \eta^{\sigma\nu}M^{\rho\mu}) \label{eq:comm-poincare-lorentz}\\
        \comm{P^\mu}{M^{\rho\sigma}} &= i (\eta^{\mu\rho} P^\sigma - \eta^{\mu\sigma} P^\rho),
        \label{eq:comm-poincare-lorentzmomentum}\\
        \comm{P^\mu}{P^\nu} &= 0 \label{eq:comm-poincare-momentum},
\end{align}
\end{subequations}
%%%%%%% begin proof
\begin{proof}
    Combining eq.~\eqref{eq:representation-poincare-group-multiplication} and eq.~\eqref{eq:representation-poincare-inverse-element}, we can write
    \begin{equation*}
        U(\Lambda, a) U(\Lambda',a') U^{-1}(\Lambda,a) =  = U(\Lambda \Lambda' \Lambda^{-1}, a + \Lambda a' - \Lambda \Lambda' \Lambda^{-1}a) .
    \end{equation*}
    If we consider infinitesimal transformations~\eqref{eq:infinitesimal-poicare-transformation} about the identity $U(\1,0)$, we have
    \begin{align*}
        U(\Lambda,a) &\simeq \1 -\frac{i}{2} \omega_{\mu\nu} M^{\mu\nu} - i \epsilon_\sigma P^\sigma, \\ 
        {U^{-1}}(\Lambda,a) &\simeq \1 +\frac{i}{2} \omega_{\mu\nu} M^{\mu\nu} + i \epsilon_\sigma P^\sigma .
    \end{align*}
    Substituting and keeping only linear terms of the parameters $\omega_{\mu\nu}$, $\omega'_{\mu\nu}$, $\epsilon_\sigma$, $\epsilon'_\sigma$, the left-hand side of the starting relations reads
    \begin{align*}
        &\quad \,\left[ \1 -\frac{i}{2} \omega_{\mu\nu} M^{\mu\nu} - i \epsilon_\alpha P^\alpha  \right] \left[ \1 -\frac{i}{2} \omega'_{\rho\sigma} M^{\rho\sigma} - i \epsilon'_\beta P^\beta  \right] \left[ \1 +\frac{i}{2} \omega_{\tau\xi} M^{\tau\xi} + i \epsilon_\gamma P^\gamma  \right] \\
        &=\left[ \1 - \frac{i}{2} \omega' M - i \epsilon' P - \frac{i}{2} \omega M - \frac{1}{4} \omega \omega' M^2 - \frac{1}{2} \omega \epsilon' M P - i \epsilon P - \frac{1}{2} \epsilon \omega' P M - \epsilon \epsilon' P^2 \right] \times \\
        &\quad\, \left[ \1 + \frac{i}{2} \omega M + i \epsilon P  \right] \\
        &= \1 - \frac{i}{2} \omega' M - i \epsilon' P - \frac{i}{2} \omega M - \frac{1}{4} \omega \omega' M^2 - \frac{1}{2} \omega \epsilon' M P - i \epsilon P - \frac{1}{2} \epsilon \omega' P M - \epsilon \epsilon' P^2 \\
        &\quad\quad +\frac{i}{2}\omega M + i \epsilon P +\frac{1}{4} \omega' \omega M^2 + \frac{1}{2} \omega' \epsilon M P +\frac{1}{2} \epsilon' \omega P M + \epsilon' \epsilon P^2 +\frac{1}{4} \omega^2 M^2 + \frac{1}{2} \omega \epsilon M P \\
        &\quad\quad -\frac{i}{8} \omega \omega' \omega M^3 - \frac{i}{4} \omega \omega' \epsilon M^2 P -\frac{i}{4} \omega \epsilon' \omega M P M - \frac{i}{2} \omega \epsilon' \epsilon M P^2 +\frac{1}{2} \epsilon \omega P M + \epsilon^2 P^2 \\
        &\quad\quad -\frac{i}{4} \epsilon \omega' \omega P M^2 - \frac{i}{2} \epsilon \omega' \epsilon P M P -\frac{i}{2} \epsilon \epsilon' \omega P^2 M - i \epsilon \epsilon' \epsilon P^3 \\
        &= \1 -\frac{i}{2} \omega' M - i \epsilon' P - \frac{1}{4} \omega_{\mu\nu} \omega'_{\rho\sigma} \comm{M^{\mu\nu}}{M^{\rho\sigma}} - \epsilon_\alpha \epsilon'_\beta \comm{P^\alpha}{P^\beta} \\
        &\quad\quad +\frac{1}{2} \omega_{\mu\nu} \epsilon_\alpha \{ M^{\mu\nu}, P^\alpha \} - \frac{1}{2} \omega'_{\rho\sigma} \epsilon_\alpha \comm{P^\alpha}{M^{\rho\sigma}} + \frac{1}{2} \omega_{\mu\nu} \epsilon'_\beta \comm{P^\beta}{M^{\mu\nu}}.
    \end{align*}

    For the right-hand side, we use $\tensor{\Lambda}{^\mu_\nu} \simeq \delta^\mu_\nu + \tensor{\omega}{^\mu_\nu}$ to compute, at infinitesimal level,
    \begin{align*}
        \tensor{(\Lambda\Lambda'\Lambda^{-1})}{^\mu_\nu} &= \tensor{\Lambda}{^\mu_\rho}\tensor{(\Lambda')}{^\rho_\kappa}\tensor{(\Lambda^{-1})}{^\kappa_\nu} \simeq (\delta^\mu_\rho + \tensor{\omega}{^\mu_\rho})(\delta^\rho_\kappa + \tensor{{\omega'}}{^\rho_\kappa})(\delta^\kappa_\nu - \tensor{\omega}{^\kappa_\nu}) \\
        &= \delta^\mu_\nu + \tensor{{\omega'}}{^\mu_\nu} + \tensor{\omega}{^\mu_\rho}\tensor{{\omega'}}{^\rho_\nu} - \tensor{{\omega'}}{^\mu_\kappa}\tensor{\omega}{^\kappa_\nu} + O(\omega^2) + O({\omega'}^2)
    \end{align*}
    Then, using also $a^\mu \simeq \epsilon^\mu$, we may compute
    \begin{align*}
        (\Lambda a')^\mu = \tensor{\Lambda}{^\mu_\nu} a'^\nu &\simeq (\delta^\mu_\nu + \tensor{\omega}{^\mu_\nu}) \epsilon'^\nu = \epsilon'^\mu + \tensor{\omega}{^\mu_\nu} \epsilon'^\nu , \\
        \tensor{(\Lambda\Lambda'\Lambda^{-1})}{^\mu_\nu} a^\nu &\simeq \epsilon^\mu + \tensor{{\omega'}}{^\mu_\nu} \epsilon^\nu + \tensor{\omega}{^\mu_\kappa}\tensor{{\omega'}}{^\kappa_\nu} \epsilon^\nu - \tensor{{\omega'}}{^\mu_\kappa}\tensor{\omega}{^\kappa_\nu} \epsilon^\nu ,
    \end{align*}
    from which
    \begin{equation*}
        (a + \Lambda a' - \Lambda \Lambda' \Lambda^{-1}a)^\mu \simeq \epsilon'^\mu + \tensor{\omega}{^\mu_\nu} \epsilon'^\nu - \tensor{{\omega'}}{^\mu_\nu} \epsilon^\nu + \tensor{{\omega'}}{^\mu_\kappa}\tensor{\omega}{^\kappa_\nu} \epsilon^\nu - \tensor{\omega}{^\mu_\kappa}\tensor{{\omega'}}{^\kappa_\nu} \epsilon^\nu
    \end{equation*}

    Therefore, using eq.~\eqref{eq:infinitesimal-poicare-transformation}, we can expand $U(\Lambda\Lambda'\Lambda^{-1},a + \Lambda a' - \Lambda\Lambda'\Lambda^{-1}a)$ about the identity $U(\1,0)$, obtaining
    \begin{align*}
        U(\Lambda\Lambda'\Lambda^{-1},a + \Lambda a' - \Lambda\Lambda'\Lambda^{-1}a) &\simeq \1 - \frac{i}{2} (\omega'_{\mu\nu} + \omega_{\mu\kappa}\tensor{{\omega'}}{^\kappa_\nu}-\omega'_{\mu\kappa} \tensor{\omega}{^\kappa_\nu})M^{\mu\nu} \\
        &- i \left( \epsilon'_\mu +\omega_{\mu\nu} \epsilon'^\nu - \omega'_{\mu\nu} \epsilon^\nu + \omega'_{\mu\kappa}\tensor{\omega}{^\kappa_\nu} \epsilon^\nu - \omega_{\mu\kappa} \tensor{{\omega'}}{^\kappa_\nu} \epsilon^\nu \right) P^\mu \\
        &\simeq \1 - \frac{i}{2} (\omega'_{\mu\nu} + \omega_{\mu\kappa}\tensor{{\omega'}}{^\kappa_\nu}-\omega'_{\mu\kappa} \tensor{\omega}{^\kappa_\nu})M^{\mu\nu} \\
        &- i \left( \epsilon'_\mu +\omega_{\mu\nu} \epsilon'^\nu - \omega'_{\mu\nu} \epsilon^\nu\right) P^\mu ,
    \end{align*}
    which we can compare to the previous expansion. Indeed, comparing the two we obtain
    \begin{align*}
        &\;\quad \1 -\frac{i}{2} \omega' M - i \epsilon' P 
        - \frac{1}{4} \omega_{\mu\nu} \omega'_{\rho\sigma} \comm{M^{\mu\nu}}{M^{\rho\sigma}} 
        - \epsilon_\alpha \epsilon'_\beta \comm{P^\alpha}{P^\beta} \\
        &\;\quad\;\; +\frac{1}{2} \omega_{\mu\nu} \epsilon_\alpha \{ M^{\mu\nu}, P^\alpha \} - \frac{1}{2} \omega'_{\rho\sigma} \epsilon_\alpha \comm{P^\alpha}{M^{\rho\sigma}} + \frac{1}{2} \omega_{\mu\nu} \epsilon'_\beta \comm{P^\beta}{M^{\mu\nu}} \\
        &\overset{!}{=} \1 - \frac{i}{2} \omega'_{\mu\nu} M^{\mu\nu} - \frac{i}{2} \omega_{\mu\kappa}\tensor{{\omega'}}{^\kappa_\nu} M^{\mu\nu} + \frac{i}{2} \omega'_{\mu\kappa} \tensor{\omega}{^\kappa_\nu} M^{\mu\nu} \\
        &\quad\;\;\;  - i \epsilon'_\mu P^\mu - i \omega_{\mu\nu} \epsilon'^\nu P^\mu + i \omega'_{\mu\nu} \epsilon^\nu P^\mu   .
    \end{align*}
    Let's then compare the two sides order by order:
    \begin{itemize}
        \item From the $\omega\omega'$ term we can read the commutator $\comm{M^{\mu\nu}}{M^{\rho\sigma}}$,
    \begin{align*}
       - \frac{1}{4} \omega_{\mu\nu} \omega'_{\rho\sigma} \comm{M^{\mu\nu}}{M^{\rho\sigma}} = -\frac{i}{2} \omega_{\mu\kappa} \tensor{{\omega'}}{^\kappa_\nu} M^{\mu\nu} + \frac{i}{2} \omega'_{\mu\kappa} \tensor{\omega}{^\kappa_\nu} M^{\mu\nu} .
    \end{align*}
    Now, we can rewrite the right-hand side in the following way, subtracting a piece to make everything antisymmetric in $\mu\nu$:
    \begin{align*}
        -\frac{i}{2} \omega_{\mu\kappa} \tensor{{\omega'}}{^\kappa_\nu} M^{\mu\nu} + \frac{i}{2} \omega'_{\mu\kappa} \tensor{\omega}{^\kappa_\nu} M^{\mu\nu} = &-\frac{i}{4} \omega_{\mu\nu} \omega'_{\rho\sigma}( \eta^{\nu\rho}M^{\mu\sigma} - \eta^{\mu\rho}M^{\nu\sigma} ) \\
        &+ \frac{i}{4} \omega_{\mu\nu}\omega'_{\rho\sigma} ( \eta^{\mu\sigma}M^{\rho\nu} -\eta^{\nu\sigma}M^{\rho\mu})
    \end{align*}
    Therefore, by comparing, we obtain
    \begin{equation*}
        \comm{M^{\mu\nu}}{M^{\rho\sigma}} = i ( \eta^{\nu\rho}M^{\mu\sigma} - \eta^{\mu\rho}M^{\nu\sigma} - \eta^{\mu\sigma}M^{\rho\nu} + \eta^{\nu\sigma}M^{\rho\mu}),
    \end{equation*}
    which is exactly eq.~\eqref{eq:comm-poincare-lorentz}.
    \item Looking at the $\omega' \epsilon$ terms, we get
    \begin{equation*}
         - \frac{1}{2} \omega'_{\rho\sigma} \epsilon_\alpha \comm{P^\alpha}{M^{\rho\sigma}} = i \omega'_{\mu\nu} \epsilon^\nu P^\mu .
    \end{equation*}
    Relabelling and making sure that the right-hand side is antisymmetric with respect to the indices of $\omega'$ (which are $\rho\sigma$ below), we get
    \begin{equation*}
        -\frac{1}{2} \epsilon_{\mu} \omega'_{\rho\sigma}\comm{P^\mu}{M^{\rho\sigma}} = \frac{i}{2} \epsilon_\mu \omega'_{\rho\sigma}  ( \eta^{\mu\sigma} P^\rho - \eta^{\mu\rho}P^{\sigma} ) ,
    \end{equation*}
    which leads to
    \begin{equation*}
        \comm{P^\mu}{M^{\rho\sigma}} = i (\eta^{\mu\rho} P^\sigma - \eta^{\mu\sigma} P^\rho),
    \end{equation*}
    which is eq.~\eqref{eq:comm-poincare-lorentzmomentum}.
    \item Since there's no $\epsilon \epsilon'$ on the right-hand side, we immediatly have
    \begin{equation*}
        \comm{P^\mu}{P^\nu} = 0,
    \end{equation*}
    which is eq.~\eqref{eq:comm-poincare-momentum}.
\end{itemize}
\end{proof}
%%%%%%%% end proof


It is convenient to rearrange the Lorentz generators as in eq.~\eqref{eq:redef-lorentz-gen}, which is reported below for clarity,
\begin{equation}
    J^i = \frac{1}{2} \epsilon^{ijk} M^{jk} \iff M^{ij} = \epsilon^{ijk} J^k, \quad K^i = M^{i0} .
\end{equation}

Then, the Poincaré commutation relations~\eqref{eq:poincare-comm-relations} take the form

\begin{subequations}
\label{eq:poincare-commutation-boosts-rotations}
\noindent\centering
    \begin{minipage}{0.48\textwidth}
        \begin{align}
        \comm{J^i}{J^j} &= i \epsilon_{ijk} J^k, \\
        \comm{J^i}{K^j} &= i \epsilon_{ijk} K^k ,\\
        \comm{K^i}{K^j} &= -i \epsilon_{ijk} J^k, \\
        \comm{J^i}{P^j} &= i \epsilon_{ijk} P^k ,\\
        \comm{K^i}{P^j} &= i \delta_{ij} P_0,
        \end{align}
    \end{minipage}
    \hfill
    \begin{minipage}{0.48\textwidth}
    \label{eq:bell_states}
        \begin{align}
        \comm{K^i}{P_0} &= i P^i ,\\
        \comm{P^i}{P^j} &= 0, \\
        \comm{J^i}{P_0} &= 0, \\
        \comm{P^i}{P_0} &= 0 .
        \end{align}
    \end{minipage}\bigskip
    \end{subequations}

From~\eqref{eq:poincare-commutation-boosts-rotations} we see that boosts and rotations generally do not commute unless the boost and rotation axes coincide. Moreover, $P_0$ (which becomes the Hamilton operator in the quantum theory) commutes with rotations and spatial translations but not with boosts and therefore the eigenvalues of K cannot be used for labelling physical states.


%%%%%%%%%%%%%%% CASIMIR OPERATORS %%%%%%%%%%%%%%%%%
\section{Casimir Operators}

%%%%%%%%%%%%%%% UNITARY REPRESENTATIONS %%%%%%%%%%%%%%%%%
\section{Unitary Representations of the Poincaré Group}
Moving to the quantum theory, as already discussed, \emph{Wigner's theorem} states that continuous symmetries must be implemented by unitary operators on the state space. Since Poincaré group is non-compact, due to the presence of the boosts, all unitary representations must be infinite dimensional. In quantum field theory, fields are promoted to operators which act on the Fock space. The classical action which describes the system is constructed to be Poincaré invariant, therefore, by Noether's theorem, there will be conserved charges, which will be functions of the fields. Those, after quantisation, become hermitian operators on the Fock space and furnish a unitary representation of the Poincaré algebra on the state space. In particular, calling $\hat{M}^{\mu\nu}$ the generators of Lorentz transformations and $\hat{P}^\mu$ those of the translations, we'll have, for an infinitesimal transformation (cfr.~\eqref{eq:infinitesimal-poicare-transformation})
\begin{equation}
    U(\Lambda,a) = e^{\frac{i}{2} \omega_{\mu\nu} \hat{M}^{\mu\nu} + i \epsilon_\mu \hat{P}^\mu} \simeq 1 + \frac{i}{2} \omega_{\mu\nu} \hat{M}^{\mu\nu} + i \epsilon_\mu \hat{P}^\mu .
\end{equation}

To find out the irreducible state space, we want to work with operators which commute with each other, so that states in different such states won't transform into each other. Since $\comm{\hat{P}^\mu}{\hat{P}^\nu} = 0$, it's natural to express physical state vectors in terms of eigenvectors of $\hat{P}^\mu$. We'll further denote with $\sigma$ all the other quantum numbers necessary to describe the state. We're interested in \emph{one-particle states}, which in QFT are defined, in a free theory, as eigenstates of the number operator of eigenvalue one. For such a state, the labels $\sigma$ must be \emph{discrete}. Therefore, we start from
\begin{equation}\label{eq:unitary-poincare-eigenstate-momentum}
    \hat{P}^\mu \ket{p,\sigma} = p^\mu \ket{p,\sigma} .
\end{equation}

Pure translations form an abelian subgroup of the Poincaré group and are represented, on the state space, by
\begin{equation}\label{eq:unitary-poincare-translations}
    U(1, a) = e^{-i \epsilon_\mu \hat{P}^\mu} .
\end{equation}
Combining~\eqref{eq:unitary-poincare-eigenstate-momentum} with~\eqref{eq:unitary-poincare-translations}, we can explicitly see how a state transforms under translations
\begin{equation}\label{eq:unitary-poincare-state-translation}
    U(1,a)\ket{p,\sigma} = e^{-i \epsilon_\mu \hat{P}^\mu} \ket{p,\sigma} = e^{-i \epsilon_\mu p^\mu} \ket{p,\sigma} .
\end{equation}

We must, then find, how such states transform under a Lorentz transformation. We can easily see that acting on $\ket{p,\sigma}$ with a pure Lorentz transformation $U(\Lambda,0) \coloneq U(\Lambda)$, produces an eigenstate of $\hat{P}^\mu$ of eigenvalue $\Lambda p$:
\begin{equation}
    \hat{P}^\mu U(\Lambda) \ket{p,\sigma} = \tensor{\Lambda}{^\mu_\rho} p^\rho U(\Lambda) \ket{p,\sigma}
\end{equation}
\begin{proof}
    \begin{equation*}
    \begin{split}
        \hat{P}^\mu U(\Lambda) \ket{p,\sigma} &= U(\Lambda) \left[U^{-1} \hat{P}^\mu U(\Lambda)\right] \ket{p,\sigma} = U(\Lambda) \left[ \tensor{(\Lambda^{-1})}{_\rho^\mu} \hat{P}^\rho \right] \ket{p,\sigma}  \\
        &= \tensor{\Lambda}{^\mu_\rho} p^\rho U(\Lambda) \ket{p,\sigma} .
    \end{split}
    \end{equation*}
    where we used $U^{-1}U=\1$ in the first equality, eq.\color{red}\dots\dots\color{black} in the second, and finally eq.~\eqref{eq:unitary-poincare-eigenstate-momentum}, together with the well-known relation $\tensor{(\Lambda^{-1})}{_\rho^\mu} = \tensor{\Lambda}{^\mu_\nu}$
\end{proof}

Hence, because of eq.~\eqref{eq:unitary-poincare-eigenstate-momentum}, $U(\Lambda)\ket{p,\sigma}$ must be a linear combination of $\ket{\Lambda p,\sigma}$:
\begin{equation}\label{eq:temp-1}
    U(\Lambda)\ket{p,\sigma} =  \sum_{\sigma'} C_{\sigma' \sigma} (\Lambda, p) \ket{\Lambda p,\sigma'}.
\end{equation}

In general, by using suitable linear combinations of the $\ket{p,\sigma}$, it may be possible to choose the $\sigma$ labels in such a way that the matrix $C_{\sigma' \sigma} (\Lambda, p)$ is block-diagonal. In this case, $C_{\sigma' \sigma} (\Lambda, p)$ would be a reducible representation, and the $\ket{p,\sigma}$ with $\sigma$ in any given block, will transform within the same block, i.e., according to an irreducible representation. 

Therefore, it's natural to identity the states of a \emph{specific particle} as those transforming under irreducible representations of the Poincaré group. This approach ensures that a Poincaré transformation can't change the type of particle. To construct such irreducible representation, we'll make use of the Casimir operators of the group. Roughly speaking, the eigenvalues of the Casimir operators labels different irreducible representations, which correspond to Poincaré transformations acting on \emph{different types} of particles. Since those operators commute with all the generators of the group (and so with each group element via the exponential map), their eigenvalues remain constant under any Poincaré transformation, preserving the identification of the particle type. Moreover, states that are transformed into one another within an irreducible state space are interpreted as different states of the same particle.

It's important to emphasise, however, that the eigenvalues of the Casimir operators do \emph{not} uniquely characterise an irreducible representation. For example, as we will see, the Casimirs vanish for all massless (helicity) representations. Since a comprehensive mathematical treatment would be quite technical and wouldn't provide additional physical insight, we'll stick to the interpretation of an irreducible representation as one that can't change the labels which characterise the type of particle. 

Additionally, let's note that physicists often employ an abuse of language by referring to irreducible representations as the states. Keep in mind, however, that the representation is actually the operator acting on the state.

Let's go back to~\eqref{eq:temp-1} and to the problem of working out the structure of the coefficients $C_{\sigma'\sigma}(\Lambda,p)$ in irreducible representations of the Poincaré group. In order to do that, note that the only functions of $p^\mu$ that are left invariant by all transformations $\tensor{\Lambda}{^\mu_\nu} \in SO(1,3)^+$ are $p^2 = \eta_{\mu\nu} p^\mu p^\nu$ and, for $p^2 \leq 0$, also the sign of $p^0$. Hence, for each value of $p^2$, and (for $p^2 \leq 0$) each sign of $p^0$, one can choose a standard four-momentum, say $q^\mu$, and express any $p^\mu$ of this class as
\begin{equation}
    p^\mu = \tensor{L}{^\mu_\nu}(p)q^\nu,
\end{equation}
where $\tensor{L}{^\mu_\nu}$ is some standard proper orthochronous Lorentz transformation that depends on $p^\mu$, and also implicitly on our choice of $q^\mu$. One can then define the states $\ket{p,\sigma}$ of momentum $p^\mu$ by
\begin{equation}\label{eq:temp-2}
    \ket{p,\sigma} \coloneq N(p) U(L(p)) \ket{q,\sigma},
\end{equation}
where $N(p)$ is a numerical normalisation factor. Now, acting on~\eqref{eq:temp-2} with a generic Lorentz transformation $U(\Lambda)$, we find
\begin{equation}\label{eq:temp-3}
    U(\Lambda) \ket{p,\sigma} = N(p) U(L(\Lambda p)) U(L^{-1}(\Lambda p) \Lambda L(p)) \ket{q,\sigma}
\end{equation}
\begin{proof}
    \begin{equation*}
    \begin{split}
        U(\Lambda) \ket{p,\sigma} &= N(p) U(\Lambda) U(L(p)) \ket{q,\sigma} = N(p) U(\Lambda L(p)) \ket{q,\sigma}\\
         &= N(p) U(L(\Lambda p)) U(L^{-1}(\Lambda p) \Lambda L(p)) \ket{q,\sigma} ,
    \end{split}
    \end{equation*}
    where we used the (representation) group composition rule~\eqref{eq:representation-poincare-group-multiplication} two times, together with $L L^{-1} = \1$.
\end{proof}

 The point of this last step is that the Lorentz transformation $L^{-1}(\Lambda p) \Lambda L(p)$ takes $k$ to the following values
\begin{equation}
    k \to p \to \Lambda p \to k .
\end{equation}
Therefore, $L^{-1}(\Lambda p) \Lambda L(p)$ belongs to the \emph{little  group} or \emph{stabiliser group}, i.e., the subgroup made of transformations which leave $q^\mu$ invariant,
\begin{equation}\label{eq:little-group}
    \tensor{W}{^\mu_\nu} q^\nu = q^\mu .
\end{equation}

Using eq.~\eqref{eq:temp-1}, for each $W$ in the little group satisfying~\eqref{eq:little-group}, we have
\begin{equation}
    U(W) \ket{k,\sigma} = \sum_{\sigma'} D_{\sigma'\sigma}(W) \ket{k,\sigma'} ,
\end{equation}
where the coefficients $D(W)$ furnish a representation of the little group. Then, for any $W,\bar{W}$ in the little group,
\begin{equation*}
\begin{split}
    \sum_{\sigma'} D_{\sigma'\sigma} (\bar{W}W) \ket{k,\sigma'} &= U(\bar{W}W)  \ket{k,\sigma} = U(\bar{W}) U(W) \ket{k,\sigma} = U(\bar{W}) \sum_{\sigma''} D_{\sigma''\sigma}(W) \ket{k,\sigma''} \\ &= \sum_{\sigma' \sigma''} D_{\sigma''\sigma}(W) D_{\sigma'\sigma''}(\bar{W}) \ket{k,\sigma'} ,
\end{split}
\end{equation*}
and so
\begin{equation}
    D_{\sigma'\sigma}(\bar{W}W) = \sum_{\sigma''} D_{\sigma'\sigma''} (\bar{W}) D_{\sigma'' \sigma}(W) .
\end{equation}

In particular, for $W(\Lambda,p) \equiv L^{-1}(\Lambda p) \Lambda L(p)$, eq.~\eqref{eq:temp-3} becomes
\begin{equation}
   U(\Lambda) \ket{p,\sigma} = N(p) \sum_{\sigma'} D_{\sigma'\sigma}(W(\Lambda,p))U(L(\Lambda p)) \ket{q,\sigma'} ,
\end{equation}
or, recalling the definition~\eqref{eq:temp-2},
\begin{equation}\label{eq:induced-representation}
    U(\Lambda) \ket{p,\sigma} = \frac{N(p)}{N(\Lambda p)} \sum_{\sigma'} D_{\sigma'\sigma} (W(\Lambda, p)) \ket{\Lambda p, \sigma'} .
\end{equation}

We won't go into the details of the normalisation here. We simply note that, by the usual orthonormalization procedure in quantum mechanics, we may choose states with standard momentum $q^\mu$ to be orthonormal, in the sense that
\begin{equation}
    \braket{k',\sigma'}{k,\sigma} = \delta^{(3)}(\vec{k}' - \vec{k}) \delta_{\sigma'\sigma} ,
\end{equation}
and by suitable choices of the coefficients in~\eqref{eq:induced-representation}, we can normalise states of arbitrary momentum as
\begin{equation}
    \braket{p',\sigma'}{p,\sigma} = \delta^{(3)}(\vec{p}' - \vec{p}) \delta_{\sigma'\sigma} .
\end{equation}

By eq.~\eqref{eq:induced-representation}, the problem of determining the coefficients $C_{\sigma'\sigma}$ in the transformation rule\color{red}\dots\color{black} has been reduced to the problem of determining the coefficients $D_{\sigma'\sigma}$. In other words, the problem of determining all possible irreducible representations of the Poincaré group has been reduced to the problem of finding all possible irreducible representations of the little group, depending on the class of momentum to which $q^\mu$ belongs. This approach of deriving the representations of a group from the representations of its little group is called \emph{method of induced representations}.

