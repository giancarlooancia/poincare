\section{Symmetries in Quantum Mechanics}\label{sec:symmetries-qm}
In this section, we review the concepts of states and observables in quantum mechanics, with a focus on the role of projective representations as opposed to standard representations.

Let us begin by outlining some fundamental aspects of quantum mechanics.
\begin{itemize}
    \item \emph{Physical states} are represented by \emph{rays} in a Hilbert space $\H$, which is a complex vector space with a \emph{scalar product} $\langle \cdot, \cdot \rangle$, defined by
    \begin{subequations}\label{eq:scalar-product}
        \begin{align}
        \scalar{\phi}{\psi} &= \scalar{\psi}{\phi}^* , \\
        \scalar{\phi}{\xi_1 \psi_1 + \xi_2 \psi_2} &= \xi_1 \scalar{\phi}{\psi_1} + \xi_2 \scalar{\phi}{\psi_2} ,\\
        \scalar{\eta_1 \phi_1 + \eta_2 \phi_2}{\psi} &= {\eta_1}^* \scalar{\phi_1}{\psi} + {\eta_2}^* \scalar{\phi_2}{\psi} ,
        \end{align}
        \end{subequations}
    with $\phi,\psi \in \H$. The \emph{norm} is defined as $\mynorm{\psi} \coloneq \braket{\psi}{\psi} \geq 0$, and it vanishes if and only if $\psi \equiv 0$. A \emph{ray} $\ray$ is a set of normalised vectors (satisfying $\braket{\psi}{\psi} = 1$), where $\psi, \psi' \in \ray$ if $\psi' = \xi \psi$, for some $\xi \in \C$ with $\abs{\xi} = 1$.
    \item \emph{Observables} are represented by \emph{hermitian operators} $A$ on $\H$, defined by the properties
    \begin{subequations}
    \begin{align}
        A (\xi \psi + \eta \phi) &= \xi A\psi + \eta A \phi ,\\
        A &= A^\dagger ,
    \end{align}
    \end{subequations}
    where the adjoint of an operator $A$ is defined by
    \begin{equation}\label{eq:def-adjoint}
        \scalar{\phi}{A^\dagger \psi} = \scalar{A\phi}{\psi} = \scalar{\psi}{A\phi}^* .
    \end{equation}

    Let's consider a physical state represented by a ray $\ray$, and an observable represented by the hermitian operator $A$. Then, the state has a definite value of $\alpha$ for the observable if vectors $\psi \in \ray$ are \emph{eigenstates} of $A$ of \emph{eigenvalue} $\alpha$:
    \begin{equation}
        A\psi = \alpha \psi, \quad \textup{for} \; \psi \in \ray .
    \end{equation}

    If the operator is hermitian, i.e., $A^\dagger = A$, then $\alpha \in \R$ and eigenstates with different eigenvalues are orthogonal with respect to each other.
    \item If a system is in a state represented by a ray $\ray$, and an experiment tests if it is in any one of the  mutually orthogonal rays\footnote{A pair of rays $\ray_1, \ray_2$ are orthogonal if $\scalar{\psi_1}{\psi_2} = 0, \; \forall \psi_1 \in \ray_1,  \psi_2 \in \ray_2$} $\ray_1, \ray_2, \dots$, then the probability of finding it in the state represented by $\ray_n$ is 
    \begin{equation}
        P(\ray \to \ray_n) = \abs{\scalar{\psi}{\psi_n}}^2 ,
    \end{equation}
    where $\psi \in \ray$ and $\psi_n \in \ray_n$. If $\psi_n$ form a complete set, then $\sum_n P(\ray \to \ray_n) = 1$.
\end{itemize}

Now, a \emph{necessary} condition for a transformation acting on rays to be a \emph{symmetry}, is that, if an observer $O$ sees a state represented by a ray $\ray$, or $\ray_1$, or $\ray_2$, \dots, then another observer $O'$ sees the same system in a different state, represented by a ray $\ray'$, or $\ray'_1$, or $\ray'_2$, \dots, but the two must find the same probabilities:
\begin{equation}
    P(\ray \to \ray_n) = P(\ray' \to \ray'_n) .
\end{equation}

A fundamental result, \emph{Wigner's theorem}, states that for any such rays transformation, $T \colon \ray \to \ray'$, there exists an operator $U$ on the Hilbert space, such that if $\psi \in \ray$, then $U\psi \in \ray'$, with $U$ \emph{unitary} and \emph{linear}\footnote{To be honest, it could be \emph{antiunitary} and \emph{antilinear} as well, but this last case is not of physical interest.}
\begin{subequations}
\begin{align}
    \scalar{U\psi}{U\psi} &= \scalar{\psi}{\psi} \label{eq:def-unitary} ,\\ 
    U(\xi \psi + \eta \psi) &= \xi U \psi + \eta U \psi .
\end{align}
\end{subequations}
Recalling the definition of an adjoint operator~\eqref{eq:def-adjoint}, the unitarity condition~\eqref{eq:def-unitary} can be written as
\begin{equation}
    U^\dagger = U^{-1} .
\end{equation}

The identity operator $U = \1$ trivially represents a symmetry, preserving the ray $R$. An infinitesimal symmetry transformation close to the identity is represented by a unitary operator near $U = \1$:
\begin{equation}
    U = 1 + i \epsilon T ,
\end{equation}
where $\epsilon$ is a real infinitesimal parameter. For $U$ to be unitary, $T$ must be both \emph{linear} and \emph{hermitian}, qualifying it as a potential observable. In physics, most observables arise in this way from symmetry transformations.

It is straightforward to verify that symmetry transformations from one ray to another satisfy the group axioms~\ref{def:group-axioms}. Indeed, given $T_1 \colon \ray_n \to \ray'_n$ and $T_2 \colon \ray'_n \to \ray''_n$, then $T_2 T_1$ is another symmetry transformation, $T_2 T_1 \colon \ray_n \to \ray''_n$. The inverse will be $T^{-1} \colon \ray'_n \to \ray_n$ and the identity $T=1$ leaves the ray unchanged.

The unitary operators $U(T)$ corresponding to these symmetry transformations have properties that mirror the group structure (see def.~\ref{def:representation}), but unlike the symmetry transformations themselves, the operators $U(T)$ act on vectors in Hilbert space, rather than on rays. If $T_1 \colon \ray_n \to \ray'_n$, then $U(T_1)$ acts on $\psi_n \in \ray_n$ and gives a vector $U(T_1)\psi_n \in \ray'_n$. Similarly, if $T_2 \colon \ray'_n \to \ray''_n$, then $\ray'_n \ni U(T_1)\psi_n \mapsto U(T_2)U(T_1)\psi_n \in \ray''_n$. However, $U(T_2 T_1)$ is also in this ray, since $T_2 T_1 \colon \ray_n \to \ray''_n$, so the vectors $U(T_2)U(T_1)\psi_n$ and $U(T_2T_1)\psi_n$ must differ by a phase factor $\phi_n (T_2, T_1)$
\begin{equation}
    U(T_2)U(T_1)\psi_n = e^{i\phi (T_2, T_1)} U(T_2T_1)\psi_n ,
\end{equation}
where we dropped the index $n$ for the phase since one can prove that it doesn't depend on the state (see Weinberg~\cite{weinberg}).

This type of representation is called \emph{projective representation}. Unlike a linear representation~\eqref{eq:representation-property}, it includes a phase factor. Turning back to the notation of sec.~\ref{sec:group-representation}, a symmetry group $G$ acts on the Hilbert space $\H$ via a unitary operator $\rho(g) \colon \H \to \H$, for $g \in G$, which defines a projective representation:
\begin{equation}
    \rho(g) \rho(h) = e^{i\phi(g,h)} \rho(g \circ h), \quad \phi(g,h) \in R.
\end{equation}

Therefore, in a quantum theory, we're interested in \emph{unitary projective representations} of a symmetry group $G$. Then, according to theorem~\ref{th:unitary-rep}, \emph{all} unitary projective representations of $G$ can be derived from unitary \emph{linear} representations of the universal covering group $\tilde{G}$. These, in turn, arise from representations of the Lie algebra. Further, by theorem~\ref{th:non-compact-group-rep}, since the Poincaré group is non-compact, we must study infinite dimensional representations in order to have unitary ones, which can then be associated with physical observables.