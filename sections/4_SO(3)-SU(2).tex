\section{The Groups \texorpdfstring{$SO(3)$}{SO(3)} and \texorpdfstring{$SU(2)$}{SU(2)} and their Representations}
In this section we review the main properties of the Lie groups $SO(3)$ and $SU(2)$ and their algebra, since it'll be useful to study the representations of the Lorentz group.

%%%%%%%%%%%%%%%%%%%%%%%%%%% SO(3) GROUP %%%%%%%%%%%%%%%%%%%%%%%%%%%
\subsection{The Group \texorpdfstring{$SO(3)$}{SO(3)}}
\color{red} There's an error here, concerning the exponential map and the generators. What it's written refers to the exponential map with no $i$ factor. Taking into consideration, we should have $i$ factors on infinitesimal rotation and the algebra should be made of hermitian matrices. \color{black}

In its fundamental representation, the \emph{special unitary group} $SO(n)$ is composed of $n \times n$ invertible matrices which satisfy
\begin{equation}\label{eq:so3-def}
    R \in SO(n) : R^T R = R R^T = \1, \; \det(R) = 1 .
\end{equation}
It's then easy to see that the dimension is $\dim(SO(n)) = \frac{n(n-1)}{2}$. We're interested, in particular, in the case $n=3$, which is the $3$ dimensional $SO(3)$ group.

To determine the Lie algebra $\mathfrak{so}(3)$, note that any $R \in SO(3)$ can be parametrized by three Euler angles or, equivalently, by specifying a rotation axis and an angle around it. In particular, if we consider a unit vector $\vec{n} \in \R^3$, normalised such that $\vec{n} \cdot \vec{n} = 1$, and we call $\theta$ the angle around it, the identity is given by $R(\theta = 0, \vec{n}) = \1$. Then, for an infinitesimal angle $\delta \theta$, the action on an arbitrary vector $\vec{x} \in \R^3$ is an infinitesimal rotation given by
\begin{equation}
    R(\delta \theta, \vec{n}) \vec{x} = \vec{x} + \delta \theta \vec{n} \times \vec{x} = \vec{x} + \delta \theta 
    \begin{pmatrix}
        n_2 x_3 - n_3 x_2 \\
        n_3 x_1 - n_1 x_3 \\
        n_1 y_2 - n_2 y_1
    \end{pmatrix}.
\end{equation}

If we now pich $\vec{n}$ to be the Euclidean basis vectors, we obtain
\begin{subequations}\label{eq:infinitesimal-rotation}
\begin{align}
    R(\delta \theta, \vec{n} = (1,0,0)^T) \vec{x} &= \vec{x} + \delta \theta \begin{pmatrix} 0 \\ -x_3 \\ x_2 \end{pmatrix} = \vec{x} + \delta \theta 
    \begin{pmatrix}
        0 & 0 & 0 \\
        0 & 0 & -1 \\
        0 & 1 & 0
    \end{pmatrix}
    \vec{x} \coloneq \vec{x} + \delta \theta L_1 \vec{x} ,\\
    R(\delta \theta, \vec{n} = (0,1,0)^T) \vec{x} &= \vec{x} + \delta \theta \begin{pmatrix} x_3 \\ 0 \\ -x_1 \end{pmatrix} = \vec{x} + \delta \theta 
    \begin{pmatrix}
        0 & 0 & 1 \\
        0 & 0 & 0 \\
        -1 & 0 & 0
    \end{pmatrix}
    \vec{x} \coloneq \vec{x} + \delta \theta L_2 \vec{x} ,\\
    R(\delta \theta, \vec{n} = (0,0,1)^T) \vec{x} &= \vec{x} + \delta \theta \begin{pmatrix} -x_2 \\ x_1 \\ 0 \end{pmatrix} = \vec{x} + \delta \theta 
    \begin{pmatrix}
        0 & -1 & 0 \\
        1 & 0 & 0 \\
        0 & 0 & 0
    \end{pmatrix}
    \vec{x} \coloneq \vec{x} + \delta \theta L_3 \vec{x} ,\\
\end{align}
\end{subequations}

By means of~\eqref{eq:generators-rep}, the matrices $L_i$ in eq.~\eqref{eq:infinitesimal-rotation} correspond to a basis of the Lie algebra $\mathfrak{so}(3)$. Indeed, $SO(3)$, is the connected component of $O(3)$, and the group is compact, as can be seen from the fact that the rotation angles can only take finite values. Therefore, as discussed in section~\ref{sec:lie-groups-algebras} after the definition~\ref{def:exponential-map}, the exponential map is surjective and each group element can be written as the exponential of a linear combination of the generators, as showed in eq.~\eqref{eq:exp-map-rep}. In this case, we're in the fundamental representation and the generators are represented by $L_i$.

Using the defining property~\eqref{eq:so3-def} and eq.~\eqref{eq:exp-map-rep} together with BCH formula~\eqref{eq:BCH-formula}, we get
\begin{equation*}
    \1 = R(\theta, \vec{n})^T R(\theta, \vec{n}) = \exp({i \theta \sum_i n_i \cdot ({L}_i^T + L_i) + \dots}) .
\end{equation*}
Then, for an infinitesimal transformation, it must be $L^T_i + L_i = 0$, which implies that $L_i$ are antisymmetric matrices. Further, using the formula $\det(\exp(M)) = \exp(\textup{tr}(M))$, the determinant condition $\det(R) = 1$ requires $\tr(L_i) = 0$. Therefore, we obtain that in the fundamental representation, the algebra is
\begin{equation}
    \mathfrak{so}(3) = \{ \textup{antisymmetric taceless $3 \times 3$ matrices}\},
\end{equation}
which is a three-dimensional real vector space with basis $\{L_i\}$, as anticipated.

From the explicit form of the generators, \color{red} missing $i$ factor to consider in a correct way, \color{black} we can compute the commutation relations explicitly, finding
\begin{equation}\label{eq:su2-algebra}
    \comm{J_i}{J_j} = i \sum_k \epsilon_{ijk} J_k .
\end{equation}

%%%%%%%%%%%%%%%%%%%%%%%%%%% SU(2) GROUP %%%%%%%%%%%%%%%%%%%%%%%%%%%
\subsection{The Group \texorpdfstring{$SU(2)$}{SU(2)}}
\color{red} Similarly, as before, we define the group in the fundamental representation and show the algebra is the same as before, given by eq.~\eqref{eq:su2-algebra}. \color{black}

%%%%%%%%%%%%%%%%%%%%%%%%%%% ALGEBRA REPRESENTATION %%%%%%%%%%%%%%%%%%%%%%%%%%%
\subsection{Representation of the Algebra \texorpdfstring{$\mathfrak{so}(3) \cong \mathfrak{su}(2)$}{isomorphic}}
Let's study the finite dimensional representations of the algebra $\mathfrak{so}(3) \cong \mathfrak{su}(2)$, defined by
\begin{equation}\label{eq:su2-commutation-relations}
    \comm{J_i}{J_j} = i \epsilon_{ijk} J_k .
\end{equation}

In particular, as discussed in section~\ref{sec:symmetries-qm}, the physically relevant ones are the unitary representations, acting on a vector space $V$ which is also a \emph{Hilbert space}, in the sense it carries a scalar product $\langle \cdot, \cdot \rangle$ defined by eq.~\eqref{eq:scalar-product}. In order to find the finite dimensional representation spaces $V$, we follow the usual quantum mechanical procedure.

We define the \emph{ladder operators}
\begin{equation}\label{eq:def-su2-ladder-op}
    J_\pm \coloneq J_1 \pm i J_2, \quad \textup{with} \quad \comm{J_3}{J_\pm} = \pm J_\pm, \quad \comm{J_+}{J_-} = 2 J_3 .
\end{equation}

We further define
\begin{equation}\label{eq:def-su2-total-ang}
    J^2 \coloneq {J_1}^2 + {J_2}^2 + {J_3}^2.
\end{equation}

From the fact that $J_i$ are Hermitian (\color{red}write it down well\color{black}), it's easy to check that
\begin{equation}
    (J_\pm)^\dagger = J_\mp, \quad (J^2)^\dagger = J^2 .
\end{equation}

Using the commutation relations~\eqref{eq:su2-commutation-relations} and the definitions~\eqref{eq:def-su2-ladder-op} and~\eqref{eq:def-su2-total-ang}, one can show
\begin{equation}
    \comm{J^2}{J_i} = \comm{J^2}{J_\pm} = 0,
\end{equation}
so, as already discussed in section~\eqref{sec:casimir}, $J^2$ it's a Casimir operator, that is, it commutes with all the Lie algebra generators. As known from quantum mechanics, it represents the total angular momentum operator.

Because $J_3$ and $J^2$ are Hermitian matrices on a Hilbert space $V$ which commute, the spectral theorem tells us that there is an orthonormal basis $\{ \ket{\zeta, m} \}$ of $V$, i.e., $\braket{\zeta, m}{\zeta',m'} = \delta_{mm'} \delta_{\zeta\zeta'}$, which are simultaneous eigenvectors of $J_3$ and $J^2$. The labels $m$ and $\zeta$ are chosen to be the eigenvalues of that vector: $J_3 \ket{\zeta, m} = m \ket{\zeta, m}$, and $J^2 \ket{\zeta, m} = \zeta \ket{\zeta, m}$. A priori, there could be degeneracies for these labels, i.e, more than one linearly independent basis elements with the same eigenvalues $(\zeta,m)$. However, as it turns out, the action of $J_\pm$ only depends on these values, so different basis elements with the same labels would correspond to invariant one-dimensional subspaces, i.e., we could restrict to one of these and obtain a valid representation. In the following, we will therefore assume without loss of generality that the eigenspace for each pair $(\zeta,m)$ is one-dimensional.

Since $J^2$ commutes with $J_\pm$, we have
\begin{equation*}
    J^2 J_\pm \ket{\zeta, m} = \zeta J_\pm \ket{\zeta, m},
\end{equation*}
and, further
\begin{equation*}
    J_3 J_\pm \ket{\zeta, m} = (\comm{J_3}{J_\pm} + J_\pm J_3) \ket{\zeta, m} = (\pm J_\pm + m J_\pm) \ket{\zeta, m} = (m \pm 1) J_\pm \ket{\zeta, m},
\end{equation*}
which means
\begin{equation}
    J_\pm \ket{\zeta, m} = \lambda^\pm_{\zeta,m} \ket{\zeta, m} .
\end{equation}

To find the possible values for $(\zeta, m, \lambda)$, we use
\begin{equation*}
    J^2 - {J_3}^2 = {J_1}^2 + {J_2}^2 = \frac{1}{2} (J_+ J_- + J_- J_+),
\end{equation*}
to show
\begin{equation*}
\begin{split}
    \bra{\psi} J^2 - {J_3}^2 \ket{\psi} &= \frac{1}{2} \bra{\psi} J_+ J_- + J_- J_+ \ket{\psi} = \frac{1}{2} \bra{\psi} J^\dagger_- J_- + J_+^\dagger J_+ \ket{\psi} \\ &= \frac{1}{2} \left( \mynorm{J_- \ket{\psi}}^2 + \mynorm{J_+ \ket{\psi}}^2  \right) \geq 0 ,
\end{split}
\end{equation*}
for any $\ket{\psi} \in V$. So, this implies
\begin{equation*}
    \bra{\zeta, m } J^2 - {J_3}^2 \ket{\zeta,m} = (\zeta - m^2) \braket{\zeta,m}{\zeta,m} \geq 0 .
\end{equation*}

Since $J_+$ increases the value of $m$ for fixed $\zeta$, there must be a maximal value, $m_\textup{max}(\zeta) \equiv m_\textup{max}$ such that $\ket{\zeta, m_\textup{max}} \neq 0$, but $J_+ \ket{\zeta, m_\textup{max}} = 0$. In particular, this means
\begin{equation*}
    0 = J_- J_+ \ket{\zeta, m_\textup{max}} = (J^2 - {J_3}^2 - J_3) \ket{\zeta, m_\textup{max}} = (\zeta - m^2_\textup{max} - m_\textup{max}) \ket{\zeta, m_\textup{max}} .
\end{equation*}

Likewise, because $J_-$ decreases the value of $m$, and thus also increasing $m^2$ once $m$ becomes negative, there is also a $\ket{\zeta, m_\textup{min}} \neq 0$ such that $J_- \ket{\zeta, m_\textup{min}} = 0$. Analogously, we get
\begin{equation*}
    0 = J_+ J_- \ket{\zeta, m_\textup{min}} = (J^2 - {J_3}^2 - J_3) \ket{\zeta, m_\textup{min}} = (\zeta - m^2_\textup{min} - m_\textup{min}) \ket{\zeta, m_\textup{min}} .
\end{equation*}

All together, this means
\begin{equation}
    \zeta = m_\textup{max} (m_\textup{max} + 1) = m_\textup{min} (m_\textup{min} - 1),
\end{equation}
which has solutions
\begin{equation*}
    m_\textup{max} = m_\textup{min} - 1 < m_\textup{min} \quad \textup{and} \quad m_\textup{max} = - m_\textup{min}.
\end{equation*}
The first is clearly not compatible with the assumption that $m_\textup{max} \geq m_\textup{min}$, so the second must hold. Moreover, since we can increase the value of $m$ in steps of $1$ by acting with $J_+$, $m_\textup{max} - m_\textup{min} = 2 m_\textup{max}$ must be an integer. The common notation in the literature is to define $j = m_\textup{max}$, which is in general a non-negative integer or half-integer that is otherwise not constrained.

Then, by convention, we replace the label $\zeta = j (j+1)$ by $j$, with $j \in \frac{\N_0}{2}$ and $m \in \{ -j, -j+1, \dots, j-1, j \}$. In summary, any representation space $V$ of $\mathfrak{so}(3) \cong \mathfrak{su}(2)$ has an orthonormal basis $\ket{j,m} \equiv \ket{\zeta, m}$ characterised by
\begin{equation}
    J^2 \ket{j,m} = j(j+1) \ket{j,m}, \quad J_3 \ket{j,m} = m \ket{j,m} .
\end{equation}

If $j \in \N_0$, then $m \in \Z$, and if $j \in \N$, then $m \in \Z + \frac{1}{2}$. Finally, there were the coefficients $\lambda^\pm_{\zeta, m} \equiv \lambda^\pm_{j,m}$, that remain to be determined. One can show
\begin{equation}
    J_\pm \ket{j,m} = \sqrt{(j \mp m)(j \pm m + 1)} \ket{j,m \pm 1} .
\end{equation}
\begin{proof}
    .
\end{proof}

Notice that the value of $j$ is not changed by the action of the Lie algebra generators. It means that different values of $j$ correspond to different representations, which have dimensions $2j +1$. Given the explicit action of the Lie algebra generators on the basis elements, one can explicitly verify that these representations are irreducible. The common nomenclature is to call an irreducible representation labelled by $j$ a "spin-$j$" irreducible representation of $\mathfrak{so}(3) \cong \mathfrak{su}(2)$.

For $j=1$, we obtain the $3d$ representation space $V_{j=1}$, which is the Lie algebra representation induced by the defining representation of $SO(3)$. The integer spin $j>1$ representations can be shown to correspond to (irreps inside the) tensor
products $V_1 \otimes \dots \otimes V_1 = {V_1}^{\otimes j}$, which therefore are naturally $SO(3)$ representations. But the half-integer spin representations are not $SO(3)$ representations, as we'll shortly see.


%%%%%%%%%%%%%%%%%%%%%%%%%%% HALF INTEGER SPIN IRREDUCIBLE REPRESENTATIONS %%%%%%%%%%%%%%%%%%%%%%%%%%%
\subsection{Half-integer spin irreducible representation.}
Let's fix a particular $j = \tilde{j} \in \frac{\N_0}{2}$ and name the basis element of the $2 \tilde{j} + 1$ dimensional representation space $V$, $\{ \ket{\tilde{j}, m} \}$, where $m \in \{ -\tilde{j}, \dots \tilde{j} \}$. Then, taking an element of the algebra, $X \in \mathfrak{so}(3) \cong \mathfrak{su}(2)$, the corresponding representation matrix on $V$ is given by
\begin{equation}
    M = \rho_j (X), \quad M_{ab} = \bra{\tilde{j},a} M \ket{\tilde{j},b}, \quad M \ket{\tilde{j}, m} = \sum_k \ket{\tilde{j}, k} M_{km} .
\end{equation}

Then, by means of eq.\dots, via the exponential map we get a representation of the covering group.

\color{red}
\dots
\dots
\dots

\begin{equation}
    \rho_j[R(\pi)R(\pi)]_{ab} = (-1)^{2\tilde{j}} \delta_{ab}.
\end{equation}
This shows that not all irrep of the algebra extends to irreps of $SO(3)$. They are, though, irreps of the universal covering group, which is $SU(2)$. We're interested in this last group in quantum mechanics, since \dots
\color{black}

%%%%%%%%%%%%%%%%%%%%%%%%%%%   relation and proof double cover%%%%%%%%%%%%%%%%%%%%%%%%%%%
\subsection{Relation between \texorpdfstring{$SO(3)$}{SO(3)} and \texorpdfstring{$SU(2)$}{SU(2)} and Proof of Double Coverness}
\dots
\dots
\dots