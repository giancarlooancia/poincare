Before beginning the discussion, we provide a list of useful references to help understand the physical relevance of the representations of the Poincaré group. 

Our main reference will be the QFT1 lecture notes from the University of Bologna course by Professor Michele Cicoli,~\cite{cicoli}, which introduce groups and representations, and outline the construction of one-particle state multiplets using the Casimir operators. For representation theory, we'll refer to Professor Ling Lin's group theory lecture notes~\cite{ling} from RQM course of University of Bologna.

Further, regarding books, an introductory option is Schwartz's text,~\cite{schwartz}. Another helpful reference is Maggiore's book,~\cite{maggiore}, which covers representations on fields and one-particle states. A key resource is Chapter two of Weinberg's book,~\cite{weinberg}. Although the notation may initially seem complex, the effort is well rewarded as the physical meaning becames clear by the end of the chapter. We'll occasionally refer to results from this text without reproducing all the details and computations.

Lastly, I recommend a valuable discussion on \emph{stackexchange},~\cite{stackexchange}. This provides an insightful overview of the role of fields as representations of the Poincaré group, both before and after the quantisation.

We now introduce some basic concepts of group theory which will be used throughout the discussion.