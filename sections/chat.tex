\color{blue}


\section{Group Theory Basics}

\subsection{Lie Groups and Algebra}

\begin{definition}[Group]
A group $(G, \circ)$ is a set $G$ equipped with a composition map, 
\[
\circ : G \times G \to G, \quad (g, h) \mapsto g \circ h \in G,
\]
called group multiplication, which satisfies the following axioms:
\begin{enumerate}
    \item Closure: $\forall g_1, g_2 \in G \Rightarrow g_1 \circ g_2 \in G$.
    \item Associativity: $\forall g_1, g_2, g_3 \in G \Rightarrow (g_1 \circ g_2) \circ g_3 = g_1 \circ (g_2 \circ g_3)$.
    \item Identity element: $\exists e \in G$ such that $\forall g \in G \Rightarrow g \circ e = e \circ g = g$.
    \item Inverse element: $\forall g \in G, \exists g^{-1} \in G$ such that $g \circ g^{-1} = g^{-1} \circ g = e$.
\end{enumerate}
If, in addition, $\forall g_1, g_2 \in G \Rightarrow g_1 \circ g_2 = g_2 \circ g_1$, then $G$ is said to be an \emph{abelian} group.
\end{definition}

\begin{definition}[Subgroup]
Given a group $(G, \circ)$, a \emph{subgroup} $(H, \circ)$ is a subset $H \subset G$ that itself satisfies the group axioms with the composition inherited from $G$.
\end{definition}

\noindent
\textbf{Remark.} For $H$ to be a subgroup, it must contain the same identity element $e \in G$. While an abelian group admits only abelian subgroups, a non-abelian group can have both abelian and non-abelian subgroups.

A group can be finite or infinite based on its number of elements. The \emph{dimension} of a group $G$, denoted by $\dim(G)$, represents the number of real parameters required to specify an element of $G$.

\begin{definition}[Group Homomorphism and Isomorphism]
A \emph{group homomorphism} is a function between two groups that preserves the group structure. For groups $(G_1, \circ)$ and $(G_2, \cdot)$ with $g_1, g_2 \in G_1$, a map $f : G_1 \to G_2$ is a homomorphism if
\[
f(g_1) \cdot f(g_2) = f(g_1 \circ g_2).
\]
A \emph{group isomorphism} is a bijective homomorphism.
\end{definition}

\begin{definition}[One-parameter Subgroup]
A \emph{one-parameter subgroup} is a continuous group homomorphism
\[
\varphi : \mathbb{R} \to G,
\]
where $\mathbb{R}$ is considered as an additive group. If $\varphi$ is injective, its image $\varphi(\mathbb{R})$ forms a subgroup of $G$, isomorphic to $\mathbb{R}$.
\end{definition}

\begin{definition}[Lie Group]
A \emph{Lie group} is a group whose elements depend continuously and differentiably on a set of real parameters $\theta^a$, where $a = 1, \dots, N$. Thus, a Lie group is both a group and a differentiable manifold.
\end{definition}

\noindent
\textbf{Remark.} While a Lie group could be defined as a Hausdorff topological group that behaves like a transformation group near the identity, we will focus on its manifold structure, which is more relevant to our purposes.

\noindent
Each element of the Lie group can be represented as a point on its manifold, and the dimension of the group matches that of the manifold. We denote a generic element as $g(\theta)$ and choose the coordinates $\theta^a$ such that the identity element $e$ of the group corresponds to $\theta^a = 0$, i.e., $g(0) = e$. This structure allows us to expand group elements in a Taylor-like series and consider elements infinitesimally close to the identity. The set of these “infinitesimal” elements forms the tangent space at the identity $e$, which is the basis of the Lie algebra. 

To clarify, we introduce the following definitions.

\begin{definition}[Tangent Space]
Let $M$ be a manifold. For each point $x \in M$, the \emph{tangent space} at $x$, $T_xM$, is the space of tangent vectors
\[
v = \left. \frac{d}{dt} \right|_{t=0} \gamma(t) \in T_xM,
\]
where $\gamma : \mathbb{R} \to M$ is any curve on the manifold passing through $x$. The dimension of $T_xM$ matches that of $M$.
\end{definition}

\begin{definition}[Lie Algebra]
For a Lie group $G$ with identity $e \in G$, the \emph{Lie algebra} $\mathfrak{g}$ is defined as the tangent space at the identity, i.e., $\mathfrak{g} = T_eG$.
\end{definition}

Moving forward, we will introduce additional definitions and theorems that will be essential in our exploration of the Poincaré group and its properties. Where possible, we will focus on the manifold structure of Lie groups and specify relevant topological aspects.

\begin{definition}[Connected Space]
A topological space is said to be \emph{connected} if it cannot be divided into two or more disjoint non-empty open subsets.
\end{definition}

\begin{definition}[Connected Component]
For a group $G$, the \emph{connected component} (also known as the identity component) is the largest connected subgroup of $G$ containing the identity element.
\end{definition}

\begin{definition}[Simply Connected]
A Lie group $G$ is \emph{simply connected} if any two paths $\gamma(t)$ and $\gamma'(t)$, which share the same endpoints, can be continuously deformed into one another.
\end{definition}

\noindent
The Lorentz group, for instance, has four connected components, of which we choose the proper orthochronous part. Since it is not simply connected, the following theorem becomes relevant.

\begin{theorem}
If a Lie group $G$ is not simply connected, there exists another Lie group $\tilde{G}$ with an isomorphic Lie algebra, $\tilde{\mathfrak{g}} \cong \mathfrak{g}$, that is simply connected. This $\tilde{G}$ is known as the \emph{universal cover} of $G$, and there exists a surjective group homomorphism $\pi : \tilde{G} \to G$.
\end{theorem}

Thus, the algebras of $\tilde{G}$ and $G$ coincide, i.e., $\tilde{\mathfrak{g}} \cong \mathfrak{g}$. For each point $g \in \tilde{G}$, there exists an open neighborhood $g \in U \subset \tilde{G}$ such that the restriction $\pi|_U : U \to \pi(U)$ is a diffeomorphism. Consequently, the tangent space at any $g \in \tilde{G}$ is isomorphic to the tangent space at $\pi(g) \in G$, including the tangent space at the identity.

For the proper orthochronous Lorentz group, this universal cover is the spin group, which is isomorphic to $SL(2; \mathbb{C})$.

\begin{definition}[Compact Space]
A topological space $X$ is \emph{compact} if every open cover of $X$ has a finite subcover. Roughly speaking, a Lie group is compact if its parameter space is bounded.
\end{definition}

\begin{theorem}
Every compact Lie group is isomorphic to a matrix group.
\end{theorem}

Due to the presence of boosts, the Lorentz group is non-compact, which implies that its unitary representations must be infinite-dimensional. This property directs us towards studying representations on one-particle Hilbert spaces.

We now introduce the \emph{exponential map}, a key tool for obtaining elements of $G$ from its Lie algebra $\mathfrak{g}$.

\begin{definition}[Exponential Map]
Let $G$ be a Lie group and $\mathfrak{g}$ its Lie algebra. The exponential map is defined as
\[
\exp: \mathfrak{g} \to G,
\]
where for $X \in \mathfrak{g}$, $\exp(X) = \gamma(1)$, with $\gamma : \mathbb{R} \to G$ as the unique one-parameter subgroup of $G$ such that the tangent vector at the identity equals $X$.
\end{definition}

The exponential map allows us to explore the properties of $G$ via its Lie algebra. Although generally the image $\operatorname{Im}(\exp) \subset G$ is only a neighborhood of the identity and not necessarily surjective, for connected and compact groups $G$ it is indeed surjective. However, due to the boosts, even the proper orthochronous Lorentz group is non-compact, so exponentiating the Lie algebra elements may not span all elements of $G$. For our purposes, we will assume that, in the cases of interest, each element of the connected component of $G$ can be obtained from its Lie algebra $\mathfrak{g}$.

The Lie algebra multiplication, called the \emph{Lie bracket} or \emph{commutator}, is given by
\[
[ \cdot, \cdot ] : \mathfrak{g} \times \mathfrak{g} \to \mathfrak{g}, \quad (X, Y) \mapsto [X, Y],
\]
which is antisymmetric,
\[
[X, Y] = -[Y, X],
\]
and satisfies the Jacobi identity,
\[
[X, [Y, Z]] + [Y, [Z, X]] + [Z, [X, Y]] = 0.
\]
For connected groups, the Baker-Campbell-Hausdorff (BCH) formula relates group composition to Lie algebra multiplication:
\[
\exp(X) \circ \exp(Y) = \exp \left( X + Y + \frac{1}{2}[X, Y] + \frac{1}{12}[X, [X, Y]] - \frac{1}{12}[Y, [X, Y]] + \dots \right).
\]

Given a basis $\{ T_a \}$ for the Lie algebra $\mathfrak{g}$, the Lie bracket structure can be encoded in terms of structure constants $f^c_{ab}$, where
\[
[T_a, T_b] = i f^c_{ab} T_c.
\]
These structure constants uniquely define the Lie algebra. Through the exponential map and the BCH formula, we express elements of the Lie group in terms of exponentials of linear combinations of the basis elements $\{T_a\}$, called \emph{generators} of the Lie algebra. For any $g \in G$, we have
\[
g = e^{i \theta^a T_a}.
\]


\subsection{Group Representations}

So far, we have treated groups and algebras as abstract entities. To apply these concepts and interpret them as symmetry transformations, we need to “represent” them concretely.

\begin{definition}[Group Representation]
A \emph{(linear) representation} of a group $G$ is a group homomorphism (see Definition 3) $\rho : G \to \operatorname{Aut}(V)$, mapping $G$ to the group of automorphisms\footnote{An automorphism is an isomorphism from an object to itself. In this context, the group of automorphisms is the set of invertible linear maps on a vector space.} on a vector space $V$. Formally, 
\[
\rho(e) = \operatorname{id}_V, \quad \rho(g \circ h) = \rho(g)\rho(h), \quad \rho(g^{-1}) = \rho(g)^{-1}, \quad \forall g, h \in G.
\]
For finite-dimensional vector spaces where $\dim(V) = n < \infty$, we have $\operatorname{Aut}(V) \cong GL(n; K)$, so elements of the group are represented by $n \times n$ matrices with multiplication defined by matrix group law. The \emph{dimension} of a representation $(\rho, V)$ is the same as that of $V$.
\end{definition}

\noindent
\textbf{Remark.} The dimension of a representation differs from the dimension of the group itself.

From now on, we focus on finite-dimensional representations. For a given representation $(\rho, V)$, the group acts on vectors in $V$ as linear transformations. For each $g \in G$ and $v \in V$, we have
\[
v \mapsto \rho(g)v.
\]

\begin{definition}[Reducible Representation]
A representation is called \emph{reducible} if there exists a non-zero subspace $U \subset V$ such that
\[
\rho(g)u \in U, \quad \forall u \in U, \; g \in G.
\]
If no such subspace exists, $\rho$ is called an \emph{irreducible representation}.
\end{definition}

Given a reducible representation $\rho$, we can always find a basis for $V$ such that
\[
\rho(g) = \begin{pmatrix} \rho'(g) & \beta(g) \\ 0 & \rho''(g) \end{pmatrix},
\]
where the invariant subspace is $U = \{(u, 0)^T \in V\}$. This structure gives rise to a representation $\rho'$ of smaller dimension.

\begin{definition}[Completely Reducible Representation]
A reducible representation is called \emph{completely reducible} if $\beta(g) = 0$. In this case, $\rho$ decomposes into the direct sum of two representations:
\[
\rho \cong \rho' \oplus \rho''.
\]
In other words, in a completely reducible representation, the basis vectors of $V$ can be chosen to split into subsets that remain independent under transformation.
\end{definition}

\begin{definition}[Equivalent Representations]
Two representations $\rho_1$ and $\rho_2$ of the same dimension $n$ are called \emph{equivalent} if there exists an invertible $n \times n$ matrix $S$ such that
\[
\rho_2(g) = S^{-1} \rho_1(g) S, \quad \forall g \in G.
\]
Thus, if there exists a change of basis $S$ on $V$ relating the representations, they are equivalent.
\end{definition}

\begin{definition}[Faithful Representation]
A representation $\rho$ is called \emph{faithful} if
\[
g_1 \neq g_2 \Rightarrow \rho(g_1) \neq \rho(g_2).
\]
For a non-faithful representation, there exists a subset $H \subset G$ such that $\rho(h) = 1$ for all $h \in H$, where $H$ is a subgroup of $G$.
\end{definition}

\begin{definition}[Unitary Representation]
A \emph{unitary representation} is a complex representation, $\rho : G \to GL(n; \mathbb{C})$, where each $\rho(g)$ is a unitary matrix, meaning
\[
\rho(g)^{-1} = \rho(g)^{\dagger}.
\]
\end{definition}

To conclude this section, we introduce specific types of representations that will be used in the following.

\begin{definition}[Matrix Representation]
Let $(\rho, V)$ be a representation. In a \emph{matrix representation}, $V$ is a finite-dimensional vector space (dim $V = n$), and each group element $g \in G$ is represented by an $n \times n$ matrix $\rho(g)_{ij}$, where $i, j = 1, \dots, n$. For any vector $v = (v_1, \dots, v_n) \in V$, the action of $g$ on $V$ is given by
\[
v_i \mapsto \sum_{j=1}^n \rho(g)_{ij} v_j.
\]
\end{definition}

\begin{definition}[Fundamental Representation]
The \emph{fundamental representation} of a group $G$ is the representation $D$ such that, for any $v \in V$,
\[
D(g)v = gv, \quad D(T^a) = T^a.
\]
\end{definition}

\begin{definition}[Conjugate Representation]
The \emph{conjugate representation} $\overline{D}$ is defined by
\[
\overline{D}(g)v = g^* v, \quad g^* = \left( e^{i \theta^a T^a} \right)^* = e^{-i \theta^a (T^a)^*}.
\]
In this representation, $\overline{D}(T^a) = -(T^a)^*$.
\end{definition}

\subsection{Algebra Representations}

We can similarly define a representation for a Lie algebra $\mathfrak{g}$. Note that a Lie algebra can be defined independently of any associated Lie group, simply as a vector space $\mathfrak{g}$ with an antisymmetric product, the Lie bracket $[ \cdot, \cdot ]$.

\begin{definition}[Algebra Representation]
Given a Lie algebra $\mathfrak{g}$, a \emph{representation} of $\mathfrak{g}$ is a vector space $V$ along with an algebra homomorphism (see Definition 13)
\[
\rho_\mathfrak{g} : \mathfrak{g} \to \operatorname{End}(V),
\]
where $\operatorname{End}(V)$ denotes the set of endomorphisms of $V$.
\end{definition}

The space of endomorphisms, $\operatorname{End}(V)$, has a natural vector space structure defined by the addition of linear maps and a non-necessarily invertible product defined by composition. With a chosen basis for $V$, $\operatorname{End}(V)$ becomes a space of matrices, with matrix multiplication as the product.

For compatibility with the Lie algebra structure, the representation $\rho_\mathfrak{g}$ must satisfy
\[
\rho_\mathfrak{g}([X, Y]) = \rho_\mathfrak{g}(X)\rho_\mathfrak{g}(Y) - \rho_\mathfrak{g}(Y)\rho_\mathfrak{g}(X),
\]
where the product on the right-hand side represents matrix multiplication. Given a set of generators $\{ T^a \}$ with structure constants $f^c_{ab}$, we have
\[
\rho_\mathfrak{g}([T^a, T^b]) = f^c_{ab} \rho_\mathfrak{g}(T^c).
\]

We can now demonstrate that any representation $(\rho, V)$ of a Lie group $G$ induces a representation of its Lie algebra $\mathfrak{g}$. For any $g = \exp(tX) \in G$, with $t \in \mathbb{R}$ and $X \in \mathfrak{g}$, the map $\rho(\exp(tX))$ defines a continuous “path” of transformations on the representation space $V$. This leads to a Lie algebra representation $\rho_\mathfrak{g}(X)$ on $V$, defined by
\[
\rho_\mathfrak{g}(X)v := \left. \frac{d}{dt} \rho(\exp(tX))v \right|_{t=0}, \quad \forall v \in V.
\]
Thus, $\rho_\mathfrak{g}(X)$ is a matrix of the same size as $\rho(g)$ and also acts on $V$. By applying the Baker-Campbell-Hausdorff (BCH) formula, one can show that $\rho_\mathfrak{g}$ respects the Lie bracket structure, making $(\rho_\mathfrak{g}, V)$ a valid representation of the Lie algebra.

Consider a parameterization $\rho(g(\theta)) := \rho(\theta)$ for the Lie group $G$, with generators in the representation $\rho$ defined as $T^a_\rho := \rho_\mathfrak{g}(T^a)$. In the neighborhood of the identity, we have
\[
\rho(\theta) \approx \mathbb{1} + i \theta^a T^a_\rho,
\]
with
\[
T^a_\rho \equiv -i \frac{\partial \rho}{\partial \theta^a} \bigg|_{\theta=0}.
\]
For the identity-connected component of the group, any group element $g(\theta)$ can be represented by
\[
\rho(g(\theta)) = \exp(i \theta^a T^a_\rho).
\]

\noindent
\textbf{Remark.} Although the explicit form of the generators $T^a_\rho$ depends on the specific representation, the structure constants $f^c_{ab}$, defined by $[T^a, T^b] = i f^c_{ab} T^c$, remain independent of the representation.

Conversely, not all Lie algebra representations $\rho_\mathfrak{g}$ necessarily extend to representations of the corresponding group $G$. This discrepancy arises from the global topology of $G$. In particular, all representations of the Lie algebra extend to group representations if $G$ is simply connected (see Definition 10). According to Theorem T.1, if $G$ is not simply connected, there exists a universal cover $\tilde{G}$ that is simply connected and has an isomorphic algebra, $\tilde{\mathfrak{g}} \cong \mathfrak{g}$. Consequently, we can construct representations of the algebras, which extend to representations of the covering group $\tilde{G}$.

As an example, the Lorentz group is not simply connected, and its universal cover is $SL(2; \mathbb{C})$. Therefore, not all representations of its Lie algebra extend to representations of the Lorentz group itself.

We now state two important theorems that will be essential in identifying physical observables in quantum mechanics.

\begin{theorem}
All unitary projective representations of a group $G$ originate from unitary linear representations of the universal covering group $\tilde{G}$, which in turn come from representations of the Lie algebra.
\end{theorem}

\begin{theorem}
Non-compact groups have no finite-dimensional unitary representations, except for cases where non-compact generators act trivially, i.e., as zero.
\end{theorem}

\noindent
The physical significance of this second theorem is due to the fact that in a unitary representation, the generators are Hermitian operators. According to the principles of quantum mechanics, only Hermitian operators correspond to observables. Therefore, for a non-compact group, we require an infinite-dimensional representation to identify its generators with physical observables. This requirement leads us to consider representations on the Hilbert space of one-particle states, as we will explore.

\subsection{Casimir Operators}

Casimir operators play a significant role in the study of representations. These operators are constructed from the generators $T^a$ of a Lie algebra and commute with all generators themselves. In each irreducible representation, Casimir operators are proportional to the identity matrix, with the proportionality constant serving to label the representation.

As an example, consider the algebra $\mathfrak{su}(2)$, defined by the commutation relations
\[
[J_i, J_j] = i \epsilon_{ijk} J_k.
\]
The Casimir operator for this algebra is
\[
J^2 = J_1^2 + J_2^2 + J_3^2.
\]
In an irreducible representation, $J^2$ takes the value $j(j + 1)$ times the identity matrix, where $j$ can take values $0, \frac{1}{2}, 1, \dots$.

Casimir operators are essential for characterizing representations because they allow us to classify the different possible representations of a given Lie algebra by their eigenvalues. For example, in the case of $\mathfrak{su}(2)$, the value of $j$ distinguishes different representations. 

For the Poincaré group, which we will discuss in detail later, the Casimir operators will similarly provide essential information on how one-particle states can be classified under the symmetry transformations of spacetime.

\section{Symmetries in Quantum Mechanics}

In this section, we review the concepts of states and observables in quantum mechanics, with a focus on the role of projective representations as opposed to standard representations. 

Let us begin by outlining some fundamental aspects of quantum mechanics.

\begin{itemize}
    \item \textbf{States.} Physical states are represented by rays in a Hilbert space $\mathcal{H}$, which is a complex vector space with a scalar product $\langle \cdot, \cdot \rangle$ defined by
    \begin{subequations}
        \begin{align}
            \langle \phi, \psi \rangle &= \langle \psi, \phi \rangle^*, \\
            \langle \phi, \xi_1 \psi_1 + \xi_2 \psi_2 \rangle &= \xi_1 \langle \phi, \psi_1 \rangle + \xi_2 \langle \phi, \psi_2 \rangle, \\
            \langle \eta_1 \phi_1 + \eta_2 \phi_2, \psi \rangle &= \eta_1^* \langle \phi_1, \psi \rangle + \eta_2^* \langle \phi_2, \psi \rangle,
        \end{align}
    \end{subequations}
    where $\phi, \psi \in \mathcal{H}$. The norm is defined by $\langle \psi, \psi \rangle = \langle \psi | \psi \rangle \geq 0$, and it vanishes only if $\psi \equiv 0$. A ray $R$ is a set of normalized vectors (satisfying $\langle \psi | \psi \rangle = 1$) where $\psi, \psi' \in R$ if $\psi' = \xi \psi$ for some $\xi \in \mathbb{C}$ with $|\xi| = 1$.
    
    \item \textbf{Observables.} Observables are represented by Hermitian operators $A$ on $\mathcal{H}$, defined by the properties
    \begin{subequations}
        \begin{align}
            A(\xi \psi + \eta \phi) &= \xi A \psi + \eta A \phi, \\
            A &= A^\dagger,
        \end{align}
    \end{subequations}
    where $A^\dagger$ denotes the adjoint of $A$, given by
    \[
    \langle \phi, A^\dagger \psi \rangle = \langle A \phi, \psi \rangle = \langle \psi, A \phi \rangle^*.
    \]
    If a state is represented by a ray $R$ and an observable by a Hermitian operator $A$, then the state has a definite value $\alpha$ for the observable if vectors $\psi \in R$ are eigenstates of $A$ with eigenvalue $\alpha$:
    \[
    A \psi = \alpha \psi, \quad \text{for } \psi \in R.
    \]
    If $A$ is Hermitian, $\alpha \in \mathbb{R}$ and eigenstates with different eigenvalues are orthogonal.
    
    \item \textbf{Probabilities.} If a system is in a state represented by a ray $R$, and an experiment tests if it is in one of the mutually orthogonal rays $R_1, R_2, \dots$, then the probability of finding the system in $R_n$ is
    \[
    P(R \to R_n) = |\langle \psi, \psi_n \rangle|^2,
    \]
    where $\psi \in R$ and $\psi_n \in R_n$. If the $\psi_n$ form a complete set, then $\sum_n P(R \to R_n) = 1$.
\end{itemize}

A transformation acting on rays is a \emph{symmetry} if it preserves probabilities. If an observer $\mathcal{O}$ sees a state represented by a ray $R$, or one of several states represented by rays $R_1, R_2, \dots$, then another observer $\mathcal{O}'$ sees the same system in states represented by rays $R', R_1', R_2', \dots$, with matching probabilities:
\[
P(R \to R_n) = P(R' \to R_n').
\]

\noindent
A fundamental result, \emph{Wigner's theorem}, states that for any such ray transformation $T : R \to R'$, there exists a unitary operator $U$ on the Hilbert space such that if $\psi \in R$, then $U \psi \in R'$, with $U$ satisfying
\begin{subequations}
    \begin{align}
        \langle U \psi, U \phi \rangle &= \langle \psi, \phi \rangle, \\
        U(\xi \psi + \eta \phi) &= \xi U \psi + \eta U \phi,
    \end{align}
\end{subequations}
for all $\psi, \phi \in \mathcal{H}$. Using the definition of an adjoint operator, this unitarity condition can also be expressed as
\[
U^\dagger = U^{-1}.
\]

The identity operator $U = 1$ trivially represents a symmetry, preserving the ray $R$. An infinitesimal symmetry transformation close to the identity is represented by a unitary operator near $U = 1$:
\[
U = 1 + i \epsilon T,
\]
where $\epsilon$ is a real infinitesimal parameter. For $U$ to be unitary, $T$ must be both linear and Hermitian, qualifying it as an observable. In physics, most observables arise in this way from symmetry transformations.

It is straightforward to verify that symmetry transformations from one ray to another satisfy the group axioms. Given transformations $T_1 : R_n \to R_n'$ and $T_2 : R_n' \to R_n''$, the transformation $T_2 T_1 : R_n \to R_n''$ is also a symmetry transformation. The inverse transformation $T^{-1} : R_n' \to R_n$ exists, and the identity transformation leaves the ray unchanged.

The operators $U(T)$ corresponding to these symmetry transformations form a group, but unlike the symmetry transformations themselves, the operators $U(T)$ act on vectors in Hilbert space rather than on rays. If $T_1 : R_n \to R_n'$, then $U(T_1)$ acts on $\psi_n \in R_n$ and gives $U(T_1) \psi_n \in R_n'$. Similarly, if $T_2 : R_n' \to R_n''$, then $U(T_2)U(T_1) \psi_n \in R_n''$. Since $U(T_2 T_1)$ is also in the same ray, the vectors $U(T_2) U(T_1) \psi_n$ and $U(T_2 T_1) \psi_n$ differ only by a phase factor $\phi(T_2, T_1)$:
\[
U(T_2) U(T_1) \psi_n = e^{i \phi(T_2, T_1)} U(T_2 T_1) \psi_n.
\]

This type of representation is called a \emph{projective representation}. Unlike a linear representation (Definition 15), it includes a phase factor. Referring back to Section 2.2, a symmetry group $G$ acts on the Hilbert space $\mathcal{H}$ via a unitary operator $\rho(g) : \mathcal{H} \to \mathcal{H}$ for $g \in G$, which defines a projective representation:
\[
\rho(g) \rho(h) = e^{i \phi(g, h)} \rho(g \circ h), \quad \phi(g, h) \in \mathbb{R}.
\]

Therefore, in quantum theory, we are interested in unitary projective representations of a symmetry group $G$. According to Theorem T.3, all unitary projective representations of $G$ can be derived from unitary linear representations of the universal covering group $\tilde{G}$, which, in turn, arise from representations of the Lie algebra. Furthermore, by Theorem T.4, since the Poincaré group is non-compact, we must study infinite-dimensional representations to obtain unitary ones, which can then be associated with physical observables.

\section{The Groups SO(3) and SU(2) and their Representations}

In this section, we review the essential properties of the Lie groups $SO(3)$ and $SU(2)$ and their respective algebras, as these will be useful in analyzing the representations of the Lorentz group.

\subsection{The Group SO(3)}

In its fundamental representation, the special orthogonal group $SO(n)$ consists of $n \times n$ orthogonal matrices satisfying
\[
R \in SO(n) : R^T R = R R^T = 1, \quad \det(R) = 1.
\]
It is straightforward to show that the dimension of $SO(n)$ is $\dim(SO(n)) = \frac{n(n-1)}{2}$. We are specifically interested in the case $n = 3$, giving the three-dimensional rotation group $SO(3)$.

To determine the Lie algebra $\mathfrak{so}(3)$, observe that any $R \in SO(3)$ can be parameterized by three Euler angles, or equivalently, by a rotation axis and an angle about this axis. For a unit vector $\mathbf{n} \in \mathbb{R}^3$ (normalized such that $\mathbf{n} \cdot \mathbf{n} = 1$) and an angle $\theta$ about it, the identity element is given by $R(\theta = 0, \mathbf{n}) = 1$. An infinitesimal rotation by a small angle $\delta \theta$ acts on an arbitrary vector $\mathbf{x} \in \mathbb{R}^3$ as
\[
R(\delta \theta, \mathbf{n}) \mathbf{x} = \mathbf{x} + \delta \theta \, \mathbf{n} \times \mathbf{x} = \mathbf{x} + \delta \theta \begin{pmatrix} n_2 x_3 - n_3 x_2 \\ n_3 x_1 - n_1 x_3 \\ n_1 x_2 - n_2 x_1 \end{pmatrix}.
\]
Choosing $\mathbf{n}$ along each coordinate axis, we obtain
\begin{subequations}
    \begin{align}
        R(\delta \theta, \mathbf{n} = (1, 0, 0)^T) \mathbf{x} &= \mathbf{x} + \delta \theta \begin{pmatrix} 0 \\ -x_3 \\ x_2 \end{pmatrix} = \mathbf{x} + \delta \theta L_1 \mathbf{x}, \\
        R(\delta \theta, \mathbf{n} = (0, 1, 0)^T) \mathbf{x} &= \mathbf{x} + \delta \theta \begin{pmatrix} x_3 \\ 0 \\ -x_1 \end{pmatrix} = \mathbf{x} + \delta \theta L_2 \mathbf{x}, \\
        R(\delta \theta, \mathbf{n} = (0, 0, 1)^T) \mathbf{x} &= \mathbf{x} + \delta \theta \begin{pmatrix} -x_2 \\ x_1 \\ 0 \end{pmatrix} = \mathbf{x} + \delta \theta L_3 \mathbf{x}.
    \end{align}
\end{subequations}
The matrices $L_i$ defined above correspond to a basis for the Lie algebra $\mathfrak{so}(3)$. Since $SO(3)$ is the connected component of $O(3)$ and is compact, each element of the group can be written as an exponential of a linear combination of these generators, as shown in Equation (2.32).

Using the orthogonality condition $R^T R = 1$ and Equation (2.32), together with the Baker-Campbell-Hausdorff (BCH) formula, we find that
\[
1 = R(\theta, \mathbf{n})^T R(\theta, \mathbf{n}) = \exp \left( i \theta \sum_i n_i (L_i^T + L_i) + \dots \right).
\]
For an infinitesimal transformation, this condition requires $L_i^T + L_i = 0$, implying that $L_i$ are antisymmetric matrices. Furthermore, the condition $\det(R) = 1$ implies that $L_i$ are traceless. Thus, in the fundamental representation, the algebra $\mathfrak{so}(3)$ is given by
\[
\mathfrak{so}(3) = \{ \text{antisymmetric, traceless } 3 \times 3 \text{ matrices} \}.
\]
With this basis $\{L_i\}$, we can compute the commutation relations, finding
\[
[J_i, J_j] = i \sum_k \epsilon_{ijk} J_k.
\]

\subsection{The Group SU(2)}

The group $SU(2)$ can be similarly defined in its fundamental representation, where we show that its Lie algebra has the same commutation relations as in Equation (4.5).

\subsection{Representation of the Algebra $\mathfrak{so}(3) \cong \mathfrak{su}(2)$}

We now study the finite-dimensional representations of the algebra $\mathfrak{so}(3) \cong \mathfrak{su}(2)$, defined by the commutation relations
\[
[J_i, J_j] = i \epsilon_{ijk} J_k.
\]
In particular, the physically relevant representations are unitary representations, which act on a vector space $V$ that is also a Hilbert space, equipped with a scalar product $\langle \cdot, \cdot \rangle$ as in Equation (3.1). To find the finite-dimensional representation spaces $V$, we follow the standard quantum mechanical approach.

Define the ladder operators
\[
J_\pm := J_1 \pm i J_2,
\]
which satisfy
\[
[J_3, J_\pm] = \pm J_\pm, \quad [J_+, J_-] = 2 J_3.
\]
Also define the Casimir operator
\[
J^2 := J_1^2 + J_2^2 + J_3^2.
\]
Since the operators $J_i$ are Hermitian, we have
\[
(J_\pm)^\dagger = J_\mp, \quad (J^2)^\dagger = J^2.
\]
One can show that the Casimir operator commutes with all generators:
\[
[J^2, J_i] = [J^2, J_\pm] = 0,
\]
indicating that $J^2$ serves as a Casimir operator. This operator represents the total angular momentum.

The spectral theorem guarantees the existence of an orthonormal basis $\{ |j, m\rangle \}$ of $V$, where $\langle j, m | j', m' \rangle = \delta_{mm'} \delta_{jj'}$, such that $|j, m \rangle$ are simultaneous eigenvectors of $J_3$ and $J^2$:
\[
J_3 |j, m \rangle = m |j, m \rangle, \quad J^2 |j, m \rangle = j(j+1) |j, m \rangle.
\]
Assuming without loss of generality that each eigenspace is one-dimensional, we find that each representation space $V$ has an orthonormal basis $|j, m \rangle$, with $j \in \frac{\mathbb{N}_0}{2}$ and $m \in \{-j, -j+1, \dots, j-1, j\}$.

The action of the ladder operators on these basis states is given by
\[
J_\pm |j, m \rangle = \sqrt{(j \mp m)(j \pm m + 1)} |j, m \pm 1 \rangle.
\]
This shows that different values of $j$ correspond to different representations, each with dimension $2j + 1$. These representations are irreducible and are referred to as \emph{spin-$j$} irreducible representations of $\mathfrak{so}(3) \cong \mathfrak{su}(2)$.

For $j = 1$, we obtain the three-dimensional representation corresponding to the Lie algebra representation induced by the defining representation of $SO(3)$. Representations with integer spin $j > 1$ can be obtained from tensor products of the $j = 1$ representation, while half-integer spin representations correspond to representations of the covering group $SU(2)$, as we discuss next.

\subsection{Half-integer Spin Irreducible Representation}

Let us consider a specific value $j = \tilde{j} \in \frac{\mathbb{N}_0}{2}$ and denote the basis elements of the $(2\tilde{j} + 1)$-dimensional representation space $V$ as $\{ |\tilde{j}, m \rangle \}$ with $m \in \{-\tilde{j}, \dots, \tilde{j}\}$. For an element $X \in \mathfrak{so}(3) \cong \mathfrak{su}(2)$, the corresponding representation matrix on $V$ is given by
\[
M = \rho_{\tilde{j}}(X), \quad M_{ab} = \langle \tilde{j}, a | M | \tilde{j}, b \rangle, \quad M | \tilde{j}, m \rangle = \sum_k | \tilde{j}, k \rangle M_{km}.
\]
Through the exponential map, we obtain a representation of the covering group.

For a $2\pi$ rotation, we find
\[
\rho_{\tilde{j}}[R(\pi)R(\pi)]_{ab} = (-1)^{2\tilde{j}} \delta_{ab}.
\]
This shows that not all irreducible representations of the algebra extend to irreducible representations of $SO(3)$, though they do extend to representations of the universal covering group, $SU(2)$. In quantum mechanics, we are primarily interested in the group $SU(2)$.

\subsection{Relation between SO(3) and SU(2) and Proof of Double Cover}

To conclude, $SU(2)$ is the double cover of $SO(3)$, which means there is a two-to-one mapping from $SU(2)$ to $SO(3)$. This double cover structure allows half-integer spin representations in $SU(2)$ that do not have counterparts in $SO(3)$.

\section{Poincaré Group and Algebra and Their Representations}

To analyze the properties of the Lorentz and Poincaré groups, we begin with their defining representations, keeping in mind which properties are representation-independent. We will work in a four-dimensional Minkowski spacetime with metric tensor $\eta = (\eta_{\mu \nu}) = \operatorname{diag}(+1, -1, -1, -1)$ and scalar product defined by
\[
x \cdot y := x^T \eta y = x^0 y^0 - \mathbf{x} \cdot \mathbf{y} = \eta_{\mu \nu} x^\mu y^\nu = x_\mu y^\mu.
\]

\subsection{Lorentz Group}

Lorentz transformations are those transformations $x \to x' = \Lambda x$ that leave the scalar product invariant, i.e.,
\[
(\Lambda x) \cdot (\Lambda y) = x \cdot y \Rightarrow x^T \Lambda^T \eta \Lambda y = x^T \eta y \Rightarrow \Lambda^T \eta \Lambda = \eta.
\]
In component form, this condition becomes
\[
\eta_{\mu \nu} = \eta_{\alpha \beta} \Lambda^\alpha_{\ \mu} \Lambda^\beta_{\ \nu}.
\]
Since $\eta_{\mu \nu}$ is symmetric, this condition yields 10 constraints. Given that a Lorentz transformation is represented by a $4 \times 4$ matrix, it depends on $16 - 10 = 6$ independent parameters, which can be interpreted as three parameters for boosts and three for rotations.

For an infinitesimal transformation
\[
\Lambda^\mu_{\ \nu} \approx \delta^\mu_\nu + \omega^\mu_{\ \nu},
\]
and using Eq. (5.3), we find that
\[
\omega_{\mu \nu} = -\omega_{\nu \mu}.
\]

\begin{proof}
Starting from the invariance condition $\eta_{\mu \nu} = \eta_{\alpha \beta} \Lambda^\alpha_{\ \mu} \Lambda^\beta_{\ \nu}$ and expanding $\Lambda^\mu_{\ \nu} \approx \delta^\mu_\nu + \omega^\mu_{\ \nu}$, we obtain
\[
\eta_{\mu \nu} \approx \eta_{\mu \nu} + \omega_{\mu \nu} + \omega_{\nu \mu}.
\]
Since $\eta_{\mu \nu} = \eta_{\mu \nu}$, we must have $\omega_{\mu \nu} = -\omega_{\nu \mu}$, as desired.
\end{proof}

The transformations of a space with coordinates $\{ y_1, \dots, y_n, x_1, \dots, x_m \}$ that leave the quadratic form $(y_1^2 + \cdots + y_n^2) - (x_1^2 + \cdots + x_m^2)$ invariant define the orthogonal group $O(n, m)$. Thus, the Lorentz group is $O(1, 3)$.

The group axioms are satisfied, and, in particular, each $\Lambda$ has an inverse. Using the fact that the determinant of a product is the product of the determinants, and that the transpose of a matrix has the same determinant, we find
\begin{align}
\Lambda^T \eta \Lambda &= \eta \Rightarrow (\det \Lambda)^2 = 1 \Rightarrow \det \Lambda = \pm 1, \\
\eta_{\mu \nu} \Lambda^\mu_{\ 0} \Lambda^\nu_{\ 0} &= (\Lambda^0_{\ 0})^2 - \sum_{k=1}^3 (\Lambda^k_{\ 0})^2 = 1 \Rightarrow (\Lambda^0_{\ 0})^2 \geq 1.
\end{align}
The Lorentz group thus has four disconnected components, with the subgroup satisfying $\det \Lambda = 1$ and $\Lambda^0_{\ 0} \geq 1$ being the \emph{proper orthochronous Lorentz group}, denoted $SO(1, 3)^+$. The remaining components can be obtained by combining an element of $SO(1, 3)^+$ with space or time reflection.

\subsection{Lorentz Algebra}

We have seen that the Lorentz group has six independent parameters, which we can collect into the antisymmetric matrix $\omega_{\mu \nu}$ as in Eq. (5.5). It is convenient to label the generators as $M_{\mu \nu} = -M_{\nu \mu}$, where each pair $(\mu, \nu)$ identifies a particular generator. Using the exponential map, any element $\Lambda \in SO(1, 3)^+$ can be written as
\[
\Lambda = \exp \left( -\frac{i}{2} \omega_{\mu \nu} M^{\mu \nu} \right),
\]
with a choice of constants. Given a finite-dimensional representation $(\rho, V)$ of dimension $n$, this group element is represented by the $n \times n$ matrix
\[
\Lambda_\rho = \exp \left( -\frac{i}{2} \omega_{\mu \nu} M^{\mu \nu}_\rho \right),
\]
which acts on $V$, where $M^{\mu \nu}_\rho$ are the generators in the representation $\rho$. The elements of $V$ transform under a Lorentz transformation as
\[
\varphi^i \to \left[ e^{-\frac{i}{2} \omega_{\mu \nu} M^{\mu \nu}_\rho} \right]^i_j \varphi^j.
\]
For an infinitesimal Lorentz transformation with small parameters $\omega_{\mu \nu}$, the variation of $\varphi^i$ is
\[
\delta \varphi^i = -\frac{i}{2} \omega_{\mu \nu} \left( M^{\mu \nu}_\rho \right)^i_j \varphi^j,
\]
where $M^{\mu \nu}_\rho$ are matrices labeled by $\mu$ and $\nu$, with indices $i$ and $j$ indicating the matrix representation.

The commutation relations of the Lorentz algebra generators, which are representation-independent, are given by
\[
[M^{\mu \nu}, M^{\rho \sigma}] = i \left( \eta^{\nu \rho} M^{\mu \sigma} - \eta^{\mu \rho} M^{\nu \sigma} - \eta^{\sigma \mu} M^{\rho \nu} + \eta^{\sigma \nu} M^{\rho \mu} \right).
\]
It is useful to decompose these generators into two spatial vectors:
\[
J^i = \frac{1}{2} \epsilon^{ijk} M^{jk}, \quad K^i = M^{i0},
\]
so that the Lie algebra of the Lorentz group can be written as
\begin{subequations}
    \begin{align}
        [J^i, J^j] &= i \epsilon^{ijk} J^k, \\
        [J^i, K^j] &= i \epsilon^{ijk} K^k, \\
        [K^i, K^j] &= -i \epsilon^{ijk} J^k.
    \end{align}
\end{subequations}
Equation (5.13a) is the Lie algebra of $SU(2)$, where $J$ can be interpreted as the angular momentum operator. The remaining commutation relations indicate that $K$ behaves as a spatial vector under rotation.

Using the rotation vector $\theta^i = \frac{1}{2} \epsilon^{ijk} \omega_{jk}$ and boost vector $\eta^i = \omega_{i0}$, any Lorentz transformation can be expressed as
\[
\Lambda = e^{-i \theta \cdot J + i \eta \cdot K}.
\]

\subsubsection{Trivial Representation}

In the trivial representation, a scalar field $\varphi$ is invariant under Lorentz transformations, so
\[
\delta \varphi = 0,
\]
and the generators, represented by $1 \times 1$ matrices, vanish identically:
\[
M^{\mu \nu} \equiv 0.
\]

\subsubsection{Vector Representation}

The defining (or vector) representation acts on four-vectors $V^\mu$, which transform according to
\[
V^\mu \to \Lambda^\mu_{\ \nu} V^\nu,
\]
where $\Lambda$ satisfies the Lorentz condition $\Lambda^T \eta \Lambda = \eta$. For a four-vector $V^\mu$, we find that the representation matrices are
\[
(M^{\mu \nu})^\alpha_{\ \beta} = i \left( \eta^{\mu \alpha} \delta^\nu_\beta - \eta^{\nu \alpha} \delta^\mu_\beta \right).
\]
This representation satisfies the commutation relations of the Lorentz algebra as expected, and provides a concrete example of the abstract commutation relations in Eq. (5.11).

\subsubsection{Spinor Representation}

To define the spinor representation, we use the fact that the group $SL(2; \mathbb{C})$ is the universal cover of the proper orthochronous Lorentz group $SO(1,3)^+$. Elements $\Lambda \in SO(1,3)^+$ are associated with elements $S \in SL(2; \mathbb{C})$, such that $\Lambda$ acts on four-vectors via
\[
x'^\mu = \Lambda^\mu_{\ \nu} x^\nu,
\]
while $S$ acts on spinors via
\[
\psi \to S \psi,
\]
where $\psi$ is a two-component object. The specific form of this transformation depends on the Pauli matrices and the relation between spinor and vector representations.

Spinor representations are crucial in quantum field theory, as they enable the description of particles with half-integer spin.

\section{One-particle State Representations}

In quantum field theory, one-particle states are described as irreducible representations of the Poincaré group. These representations provide insight into the properties of particles in terms of mass, spin, and energy-momentum.

\subsection{The Hilbert Space of One-particle States}

Consider the Hilbert space $\mathcal{H}$ of one-particle states. A one-particle state is specified by its four-momentum $p^\mu = (p^0, \mathbf{p})$ and additional quantum numbers $\sigma$ (e.g., spin) that distinguish states with the same four-momentum. We denote these states by $|p, \sigma \rangle$, with the inner product
\[
\langle p, \sigma | p', \sigma' \rangle = 2 p^0 (2\pi)^3 \delta^3(\mathbf{p} - \mathbf{p}') \delta_{\sigma \sigma'}.
\]

The action of the Poincaré group on $\mathcal{H}$ is specified by a unitary representation $U(\Lambda, a)$, where $\Lambda$ is a Lorentz transformation and $a$ is a spacetime translation. For a state $|p, \sigma \rangle$, we have
\[
U(\Lambda, a) |p, \sigma \rangle = e^{-i \Lambda p \cdot a} \sum_{\sigma'} C_{\sigma \sigma'}(\Lambda, p) | \Lambda p, \sigma' \rangle,
\]
where $C_{\sigma \sigma'}(\Lambda, p)$ is a unitary matrix that acts on the spin indices.

\subsection{Classification by Casimir Operators}

To classify the irreducible representations of the Poincaré group, we use its Casimir operators. These operators are constructed from the generators of translations $P_\mu$ and Lorentz transformations $M_{\mu \nu}$, which satisfy the Poincaré algebra. The two Casimir operators are:
\begin{itemize}
    \item The squared four-momentum operator
    \[
    P^2 = P_\mu P^\mu.
    \]
    This operator represents the squared mass of the particle, as it commutes with all other generators in the algebra. In an irreducible representation, its eigenvalue is $m^2$, where $m$ is the particle's mass.
    
    \item The Pauli-Lubanski vector
    \[
    W^\mu = -\frac{1}{2} \epsilon^{\mu \nu \rho \sigma} P_\nu M_{\rho \sigma},
    \]
    which satisfies
    \[
    [P_\mu, W^\nu] = 0.
    \]
    The squared Pauli-Lubanski operator, $W^2 = W_\mu W^\mu$, commutes with all generators and, in an irreducible representation, takes the eigenvalue $-m^2 s(s+1)$, where $s$ is the particle’s spin.
\end{itemize}

Thus, irreducible representations are characterized by the eigenvalues of $P^2$ and $W^2$, which correspond to the mass and spin of the particle.

\subsection{Massive and Massless Representations}

\begin{itemize}
    \item \textbf{Massive case} ($m > 0$): For particles with non-zero mass, we can move to the rest frame where $p^\mu = (m, 0, 0, 0)$. In this frame, the Pauli-Lubanski vector has components
    \[
    W^0 = 0, \quad W^i = m J^i,
    \]
    where $J^i$ are the generators of rotations, which correspond to the spin of the particle. The spin $s$ is quantized as $s \in \{0, \frac{1}{2}, 1, \dots \}$, where half-integer values arise when considering the covering group $SL(2; \mathbb{C})$ instead of $SO(1,3)^+$.
    
    \item \textbf{Massless case} ($m = 0$): For massless particles, there is no rest frame, as they always travel at the speed of light. The relevant subgroup that leaves the four-momentum $p^\mu = (E, 0, 0, E)$ invariant is the little group $ISO(2)$, which includes rotations around and translations along the direction of $p^\mu$. The Pauli-Lubanski vector $W^\mu$ becomes proportional to $p^\mu$ itself:
    \[
    W^\mu = \lambda p^\mu,
    \]
    where $\lambda$ is interpreted as the \emph{helicity} of the particle, taking values in $\{\pm s\}$.
\end{itemize}

In summary:
\begin{itemize}
    \item \textbf{Massive representations} are characterized by two parameters: the mass $m$ and the spin $s$.
    \item \textbf{Massless representations} are characterized by the helicity $\lambda$.
\end{itemize}

The study of these representations in the context of one-particle states provides insight into the properties of elementary particles, including their classification under the Poincaré symmetry group.


\color{black}