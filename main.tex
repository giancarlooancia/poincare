\documentclass[a4paper,11pt]{report}
\pdfoutput=1 % if your are submitting a pdflatex (i.e. if you have
             % images in pdf, png or jpg format)

\usepackage{jheppub} % for details on the use of the package, please
                     % see the JHEP-author-manual

\usepackage[T1]{fontenc} % if needed

%**************** Utilities **************
\usepackage{lipsum}
\usepackage{comment}
\usepackage{enumitem}
\usepackage{color}

% *************** Math and Physics *************
\usepackage{mathtools} % per valore assoluto e norma
\usepackage{braket} % per i comandi \Set e \Bra e simili
\usepackage{amsthm} % per teoremi e dimostrazioni
\usepackage{tensor} % per tensori e indici alto/basso
\usepackage{physics}
\usepackage{braket}
\usepackage{dsfont}
% ******************************************
% ******************************************************************************
% *************************** INSIEMISTICA *************************************
\newcommand{\numberset}{\mathbb}
\newcommand{\N}{\numberset{N}}
\newcommand{\Z}{\numberset{Z}}
\newcommand{\R}{\numberset{R}}
\newcommand{\C}{\numberset{C}}
\newcommand{\1}{\mathds{1}}
% ******************************************************************************
% ****************************** OPERATORS **************************************
%\DeclarePairedDelimiter{\abs}{\lvert}{\rvert}
\DeclarePairedDelimiter{\mynorm}{\lVert}{\rVert}
\DeclarePairedDelimiter{\inner}{\langle}{\rangle}
\DeclareMathOperator{\sgn}{sgn}
\DeclareMathOperator{\Realpart}{Re} % ridefinisco parte reale
\DeclareMathOperator{\Impart}{Im}
\renewcommand{\Re}{\Realpart} % ridefinisco parte reale (altrimenti dà simbolo in gotico)
\renewcommand{\Im}{\Impart} 
\DeclareMathOperator*{\argmax}{arg\,max}
%\DeclareMathOperator{\Tr}{Tr}
%\DeclareMathOperator{\Res}{Res}


% ******************************************************************************
% *************************** VECTOR CALCULUS **********************************
\newcommand{\bcdot}{\boldsymbol{\cdot}} % così \bcdot è prodotto scalare in grassetto
\renewcommand{\vec}{\boldsymbol}
\newcommand{\del}{\vec{\nabla}}


% ******************************************************************************
% *************************** DIFFERENTIATION **********************************
\newcommand{\ud}{\mathop{}\!\mathrm{d}}
\newcommand{\udd}{{\ud}^2}
\newcommand{\udt}{{\ud}^3}
\newcommand{\udq}{{\ud}^4}
\newcommand{\bb}[1]{\mathbb{#1}}
\newcommand{\de}{\partial}


% ******************************************************************************
% ******************************* THEOREMS *************************************
\theoremstyle{plain}
\newtheorem{theorem}{Theorem}

\theoremstyle{plain}
\newtheorem*{principle}{Principle}

\theoremstyle{plain}
\newtheorem{lemma}{Lemma}

\theoremstyle{definition}
\newtheorem{definition}{Definition}

\theoremstyle{remark}
\newtheorem*{remark}{Remark}

\renewcommand{\thelemma}{L.\arabic{lemma}}
\renewcommand{\thetheorem}{T.\arabic{theorem}}

\newenvironment{innerproof}
 {\renewcommand{\qedsymbol}{}\proof}
 {\endproof}

% ******************************************************************************
% ******************************* SHORTCUTS ************************************
\newcommand{\invgamma}{\sqrt{1- \frac{v^2}{c^2}}}
\renewcommand{\L}{\mathcal{L}}
\newcommand{\g}{\mathfrak{g}}
\newcommand{\h}{\mathfrak{h}}
\newcommand{\M}{\mathcal{M}}
\newcommand{\D}{\mathcal{D}}
\newcommand{\qi}{q^{(m)}}
\newcommand{\qf}{q^{(g)}}
\newcommand{\bdelta}{\bar{\delta}}
\newcommand{\deltat}{{\delta}^3}
\newcommand{\deltaq}{{\delta}^4}
\renewcommand{\epsilon}{\varepsilon}
\newcommand{\phis}{{\phi}^*}
\newcommand{\hbarq}{{\hbar}^2}
\renewcommand{\H}{\mathcal{H}}
\newcommand{\q}{\hat{\vec{q}}}
\newcommand{\Tau}{\mathcal{T}}
\newcommand{\ray}{\mathcal{R}}

% ******************************************************************************
% ************************ SHORTCUTS WITH ARGUMENTS ****************************
\newcommand{\bravec}[1]{\bra{\vec{#1}}} 
\newcommand{\ketvec}[1]{\ket{\vec{#1}}} 
\renewcommand{\op}[1]{\hat{#1}}
\newcommand{\opvec}[1]{\op{\vec{#1}}}
\newcommand{\dual}[1]{\widetilde{#1}}
\DeclareMathOperator{\Aut}{Aut}
\DeclareMathOperator{\id}{id}

% ******************************************************************************
% ******************************** GRAPHICS ************************************
\renewcommand\qedsymbol{$\blacksquare$}

% ******************************************************************************
% ******************************** matrices and scalar prodict ************************
\newcommand{\irow}[1]{% inline row vector
  \begin{smallmatrix}(\,#1\,)\end{smallmatrix}%
}

\newcommand{\icol}[1]{% inline column vector
  \left(\begin{smallmatrix}#1\end{smallmatrix}\right)%
}

\newcommand{\scalar}[2]{\langle #1, #2 \rangle}
\renewcommand{\norm}[1]{\scalar{#1}{#1}}

%%%%%%%%%%%%%%%%%%% JHEP SETTINGS %%%%%%%%%%%%%%%%%%%%%%%

\bibliographystyle{jhep}

\title{Representations of Poincaré Group}

\author[a]{Giancarlo Oancia}
\author[a]{Gian Luigi Paganini}
\affiliation[a]{University of Bologna}

\emailAdd{giancarlo.oancia@studio.unibo.it} 
\emailAdd{gianluigi.paganini@studio.unibo.it}

\dedicated{Dedicated to students: \\ 
to their sweat and toil, \\
to the incomprehensibility of physics, \\ 
and to the profound beauty of symmetries.}

\abstract{This document discusses the finite and infinite dimensional representations of the Poincaré Group. In particular, it gives special attention to the relationship between one-particle states and fields, both before and after quantisation.}

\keywords{Lorentz group, Poincaré group}

%%%%%%%%%%%%%%%% CHAPTER GRAPHICS %%%%%%%%%%%%
\usepackage{titlesec}
\titleformat{\chapter}{}{}{0em}{\bf\LARGE}



%%%%%%%%%%%%%%%% DOCUMENT %%%%%%%%%%%%%%%%%%%%%%%%
\begin{document}
\maketitle
\flushbottom
\addtocontents{toc}{\protect\thispagestyle{empty}}

\chapter{Introduction}
\pagenumbering{arabic}
\section{Introduction}
Before beginning the discussion, we provide a list of useful references to help understand the physical relevance of the representations of the Poincaré group. 

Our main reference will be the QFT1 lecture notes from the University of Bologna course by Professor Michele Cicoli,~\cite{cicoli}, which introduce groups and representations, and outline the construction of one-particle state multiplets using the Casimir operators. For representation theory, we'll refer to Professor Ling Lin's group theory lecture notes~\cite{ling} from RQM course of University of Bologna.

Further, regarding books, an introductory option is Schwartz's text,~\cite{schwartz}. Another helpful reference is Maggiore's book,~\cite{maggiore}, which covers representations on fields and one-particle states. A key resource is Chapter two of Weinberg's book,~\cite{weinberg}. Although the notation may initially seem complex, the effort is well rewarded as the physical meaning becames clear by the end of the chapter. We'll occasionally refer to results from this text without reproducing all the details and computations.

Lastly, I recommend a valuable discussion on \emph{stackexchange},~\cite{stackexchange}. This provides an insightful overview of the role of fields as representations of the Poincaré group, both before and after the quantisation.

We now introduce some basic concepts of group theory which will be used throughout the discussion.

\chapter{Group Theory Basics}
\section{Lie Groups and Algebra}\label{sec:lie-groups-algebras}
\begin{definition}[Group]\label{def:group-axioms}
    A \emph{group} $(G,\circ)$ is a set $G$ equipped with a composition map,
    \begin{equation}
        \circ \colon G \times G \to G, \quad (g,h) \mapsto g \circ h \in G ,
    \end{equation}
    called \emph{group multiplication}, which satisfies the following axioms:
    \begin{enumerate}
        \item \emph{Closure}: $\forall g_1, g_2 \in G \implies g_1 \circ g_2 \in G$.
        \item \emph{Associativity}: $\forall g_1, g_2, g_3 \in G \implies (g_1 \circ g_2) \circ g_3 = g_1 \circ (g_2 \circ g_3)$.
        \item \emph{Neutral element}: $\exists e \in G$ such that $\forall g \in G \implies g \circ e = e \circ g = g$.
        \item \emph{Inverse element}: $\forall g \in G, \exists g^{-1}$ such that $g \circ g^{-1} = g^{-1} \circ g = e$.
    \end{enumerate}
    If, in addition, $\forall g_1, g_2 \in G \implies g_1 \circ g_2 = g_2 \circ g_1$, then $G$ is said to be \emph{abelian}.
\end{definition}

\begin{definition}[Subgroup]
    Given a group $(G, \circ)$, a \emph{subgroup} $(H,\circ)$ is a subset $H \subset G$ that satisfies the group axioms with the composition inherited from $G$.
\end{definition}

\begin{remark}
    For $H$ to be a subgroup, it must contain the same identity element $e \in G$. While abelian group admits \emph{only} abelian subgroups, a non abelian group \emph{can} have both abelian and non abelian subgroups.
\end{remark}

A group can be \emph{finite} or \emph{infinite} based on its number of elements. The \emph{dimension} of a group $G$, denoted by $\dim(G)$, represents the number of \emph{real} parameters required to specify an element of $G$.

\begin{definition}[Group homomorphism and isomorphism]\label{def:group-homomorphism}
    A \emph{group homomorphism} is a function between two groups which preserves the group structure. For groups $(G_1, \circ)$ and $(G_2, \bcdot)$, given $g_1, g_2 \in G_1$, a map $f\colon G_1 \to G_2$ is a homomorphism if
    \begin{equation}
        f(g_1) \bcdot f(g_2) = f(g_1 \circ g_2).
    \end{equation}

    A \emph{group isomorphism} is a bijective homomorphism.
\end{definition}

\begin{definition}[One-parameter subgroup]\label{def:one-parameter}
    A \emph{one-parameter subgroup} is a continuous group homomorphism
    \begin{equation}
        \phi\colon \R \to G ,
    \end{equation}
    where $\R$ is considered as an additive group. If $\phi$ is injective, its image $\phi(\R)$ forms a subgroup of $G$, isomorphic to $R$.
\end{definition}

\begin{definition}[Lie group]
    A \emph{Lie group} is a group whose elements depend continuously and differentiably on a set of real parameters $\theta^a$, where  $a = 1, \dots N$. Thus, a Lie group is both a group and a differentiable manifold.
\end{definition}

\begin{remark}
    While a Lie group could be defined as a Hausdorff topological group that behaves
    like a transformation group near the identity, we will focus on its manifold structure, which
    is more relevant to our purposes.
\end{remark}

Each element of the Lie group can be represented as a point on its manifold, and the dimension of the group matches that of the manifold. We denote a generic element as $g(\theta)$ and choose the coordinates $\theta^a$ such that the identity element $e$ of the group corresponds to $\theta^a = 0$, i.e., $g(0) = e$. This structure allows us to expand group elements in a Taylor-like series and consider elements infinitesimally close to the identity. The set of these “infinitesimal” elements forms the tangent space at the identity $e$, which is the basis of the Lie algebra. 

To clarify, we introduce the following definitions.

\begin{definition}[Tangent space]
    Let $\M$ be a manifold. For each point $x \in \M$, the \emph{tangent space} at $x$, $T_x \M$, is the space of tangent vectors
    \begin{equation}
        v = \left. \frac{\ud}{\ud t}\right|_{t=0} \gamma(t)  \in T_x \M,
    \end{equation}
    where $\gamma\colon \R \to \M$ is any curve on the manifold passing through $x$. The dimension of $T_xM$ matches that of $M$.
\end{definition}

\begin{definition}[Lie algebra]
    For a Lie group $G$ with identity $e \in G$, the \emph{Lie Algebra} $\g$ is the tangent space at the identity, $\g = T_e G$.
\end{definition}


Let's now introduce some definitions and theorems which will be useful later for studying the properties of the Poincaré group. Recall that, as a manifold, a Lie group is also a topological space, allowing us to apply the results developed for such spaces. However, whenever possible, we'll specify concepts in terms of the manifold structure upon the topological space, to provide greater clarity and focus on the applications of interest.

\begin{definition}[Connected space]\label{def:connected}
    A topological space is said to be \emph{connected} if it can't be represented as the union of two or more disjoint non-empty open subsets.
\end{definition}

\begin{definition}[Connected component]\label{def:connected-component}
    For a group $G$, the \emph{connected component} (also known as \emph{identity component} or \emph{unity component}), is the largest connected subgroup of $G$ containing the identity element.
\end{definition}

\begin{definition}[Simply connected]\label{def:simply-connected}
    A Lie group $G$ is \emph{simply connected} if any two paths $\gamma(t)$, $\gamma'(t)$, which share the same endpoints, can be continuously deformed into one another.
\end{definition}

The Lorentz group, for example, has four connected components, of which we choose the proper orthochronous part. Since it's not simply connected, the following theorem becomes relevant.

\begin{theorem}\label{th:universal-cover}
    If $G$ is \emph{not} simply connected, there exists another Lie group $\tilde{G}$ with \emph{isomorphic} Lie algebra, $\tilde{\g} \cong \g$, which is simply connected. This $\tilde{G}$ is called \emph{universal cover} of $G$. There is thus a projection map $\pi \colon \tilde{G} \to G$ which is a surjective group homomorphism.
\end{theorem}

Thus, the algebras of $\tilde{G}$ and $G$ coincide, i.e., $\tilde{\g} \cong \g$. This means that for each point $g \in \tilde{G}$, there exists an open neighbourhood $g \in U \subset \tilde{G}$ such that the restriction $\pi|_U \colon U \to \Phi(U)$ is a diffeomorphism\footnote{A \emph{diffeomorphism} is a smooth invertible map.}. Consequently, the tangent space at $g \in \tilde{G}$ is isomorphic to the tangent space at $\pi(g) \in G$, including the tangent space at the identity.

For the proper orthochronous Lorentz group, the universal cover is the spin group, which is isomorphic to $SL(2;\C)$.

\begin{definition}[Compact space]\label{def:compact}
    A topological space $X$ is \emph{compact} if every open cover of $X$ has a finite subcover. Roughly speaking, a Lie group is compact if its parameter space is bounded.
\end{definition}

\begin{theorem}
    Every compact Lie group is isomorphic to a matrix group.
\end{theorem}

Due to the presence of boosts, the Lorentz group is \emph{non}-compact, which implies that its unitary representations must be infinite-dimensional, as we'll shortly see. This property directs us toward studying representations on one-particle Hilbert spaces.

We now introduce the exponential map, a key tool for obtaining elements of $G$ from its Lie algebra $\g$.

\begin{definition}[Exponential map]\label{def:exponential-map}
    Let $G$ be a Lie group and $\g$ be its Lie algebra. The \emph{exponential map} is defined as
    \begin{equation}\label{eq:exp-map}
        \exp\colon \g \to G ,
    \end{equation}
    where for $X \in \g$, $\exp(X) = \gamma(1)$, with $\gamma\colon \R \to G$ the unique one-parameter subgroup of $G$ (see def.~\ref{def:one-parameter}) whose tangent vector at the identity is equal to $X$.
\end{definition}

The exponential map, putting into correspondence a Lie algebra and the associated Lie group, allows one to simplify the study of the Lie group itself, since the Lie algebra is a linearised version of the latter. In this sense, the Lie group can be reconstructed from its Lie algebra via the exponential map. However, there are some limitations on the information that are encoded in the Lie algebra. This can be more easily understood by looking at the path description: if one considers a curve on the Lie group passing through the identity, such a path is differentiable (and, hence, continuous) as long as it does not encounter discontinuities. The presence of these discontinuities involves a substantial distinction between the various submanifolds that can be defined on a Lie group. In particular, since the objective is to reconstruct a Lie group and not just a submanifold, it is necessary to identify the region that contains the neutral element, i.e., a neighbourhood of the identity, with the additional requirement that on this part of the Lie group the exponential map is a differentiable curve. The submanifold that satisfies these properties corresponds to the component connected with the identity, which is a Lie subgroup of the original Lie group. This implies that, given a Lie algebra, it is possible to reconstruct only the component of the Lie group that is connected with the identity.

Without delving too deeply into the details, let's note that, in general, the image of the exponential map, $\Im(\exp) \subseteq G$, is a \emph{neighbourhood} of $e$, and the map itself is \emph{not} necessary surjective. However, for a group $G$ that is both \emph{connected} (def.~\ref{def:connected}) and \emph{compact} (def.~\ref{def:compact}), the exponential map \emph{is} surjective, i.e., $\Im(\exp) = G$. The issue with the Lorentz group is that, even when considering only the connected component, specifically the proper orthochronous Lorentz group, it is non-compact, due to the presence of boosts. Therefore, a priori, one can't be sure that exponentiating the Lie algebra elements will allow us to reach all the elements of the group. Demonstrating the surjectivity of the exponential map
\begin{equation}
    \exp \colon \mathfrak{so}(1,d) \to SO^+(1,d)
\end{equation}
for the proper orthochronous Lorentz group in general spacetime dimensions $D = d + 1$ is a rather technical and challenging subject. We'll simply assume this property, so that for the cases of interest in this document, we can safely affirm that through the exponential map we can obtain each element of the connected component of $G$ from its Lie algebra $\g$. 

The Lie algebra multiplication, called \emph{Lie bracket} or \emph{commutator}, is denoted by
\begin{equation}
    \comm{\cdot}{\cdot} \colon \g \times \g \to \g, \quad (X,Y) \mapsto \comm{X}{Y}.
\end{equation}
It is an antisymmetric multiplication
\begin{equation}
    \comm{\alpha X + \beta Y}{Z} = -\comm{Z}{\alpha X + \beta Y} = \alpha \comm{X}{Z} + \beta \comm{Y}{Z},
\end{equation}
and satisfies the \emph{Jacobi identity}
\begin{equation}
    \comm{X}{\comm{Y}{Z}} + \comm{Y}{\comm{Z}{X}} + \comm{Z}{\comm{X}{Y}} = 0.
\end{equation}

Further, for the connected component, the \emph{Baker-Campbell-Hausdorff} (BCH) formula relates the group composition with the Lie algebra multiplication,
\begin{equation}\label{eq:BCH-formula}
    \exp(X) \circ \exp(Y) = \exp\left( X + Y + \frac{1}{2}\comm{X}{Y} + \frac{1}{12} \comm{X}{\comm{X}{Y}} - \frac{1}{12} \comm{Y}{\comm{X}{Y}} + \dots \right).
\end{equation}

Given a vector space basis $\{T_a\}$ of a Lie algebra $\g$, the Lie bracket structure can be encoded in terms of the \emph{structure constants} $\tensor{f}{_a_b^c}$, where,
\begin{equation}\label{eq:structure-contants}
   \comm{T_a}{T_b} = i \tensor{f}{_a_b^c} T_c.
\end{equation}
Any consistent set of structure constants then define a Lie algebra. Through the exponential map and the BCH formula, we can express elements of the Lie group in terms of exponentials of linear combinations of the basis elements $\{T_a\}$, called \emph{generators} of the Lie algebra. In particular, for any $g \in G$, we have
\begin{equation}\label{eq:temp-7}
    g(\theta) = e^{i \theta^a T_a},
\end{equation}

It is a standard exercise to start from $g(\alpha) g(\beta) g^{-1}(\alpha) g^{-1}(\beta) \equiv g(\theta)$ and expand it near the identity up to the second order, to obtain

\begin{equation}
    g(\alpha) g(\beta) g^{-1}(\alpha) g^{-1}(\beta) \simeq \1 - \alpha_a \beta_b \comm{T^a}{T^b} = \1 + i \theta_a T^a.
\end{equation}

\begin{proof}
    Using eq.~\eqref{eq:temp-7} and expanding the exponential up to the second order, we obtain
\begin{equation*}
\begin{split}
    &\left( \1 + i \alpha_i T^i - \frac{1}{2} (\alpha_i T^i) (\alpha_j T^j)\right)
    \left( \1 + i \beta_i T^i - \frac{1}{2} (\beta_i T^i) (\beta_j T^j)\right) \\
    &\left( \1 - i \alpha_i T^i - \frac{1}{2} (\alpha_i T^i) (\alpha_j T^j)\right)
    \left( \1 - i \beta_i T^i - \frac{1}{2} (\beta_i T^i) (\beta_j T^j)\right) \\
    &\simeq \left( \1 + i (\beta T) - \frac{1}{2}(\beta T) (\beta T) + i (\alpha T) - (\alpha T) (\beta T) - \frac{1}{2} (\alpha T)(\alpha T) \right) \\
    & \quad \left( \1 - i (\beta T) - \frac{1}{2}(\beta T) (\beta T) - i (\alpha T) - (\alpha T) (\beta T) - \frac{1}{2} (\alpha T)(\alpha T) \right) \\
    &\simeq \1 - i(\beta T) - \frac{1}{2} (\beta T) (\beta T) - i (\alpha T) - (\alpha T)(\beta T) - \frac{1}{2}(\alpha T)(\alpha T) + i (\beta T) + (\beta T)(\beta T) + \\ &(\beta T)(\alpha T) - \frac{1}{2} (\beta T)(\beta T) - i (\alpha T) + (\alpha T)(\beta T) + (\alpha T)(\alpha T) - (\alpha T)(\beta T) - \frac{1}{2}(\alpha T)(\alpha T) \\
    &\simeq \1 - \alpha_i \beta_j (T^i T^j - T^j T^i) = \1 - \alpha_i \beta_j \comm{T^i}{T^j} = \1 + i \theta_k T^k , \qedhere
\end{split}
\end{equation*}
\end{proof}
where, by means of~\eqref{eq:structure-contants}, we defined $\alpha_a \beta_b \tensor{f}{^a^b_c} = -\theta_c$. So, a group is \emph{abelian} if all its \emph{generators commute}, i.e., $\comm{T^a}{T^b} = 0$, or rather, $\theta_a = 0$.

To conclude this section, let us provide a few definitions to clarify what is meant by an
isomorphism between Lie algebras. This will help to better grasp in what sense the universal cover of a group and the group itself share “the same algebra”.

\begin{definition}[Lie algebra homomorphism]\label{def:algebra-homomorphism}
    Let $\g$ and $\h$ be two Lie algebras. A \emph{Lie algebra homomorphism} from $\g$ to $\h$ is a linear map $\phi \colon \g \to \h$ that preserves the Lie bracket, meaning
    \begin{equation}
        \comm{\phi(X)}{\phi(Y)}_\h = \phi(\comm{X}{Y}_\g), \quad \forall X,Y \in \g.
    \end{equation}
\end{definition}

\begin{definition}[Lie algebra isomorphism]
    A \emph{Lie algebra isomorphism} is a Lie algebra homomorphism which is also a vector space isomorphism.
\end{definition}

Recall that we defined the Lie algebra of a Lie group as the tangent space at the identity, which makes it a vector space as well. If two algebras are isomorphic, we write $\g \cong \h$, and in this case, they have the same dimension, $\dim \g = \dim \h$. Choosing bases $\{X_a\}$ for $X$ and $\{Y_a\}$ for $Y$, the structure constants must be match, so we have
\begin{gather*}
    \comm{X_a}{X_b}_\g = i \tensor{f}{_a_b^c} X_c ,\\
    \comm{Y_a}{Y_b}_\h = i \tensor{f}{_a_b^c} Y_c .
\end{gather*}

\begin{proof}
    \color{red} \dots \color{black}
\end{proof}

Therefore, theorem~\ref{th:universal-cover} implies that a non-simply connected Lie group $G$ and its universal cover $\tilde{G}$ have isomorphic algebras, $\tilde{\g} \cong \g$, sharing the same structure constants.

    

%%%%%%%%%%%%%%%% GROUP REPRESENTATIONS%%%%%%%%%%%%%%%%%%%%%%
\section{Group Representations}\label{sec:group-representation}
So far, we've treated groups and algebras as abstract entities. To apply these concepts and interpret them as symmetry transformations, we need to “represent” them concretely.

\begin{definition}[Group representation]\label{def:representation}
    A \emph{(linear) representation} of a group $G$ is a group homomorphism (see def.~\ref{def:group-homomorphism}), $\rho \colon G \to \Aut(V)$, mapping $G$ to the group of automorphisms\footnote{An \emph{automorphism} is an isomorphism from an object into itself. Here, the group of automorphisms is the set of invertible linear maps on a vector space.} on a vector space $V$. Formally
    \begin{equation}\label{eq:representation-property}
        \forall g, h \in G: \rho(e) = \id_V, \quad \rho(g \circ h) = \rho(g)\rho(h), \quad \rho(g^{-1}) = {\rho(g)}^{-1} .
    \end{equation}
\end{definition}
For finite-dimensional vector spaces, where $\dim(V) = n < \infty$, we have $\Aut(V) \cong GL(n;K)$, so elements of the group are represented by $n \times n$ matrices with multiplication defined by matrix group law. The \emph{dimension of a representation} $(\rho, V)$ is the same as the dimension of $V$.

\begin{remark}
    The dimension of a representation differs from the dimension of the group itself.
\end{remark}

From now on, we focus on finite-dimensional representations. For a given representation $(\rho, V)$, the group acts on the vectors of $V$ as linear transformations. For each $g \in G$ and $v \in V$, we have
\begin{equation}\label{eq:representation-transformation}
    v \mapsto \rho(g) v .
\end{equation}

\begin{definition}[Reducible representation]\label{def:reducible-rep}
    A representation is called \emph{reducible} if there exists a non-zero subspace $\{0\} \neq U \subseteq V$ such that
\begin{equation}
    \forall u \in U \colon \rho(g)u \in U.
\end{equation}
If no such subspace exists, $\rho$ is called an \emph{irreducible representation}.
\end{definition}

 Given a reducible representation $\rho$, we can always find a basis for $V$ such that
\begin{equation}
    \rho(g) = 
    \begin{pmatrix}
        \tilde{\rho}(g) & \beta(g) \\
        0 & \rho'(g)
    \end{pmatrix}
\end{equation}
where $U = \{ \icol{u \\ 0} \in V \}$ is the invariant subspace. This structure gives rise to a representation $\tilde{\rho}$ of smaller dimension. 

\begin{definition}[Completely reducible representation]
    A reducible representation is called \emph{completely reducible} if $\beta(g) = 0$. In this case, $\rho$ decomposes into the direct sum of two representations,
\begin{equation}
    \rho \cong \tilde{\rho} \oplus \rho' .
\end{equation}
\end{definition}

In other words, in a completely reducible representation, the basis vectors of $V$ can be chosen to split into subsets that remain independent under the transformation.~\eqref{eq:representation-transformation}

\begin{definition}[Equivalent representations]
    Two representations $\rho_1$ and $\rho_2$ of the same dimension $n$ are called \emph{equivalent} if there exists an invertible $n \times n$ matrix $S$ such that
    \begin{equation}
         \forall g \in G: \rho_2(g) = S^{-1} \rho_1(g) S,
    \end{equation}
    Thus, if there exists a change of basis $S$ on $V$ relating the representations, they are equivalent.
\end{definition}

\begin{definition}[Faithful representation]
    A representation $\rho$ is called \emph{faithful} if 
\begin{equation}
    g_1 \neq g_2 \implies \rho(g_1) \neq \rho(g_2) .
\end{equation}
For a \emph{non-faithful} representation, there exists a subset $H \subset G$ for which $\rho(h) = 1$ for $h \in H$.
\end{definition}

\begin{definition}[Unitary representation]
    A \emph{unitary representation} is a complex representation, $\rho \colon G \to GL(n;\C)$, where $\rho(g)$ is a unitary matrix, meaning,
\begin{equation}
    \rho(g^{-1}) = {\rho(g)}^{-1} = {\rho(g)}^\dagger .
\end{equation}
\end{definition}


To conclude this section, we introduce specific types of representations that will be used in the following.

\begin{definition}[Matrix representation]
    Let $(\rho, V)$ be a representation. In a \emph{matrix representation}, $V$ is a finite-dimensional vector space $(\textup{dim} V = n)$, and each group element $g \in G$ is represented by an $n \times n$ matrix $\tensor{\rho(g)}{^i_j}$, where $i,j = 1, \dots, n$. For any vector $v = (v^1, \dots, v^n) \in V$, the action of $g$ on $V$ is given by
    \begin{equation}
        v^i \mapsto \tensor{\rho(g)}{^i_j} v^j .
    \end{equation}
\end{definition}

\begin{definition}[Fundamental representation]
    The \emph{fundamental representation} of a group $G$ is the representation $D$ such that, for any $v \in V$,
    \begin{equation}
        \quad D(g) v = g v, \quad D(T^a) = T^a.
    \end{equation}
\end{definition}

\begin{definition}[Conjugate representation]
    The \emph{conjugate representation} $\bar{D}$ is defined by
    \begin{equation}
        \forall v \in V, \quad \bar{D}(g)v = g^* v, \quad g^* = {\left(e^{i {\theta}_a T^a}\right)}^* = e^{-i \theta_a {(T^a)}^*} \implies \bar{D}(T^a) = - (T^a)^* .
    \end{equation}
\end{definition}

\begin{definition}[Adjoint representation]
    \color{red} The generators are \dots \color{black}
\end{definition}



%%%%%%%%%%%%%%%%%%%%%%% ALGEBRA REPRESENTATIONS %%%%%%%%%%%%%%%%%%%%%%%%%%
\section{Algebra Representations}
We can similarly define a representation for a Lie algebra $\g$. Note that a Lie algebra can be defined independently of any associated Lie group, simply as a vector space $\g$ with an antisymmetric product, the Lie bracket $\comm{\cdot}{\cdot}$.

\begin{definition}[Algebra representation]
    Given an algebra $\g$, a representation of $\g$ is a vector space $V$ with an algebra homomorphism (see def.~\ref{def:algebra-homomorphism})
    \begin{equation}
        \rho_\g \colon \g \to \textup{End}(V),
    \end{equation}
    where $\textup{End}(V)$ denotes the set of endomorphisms of $V$.\footnote{An \emph{endomorphism} is a linear map from $V$ to $V$, not necessarily invertible.}
\end{definition}

The space of endomorphisms, $\textup{End}(V)$, has a natural vector space structure defined by the addition of linear maps and a non-necessarily invertible product defined by composition. With a chosen basis for $V$, $\textup{End}(V)$ becomes a space of matrices, with matrix multiplication as the product.

For compatibility with the Lie algebra structure, the representation $\rho_\g$ must satisfy
\begin{equation}
    \rho_\g \left( \comm{X}{Y}  \right) = \rho_\g (X) \rho_\g (Y) - \rho_\g (Y) \rho_\g (X) ,
\end{equation}
where the product on the right-hand side represents matrix multiplication. Given a set of generators $\{T_a\}$, with structure constants $\tensor{f}{_a_b^c}$, we have
\begin{equation}
    \rho_\g \left( \comm{T_a}{T_b} \right)  = \tensor{f}{_a_b^c} \rho_\g (T_c) .
\end{equation}

We can now demonstrate that \emph{any representation} $(\rho, V)$ \emph{of a Lie group} $G$ \emph{induces a representation of its Lie algebra} $\g$. If $G \ni g = \exp(tX)$, for $t\in \R$ and $X \in \g$, then $\rho \left(exp(tX)\right)$ defines a “path” of transformations on the representation space $V$. We can then define a representation of $\g$ on the \emph{same} space $V$, via
\begin{equation}
    \forall v \in V : \rho_\g(X)(v) \coloneq \left. \left[ \frac{\ud}{\ud t} \rho \left( \exp(tX)(v) \right) \right] \right|_{t=0} .
\end{equation}
Hence, $\rho_\g(X)$ is a matrix of the same size as $\rho(g)$, and it also acts on $V$. One can show explicitly, using the BCH formula, that $\rho_\g$ respects the bracket structure. Thus, $(\rho_\g, V)$ is a representation of the Lie algebra.
\begin{proof}
    \color{red} Using BCH formula \dots \dots, then by definition \dots we have a representation \color{black}
\end{proof}

Let's then consider a presentation of the Lie group $G$, $\rho(g(\theta)) \coloneq \rho(\theta)$ and denote the generators of the group in the representation $\rho$ as $\rho_\g (T^a) \coloneq T^a_\rho$. By assumption of smoothness, in the neighbourhood of the identity,
\begin{equation}\label{eq:generators-rep}
    \rho(\theta) \simeq \1 + i \theta_a T_\rho^a,
\end{equation}
with 
\begin{equation}
    T_\rho^a \equiv \left. -i \frac{\partial \rho}{\partial \theta_a}\right|_{\theta=0}.
\end{equation}
For the component of the group manifold connected to the identity, a generic group element $g(\theta)$ can always be represented by
\begin{equation}\label{eq:exp-map-rep}
    \rho(g(\theta)) = e^{i \theta_a T_\rho^a} .
\end{equation}

\begin{remark}
    While the explicit form of the generators, $T_\rho^a$, depends on the representation, the structure constants $\tensor{f}{_a_b^c}$ of eq.~\eqref{eq:structure-contants} are independent of the representation.
\end{remark}

Conversely, not all Lie algebra representations $\rho_\g$ necessarily extend to representations of the corresponding group $G$.  This discrepancy arises from the \emph{global topology} of $G$. In particular, all representations of the Lie algebra extend to group representations if $G$ is \emph{simply connected} (see def.~\ref{def:simply-connected}). According to theorem~\ref{th:universal-cover}, if $G$ is not simply connected, there exists a universal cover $\tilde{G}$ that is simply connected and has an isomorphic algebra, $\tilde{\g} \cong \g$. Consequently, knowing the structure constant of the algebras $\tilde{\g}$ or $\g$, which are the same, we can construct representations of the algebras, which will be extended to representation of the covering group $\tilde{G}$.

As an example, the Lorentz group is \emph{not} simply connected, and its universal cover is $SL(2;\C)$. Theregore, not all representations of its Lie algebra extend to representations of the group.

We now state two important theorems that will be essential in identifying physical observables in quantum mechanics.
\begin{theorem}\label{th:unitary-rep}
    \emph{All} unitary projective representations of a group $G$ arises from a unitary \emph{linear} representation of the universal covering group $\tilde{G}$. These, in turn, come from representations of the Lie algebra.
\end{theorem}


\begin{theorem}\label{th:non-compact-group-rep}
    Non-compact groups have no unitary representations of finite dimension, except for representations in which non-compact generators are represented trivially, i.e., as zero.
\end{theorem}

\color{red} Are those valid in infinite dimension?\color{black}

The physical relevance of this second theorem is due to the fact that in a unitary representation, the generators are Hermitian operators. According to the principles of quantum mechanics, only Hermitian operators correspond to observables. Therefore, for a non-compact group, in order to identify its generators with physical observables we need an infinite-dimensional representation. This requirement leads us to consider representations on the Hilbert space of one-particle states, as we will explore.



%%%%%%%%%%%%%%%% CASIMIR OPERATORS %%%%%%%%%%%%%%%
\section{Casimir Operators}\label{sec:casimir}
Casimir operators play a significant role in the study of representations. These operators are constructed from the generators Ta of a Lie algebra and commute with all generators themselves. In each irreducible representation, Casimir operators are proportional to the identity matrix, with the proportionality constant used to label the representation.

\chapter{Symmetries in Quantum Mechanics}\label{sec:symmetries-qm}
In this section, we review the concepts of states and observables in quantum mechanics, with a focus on the role of projective representations as opposed to standard representations.

Let us begin by outlining some fundamental aspects of quantum mechanics.
\begin{itemize}
    \item \emph{Physical states} are represented by \emph{rays} in a Hilbert space $\H$, which is a complex vector space with a \emph{scalar product} $\langle \cdot, \cdot \rangle$, defined by
    \begin{subequations}\label{eq:scalar-product}
        \begin{align}
        \scalar{\phi}{\psi} &= \scalar{\psi}{\phi}^* , \\
        \scalar{\phi}{\xi_1 \psi_1 + \xi_2 \psi_2} &= \xi_1 \scalar{\phi}{\psi_1} + \xi_2 \scalar{\phi}{\psi_2} ,\\
        \scalar{\eta_1 \phi_1 + \eta_2 \phi_2}{\psi} &= {\eta_1}^* \scalar{\phi_1}{\psi} + {\eta_2}^* \scalar{\phi_2}{\psi} ,
        \end{align}
        \end{subequations}
    with $\phi,\psi \in \H$. The \emph{norm} is defined as $\mynorm{\psi} \coloneq \braket{\psi}{\psi} \geq 0$, and it vanishes if and only if $\psi \equiv 0$. A \emph{ray} $\ray$ is a set of normalised vectors (satisfying $\braket{\psi}{\psi} = 1$), where $\psi, \psi' \in \ray$ if $\psi' = \xi \psi$, for some $\xi \in \C$ with $\abs{\xi} = 1$.
    \item \emph{Observables} are represented by \emph{hermitian operators} $A$ on $\H$, defined by the properties
    \begin{subequations}
    \begin{align}
        A (\xi \psi + \eta \phi) &= \xi A\psi + \eta A \phi ,\\
        A &= A^\dagger ,
    \end{align}
    \end{subequations}
    where the adjoint of an operator $A$ is defined by
    \begin{equation}\label{eq:def-adjoint}
        \scalar{\phi}{A^\dagger \psi} = \scalar{A\phi}{\psi} = \scalar{\psi}{A\phi}^* .
    \end{equation}

    Let's consider a physical state represented by a ray $\ray$, and an observable represented by the hermitian operator $A$. Then, the state has a definite value of $\alpha$ for the observable if vectors $\psi \in \ray$ are \emph{eigenstates} of $A$ of \emph{eigenvalue} $\alpha$:
    \begin{equation}
        A\psi = \alpha \psi, \quad \textup{for} \; \psi \in \ray .
    \end{equation}

    If the operator is hermitian, i.e., $A^\dagger = A$, then $\alpha \in \R$ and eigenstates with different eigenvalues are orthogonal with respect to each other.
    \item If a system is in a state represented by a ray $\ray$, and an experiment tests if it is in any one of the  mutually orthogonal rays\footnote{A pair of rays $\ray_1, \ray_2$ are orthogonal if $\scalar{\psi_1}{\psi_2} = 0, \; \forall \psi_1 \in \ray_1,  \psi_2 \in \ray_2$} $\ray_1, \ray_2, \dots$, then the probability of finding it in the state represented by $\ray_n$ is 
    \begin{equation}
        P(\ray \to \ray_n) = \abs{\scalar{\psi}{\psi_n}}^2 ,
    \end{equation}
    where $\psi \in \ray$ and $\psi_n \in \ray_n$. If $\psi_n$ form a complete set, then $\sum_n P(\ray \to \ray_n) = 1$.
\end{itemize}

Now, a \emph{necessary} condition for a transformation acting on rays to be a \emph{symmetry}, is that, if an observer $O$ sees a state represented by a ray $\ray$, or $\ray_1$, or $\ray_2$, \dots, then another observer $O'$ sees the same system in a different state, represented by a ray $\ray'$, or $\ray'_1$, or $\ray'_2$, \dots, but the two must find the same probabilities:
\begin{equation}
    P(\ray \to \ray_n) = P(\ray' \to \ray'_n) .
\end{equation}

A fundamental result, \emph{Wigner's theorem}, states that for any such rays transformation, $T \colon \ray \to \ray'$, there exists an operator $U$ on the Hilbert space, such that if $\psi \in \ray$, then $U\psi \in \ray'$, with $U$ \emph{unitary} and \emph{linear}\footnote{To be honest, it could be \emph{antiunitary} and \emph{antilinear} as well, but this last case is not of physical interest.}
\begin{subequations}
\begin{align}
    \scalar{U\psi}{U\psi} &= \scalar{\psi}{\psi} \label{eq:def-unitary} ,\\ 
    U(\xi \psi + \eta \psi) &= \xi U \psi + \eta U \psi .
\end{align}
\end{subequations}
Recalling the definition of an adjoint operator~\eqref{eq:def-adjoint}, the unitarity condition~\eqref{eq:def-unitary} can be written as
\begin{equation}
    U^\dagger = U^{-1} .
\end{equation}

The identity operator $U = \1$ trivially represents a symmetry, preserving the ray $R$. An infinitesimal symmetry transformation close to the identity is represented by a unitary operator near $U = \1$:
\begin{equation}
    U = 1 + i \epsilon T ,
\end{equation}
where $\epsilon$ is a real infinitesimal parameter. For $U$ to be unitary, $T$ must be both \emph{linear} and \emph{hermitian}, qualifying it as a potential observable. In physics, most observables arise in this way from symmetry transformations.

It is straightforward to verify that symmetry transformations from one ray to another satisfy the group axioms~\ref{def:group-axioms}. Indeed, given $T_1 \colon \ray_n \to \ray'_n$ and $T_2 \colon \ray'_n \to \ray''_n$, then $T_2 T_1$ is another symmetry transformation, $T_2 T_1 \colon \ray_n \to \ray''_n$. The inverse will be $T^{-1} \colon \ray'_n \to \ray_n$ and the identity $T=1$ leaves the ray unchanged.

The unitary operators $U(T)$ corresponding to these symmetry transformations have properties that mirror the group structure (see def.~\ref{def:representation}), but unlike the symmetry transformations themselves, the operators $U(T)$ act on vectors in Hilbert space, rather than on rays. If $T_1 \colon \ray_n \to \ray'_n$, then $U(T_1)$ acts on $\psi_n \in \ray_n$ and gives a vector $U(T_1)\psi_n \in \ray'_n$. Similarly, if $T_2 \colon \ray'_n \to \ray''_n$, then $\ray'_n \ni U(T_1)\psi_n \mapsto U(T_2)U(T_1)\psi_n \in \ray''_n$. However, $U(T_2 T_1)$ is also in this ray, since $T_2 T_1 \colon \ray_n \to \ray''_n$, so the vectors $U(T_2)U(T_1)\psi_n$ and $U(T_2T_1)\psi_n$ must differ by a phase factor $\phi_n (T_2, T_1)$
\begin{equation}
    U(T_2)U(T_1)\psi_n = e^{i\phi (T_2, T_1)} U(T_2T_1)\psi_n ,
\end{equation}
where we dropped the index $n$ for the phase since one can prove that it doesn't depend on the state (see Weinberg~\cite{weinberg}).

This type of representation is called \emph{projective representation}. Unlike a linear representation~\eqref{eq:representation-property}, it includes a phase factor. Turning back to the notation of sec.~\ref{sec:group-representation}, a symmetry group $G$ acts on the Hilbert space $\H$ via a unitary operator $\rho(g) \colon \H \to \H$, for $g \in G$, which defines a projective representation:
\begin{equation}
    \rho(g) \rho(h) = e^{i\phi(g,h)} \rho(g \circ h), \quad \phi(g,h) \in R.
\end{equation}

Therefore, in a quantum theory, we're interested in \emph{unitary projective representations} of a symmetry group $G$. Then, according to theorem~\ref{th:unitary-rep}, \emph{all} unitary projective representations of $G$ can be derived from unitary \emph{linear} representations of the universal covering group $\tilde{G}$. These, in turn, arise from representations of the Lie algebra. Further, by theorem~\ref{th:non-compact-group-rep}, since the Poincaré group is non-compact, we must study infinite dimensional representations in order to have unitary ones, which can then be associated with physical observables.

\chapter{The Groups \texorpdfstring{$SO(3)$}{SO(3)} and \texorpdfstring{$SU(2)$}{SU(2)} and their Representations}
\color{blue}
In this section we review the main properties of the Lie groups $SO(3)$ and $SU(2)$ and their algebra, since it'll be useful to study the representations of the Lorentz group.

%%%%%%%%%%%%%%%%%%%%%%%%%%% SO(3) GROUP %%%%%%%%%%%%%%%%%%%%%%%%%%%
\section{The Group \texorpdfstring{$SO(3)$}{SO(3)}}
\color{red} ciao There's an error here, concerning the exponential map and the generators. What it's written refers to the exponential map with no $i$ factor. Taking into consideration, we should have $i$ factors on infinitesimal rotation and the algebra should be made of hermitian matrices. \color{blue}

In its fundamental representation, the \emph{special unitary group} $SO(n)$ is composed of $n \times n$ invertible matrices which satisfy
\begin{equation}\label{eq:so3-def}
    R \in SO(n) : R^T R = R R^T = \1, \; \det(R) = 1 .
\end{equation}
It's then easy to see that the dimension is $\dim(SO(n)) = \frac{n(n-1)}{2}$. We're interested, in particular, in the case $n=3$, which is the $3$ dimensional $SO(3)$ group.

To determine the Lie algebra $\mathfrak{so}(3)$, note that any $R \in SO(3)$ can be parametrized by three Euler angles or, equivalently, by specifying a rotation axis and an angle around it. In particular, if we consider a unit vector $\vec{n} \in \R^3$, normalised such that $\vec{n} \cdot \vec{n} = 1$, and we call $\theta$ the angle around it, the identity is given by $R(\theta = 0, \vec{n}) = \1$. Then, for an infinitesimal angle $\delta \theta$, the action on an arbitrary vector $\vec{x} \in \R^3$ is an infinitesimal rotation given by
\begin{equation}
    R(\delta \theta, \vec{n}) \vec{x} = \vec{x} + \delta \theta \vec{n} \times \vec{x} = \vec{x} + \delta \theta 
    \begin{pmatrix}
        n_2 x_3 - n_3 x_2 \\
        n_3 x_1 - n_1 x_3 \\
        n_1 y_2 - n_2 y_1
    \end{pmatrix}.
\end{equation}

If we now pich $\vec{n}$ to be the Euclidean basis vectors, we obtain
\begin{subequations}\label{eq:infinitesimal-rotation}
\begin{align}
    R(\delta \theta, \vec{n} = (1,0,0)^T) \vec{x} &= \vec{x} + \delta \theta \begin{pmatrix} 0 \\ -x_3 \\ x_2 \end{pmatrix} = \vec{x} + \delta \theta 
    \begin{pmatrix}
        0 & 0 & 0 \\
        0 & 0 & -1 \\
        0 & 1 & 0
    \end{pmatrix}
    \vec{x} \coloneq \vec{x} + \delta \theta L_1 \vec{x} ,\\
    R(\delta \theta, \vec{n} = (0,1,0)^T) \vec{x} &= \vec{x} + \delta \theta \begin{pmatrix} x_3 \\ 0 \\ -x_1 \end{pmatrix} = \vec{x} + \delta \theta 
    \begin{pmatrix}
        0 & 0 & 1 \\
        0 & 0 & 0 \\
        -1 & 0 & 0
    \end{pmatrix}
    \vec{x} \coloneq \vec{x} + \delta \theta L_2 \vec{x} ,\\
    R(\delta \theta, \vec{n} = (0,0,1)^T) \vec{x} &= \vec{x} + \delta \theta \begin{pmatrix} -x_2 \\ x_1 \\ 0 \end{pmatrix} = \vec{x} + \delta \theta 
    \begin{pmatrix}
        0 & -1 & 0 \\
        1 & 0 & 0 \\
        0 & 0 & 0
    \end{pmatrix}
    \vec{x} \coloneq \vec{x} + \delta \theta L_3 \vec{x} ,\\
\end{align}
\end{subequations}

By means of~\eqref{eq:generators-rep}, the matrices $L_i$ in eq.~\eqref{eq:infinitesimal-rotation} correspond to a basis of the Lie algebra $\mathfrak{so}(3)$. Indeed, $SO(3)$, is the connected component of $O(3)$, and the group is compact, as can be seen from the fact that the rotation angles can only take finite values. Therefore, as discussed in section~\ref{sec:lie-groups-algebras} after the definition~\ref{def:exponential-map}, the exponential map is surjective and each group element can be written as the exponential of a linear combination of the generators, as showed in eq.~\eqref{eq:exp-map-rep}. In this case, we're in the fundamental representation and the generators are represented by $L_i$.

Using the defining property~\eqref{eq:so3-def} and eq.~\eqref{eq:exp-map-rep} together with BCH formula~\eqref{eq:BCH-formula}, we get
\begin{equation*}
    \1 = R(\theta, \vec{n})^T R(\theta, \vec{n}) = \exp({i \theta \sum_i n_i \cdot ({L}_i^T + L_i) + \dots}) .
\end{equation*}
Then, for an infinitesimal transformation, it must be $L^T_i + L_i = 0$, which implies that $L_i$ are antisymmetric matrices. Further, using the formula $\det(\exp(M)) = \exp(\textup{tr}(M))$, the determinant condition $\det(R) = 1$ requires $\tr(L_i) = 0$. Therefore, we obtain that in the fundamental representation, the algebra is
\begin{equation}
    \mathfrak{so}(3) = \{ \textup{antisymmetric taceless $3 \times 3$ matrices}\},
\end{equation}
which is a three-dimensional real vector space with basis $\{L_i\}$, as anticipated.

From the explicit form of the generators, \color{red} missing $i$ factor to consider in a correct way, \color{blue} we can compute the commutation relations explicitly, finding
\begin{equation}\label{eq:su2-algebra}
    \comm{J_i}{J_j} = i \sum_k \epsilon_{ijk} J_k .
\end{equation}

%%%%%%%%%%%%%%%%%%%%%%%%%%% SU(2) GROUP %%%%%%%%%%%%%%%%%%%%%%%%%%%
\section{The Group \texorpdfstring{$SU(2)$}{SU(2)}}
\color{red} Similarly, as before, we define the group in the fundamental representation and show the algebra is the same as before, given by eq.~\eqref{eq:su2-algebra}. \color{blue}

%%%%%%%%%%%%%%%%%%%%%%%%%%% ALGEBRA REPRESENTATION %%%%%%%%%%%%%%%%%%%%%%%%%%%
\section{Representation of the Algebra \texorpdfstring{$\mathfrak{so}(3) \cong \mathfrak{su}(2)$}{isomorphic}}
Let's study the finite dimensional representations of the algebra $\mathfrak{so}(3) \cong \mathfrak{su}(2)$, defined by
\begin{equation}\label{eq:su2-commutation-relations}
    \comm{J_i}{J_j} = i \epsilon_{ijk} J_k .
\end{equation}

In particular, as discussed in section~\ref{sec:symmetries-qm}, the physically relevant ones are the unitary representations, acting on a vector space $V$ which is also a \emph{Hilbert space}, in the sense it carries a scalar product $\langle \cdot, \cdot \rangle$ defined by eq.~\eqref{eq:scalar-product}. In order to find the finite dimensional representation spaces $V$, we follow the usual quantum mechanical procedure.

We define the \emph{ladder operators}
\begin{equation}\label{eq:def-su2-ladder-op}
    J_\pm \coloneq J_1 \pm i J_2, \quad \textup{with} \quad \comm{J_3}{J_\pm} = \pm J_\pm, \quad \comm{J_+}{J_-} = 2 J_3 .
\end{equation}

We further define
\begin{equation}\label{eq:def-su2-total-ang}
    J^2 \coloneq {J_1}^2 + {J_2}^2 + {J_3}^2.
\end{equation}

From the fact that $J_i$ are Hermitian (\color{red}write it down well\color{blue}), it's easy to check that
\begin{equation}
    (J_\pm)^\dagger = J_\mp, \quad (J^2)^\dagger = J^2 .
\end{equation}

Using the commutation relations~\eqref{eq:su2-commutation-relations} and the definitions~\eqref{eq:def-su2-ladder-op} and~\eqref{eq:def-su2-total-ang}, one can show
\begin{equation}
    \comm{J^2}{J_i} = \comm{J^2}{J_\pm} = 0,
\end{equation}
so, as already discussed in section~\eqref{sec:casimir}, $J^2$ it's a Casimir operator, that is, it commutes with all the Lie algebra generators. As known from quantum mechanics, it represents the total angular momentum operator.

Because $J_3$ and $J^2$ are Hermitian matrices on a Hilbert space $V$ which commute, the spectral theorem tells us that there is an orthonormal basis $\{ \ket{\zeta, m} \}$ of $V$, i.e., $\braket{\zeta, m}{\zeta',m'} = \delta_{mm'} \delta_{\zeta\zeta'}$, which are simultaneous eigenvectors of $J_3$ and $J^2$. The labels $m$ and $\zeta$ are chosen to be the eigenvalues of that vector: $J_3 \ket{\zeta, m} = m \ket{\zeta, m}$, and $J^2 \ket{\zeta, m} = \zeta \ket{\zeta, m}$. A priori, there could be degeneracies for these labels, i.e, more than one linearly independent basis elements with the same eigenvalues $(\zeta,m)$. However, as it turns out, the action of $J_\pm$ only depends on these values, so different basis elements with the same labels would correspond to invariant one-dimensional subspaces, i.e., we could restrict to one of these and obtain a valid representation. In the following, we will therefore assume without loss of generality that the eigenspace for each pair $(\zeta,m)$ is one-dimensional.

Since $J^2$ commutes with $J_\pm$, we have
\begin{equation*}
    J^2 J_\pm \ket{\zeta, m} = \zeta J_\pm \ket{\zeta, m},
\end{equation*}
and, further
\begin{equation*}
    J_3 J_\pm \ket{\zeta, m} = (\comm{J_3}{J_\pm} + J_\pm J_3) \ket{\zeta, m} = (\pm J_\pm + m J_\pm) \ket{\zeta, m} = (m \pm 1) J_\pm \ket{\zeta, m},
\end{equation*}
which means
\begin{equation}
    J_\pm \ket{\zeta, m} = \lambda^\pm_{\zeta,m} \ket{\zeta, m} .
\end{equation}

To find the possible values for $(\zeta, m, \lambda)$, we use
\begin{equation*}
    J^2 - {J_3}^2 = {J_1}^2 + {J_2}^2 = \frac{1}{2} (J_+ J_- + J_- J_+),
\end{equation*}
to show
\begin{equation*}
\begin{split}
    \bra{\psi} J^2 - {J_3}^2 \ket{\psi} &= \frac{1}{2} \bra{\psi} J_+ J_- + J_- J_+ \ket{\psi} = \frac{1}{2} \bra{\psi} J^\dagger_- J_- + J_+^\dagger J_+ \ket{\psi} \\ &= \frac{1}{2} \left( \mynorm{J_- \ket{\psi}}^2 + \mynorm{J_+ \ket{\psi}}^2  \right) \geq 0 ,
\end{split}
\end{equation*}
for any $\ket{\psi} \in V$. So, this implies
\begin{equation*}
    \bra{\zeta, m } J^2 - {J_3}^2 \ket{\zeta,m} = (\zeta - m^2) \braket{\zeta,m}{\zeta,m} \geq 0 .
\end{equation*}

Since $J_+$ increases the value of $m$ for fixed $\zeta$, there must be a maximal value, $m_\textup{max}(\zeta) \equiv m_\textup{max}$ such that $\ket{\zeta, m_\textup{max}} \neq 0$, but $J_+ \ket{\zeta, m_\textup{max}} = 0$. In particular, this means
\begin{equation*}
    0 = J_- J_+ \ket{\zeta, m_\textup{max}} = (J^2 - {J_3}^2 - J_3) \ket{\zeta, m_\textup{max}} = (\zeta - m^2_\textup{max} - m_\textup{max}) \ket{\zeta, m_\textup{max}} .
\end{equation*}

Likewise, because $J_-$ decreases the value of $m$, and thus also increasing $m^2$ once $m$ becomes negative, there is also a $\ket{\zeta, m_\textup{min}} \neq 0$ such that $J_- \ket{\zeta, m_\textup{min}} = 0$. Analogously, we get
\begin{equation*}
    0 = J_+ J_- \ket{\zeta, m_\textup{min}} = (J^2 - {J_3}^2 - J_3) \ket{\zeta, m_\textup{min}} = (\zeta - m^2_\textup{min} - m_\textup{min}) \ket{\zeta, m_\textup{min}} .
\end{equation*}

All together, this means
\begin{equation}
    \zeta = m_\textup{max} (m_\textup{max} + 1) = m_\textup{min} (m_\textup{min} - 1),
\end{equation}
which has solutions
\begin{equation*}
    m_\textup{max} = m_\textup{min} - 1 < m_\textup{min} \quad \textup{and} \quad m_\textup{max} = - m_\textup{min}.
\end{equation*}
The first is clearly not compatible with the assumption that $m_\textup{max} \geq m_\textup{min}$, so the second must hold. Moreover, since we can increase the value of $m$ in steps of $1$ by acting with $J_+$, $m_\textup{max} - m_\textup{min} = 2 m_\textup{max}$ must be an integer. The common notation in the literature is to define $j = m_\textup{max}$, which is in general a non-negative integer or half-integer that is otherwise not constrained.

Then, by convention, we replace the label $\zeta = j (j+1)$ by $j$, with $j \in \frac{\N_0}{2}$ and $m \in \{ -j, -j+1, \dots, j-1, j \}$. In summary, any representation space $V$ of $\mathfrak{so}(3) \cong \mathfrak{su}(2)$ has an orthonormal basis $\ket{j,m} \equiv \ket{\zeta, m}$ characterised by
\begin{equation}
    J^2 \ket{j,m} = j(j+1) \ket{j,m}, \quad J_3 \ket{j,m} = m \ket{j,m} .
\end{equation}

If $j \in \N_0$, then $m \in \Z$, and if $j \in \N$, then $m \in \Z + \frac{1}{2}$. Finally, there were the coefficients $\lambda^\pm_{\zeta, m} \equiv \lambda^\pm_{j,m}$, that remain to be determined. One can show
\begin{equation}
    J_\pm \ket{j,m} = \sqrt{(j \mp m)(j \pm m + 1)} \ket{j,m \pm 1} .
\end{equation}
\begin{mdframed}
\begin{innerproof}
    .
\end{innerproof}
\end{mdframed}

Notice that the value of $j$ is not changed by the action of the Lie algebra generators. It means that different values of $j$ correspond to different representations, which have dimensions $2j +1$. Given the explicit action of the Lie algebra generators on the basis elements, one can explicitly verify that these representations are irreducible. The common nomenclature is to call an irreducible representation labelled by $j$ a "spin-$j$" irreducible representation of $\mathfrak{so}(3) \cong \mathfrak{su}(2)$.

For $j=1$, we obtain the $3d$ representation space $V_{j=1}$, which is the Lie algebra representation induced by the defining representation of $SO(3)$. The integer spin $j>1$ representations can be shown to correspond to (irreps inside the) tensor
products $V_1 \otimes \dots \otimes V_1 = {V_1}^{\otimes j}$, which therefore are naturally $SO(3)$ representations. But the half-integer spin representations are not $SO(3)$ representations, as we'll shortly see.


%%%%%%%%%%%%%%%%%%%%%%%%%%% HALF INTEGER SPIN IRREDUCIBLE REPRESENTATIONS %%%%%%%%%%%%%%%%%%%%%%%%%%%
\section{Half-integer spin irreducible representation.}
Let's fix a particular $j = \tilde{j} \in \frac{\N_0}{2}$ and name the basis element of the $2 \tilde{j} + 1$ dimensional representation space $V$, $\{ \ket{\tilde{j}, m} \}$, where $m \in \{ -\tilde{j}, \dots \tilde{j} \}$. Then, taking an element of the algebra, $X \in \mathfrak{so}(3) \cong \mathfrak{su}(2)$, the corresponding representation matrix on $V$ is given by
\begin{equation}
    M = \rho_j (X), \quad M_{ab} = \bra{\tilde{j},a} M \ket{\tilde{j},b}, \quad M \ket{\tilde{j}, m} = \sum_k \ket{\tilde{j}, k} M_{km} .
\end{equation}

Then, by means of eq.\dots, via the exponential map we get a representation of the covering group.

\color{red}
\dots
\dots
\dots

\begin{equation}
    \rho_j[R(\pi)R(\pi)]_{ab} = (-1)^{2\tilde{j}} \delta_{ab}.
\end{equation}
This shows that not all irrep of the algebra extends to irreps of $SO(3)$. They are, though, irreps of the universal covering group, which is $SU(2)$. We're interested in this last group in quantum mechanics, since \dots
\color{blue}

%%%%%%%%%%%%%%%%%%%%%%%%%%%   relation and proof double cover%%%%%%%%%%%%%%%%%%%%%%%%%%%
\section{Relation between \texorpdfstring{$SO(3)$}{SO(3)} and \texorpdfstring{$SU(2)$}{SU(2)} and Proof of Double Coverness}
\dots
\dots
\dots
\color{black}

\chapter{Lorentz Group and its Representations}
To study the properties of the Lorentz and Poincaré groups, we start from their defining representations, keeping in mind what properties are representation-independent. In particular, we'll work in a 4-dimensional Minkowski spacetime with metric tensor 
\begin{equation}
    \eta = (\eta_{\mu\nu}) = \textup{diag}(+1,-1,-1,-1),
\end{equation}
and scalar product defined by
\begin{equation}
    x \cdot y \coloneq x^T \eta y = x^0y^0-\vec{x}\cdot\vec{y} = \eta_{\mu\nu} x^\mu y^\nu = x_\mu y^\mu .
\end{equation}

\color{red}L'ho scritto a caso come segnaposto, va scritta questa piccola introduzione. \color{blue}
We'll start from Lorentz group, in particular from its defining representation. We'll work out the algebra in the fundamental representation, taking infinitesimal transformation and computing the commutation relations. Then, we'll study the finite-dimensional representations, and see how to label them via SU(2) indices. We'll introduce spinor representation and explain why are they important in quantum mechanics. Finally, we'll introduce field representations, and explain how to construct a lorentz invariant action, and that a representation of the lorentz group is given by the nother charges under the symmetry. This will allow us to move forward and study the infinite dimensional representations of the Poincare group.
\color{black}


%%%%%%%%%%%%%%% LORENTZ GROUP %%%%%%%%%%%%%%%%%
\section{Lorentz Group}\label{sec:lorentz-group}
Lorentz transformations are those transformations $x \to x' = \Lambda x$ which leave the scalar product invariant, i.e.,
\begin{equation}
    (\Lambda x) \cdot (\Lambda y) = x \cdot y \implies x^T \Lambda^T \eta \Lambda y = x^T \eta y \implies \Lambda^T \eta \Lambda = \eta .
\end{equation}
Written in components, this condition becomes
\begin{equation}\label{eq:lorentz-transf-def-components}
    \eta_{\mu\nu} = \eta_{\alpha\beta} \tensor{\Lambda}{^\alpha_\mu} \tensor{\Lambda}{^\beta_\nu}.
\end{equation}

Since $\eta_{\mu\nu}$ is symmetric, this gives $10$ constraints. Further, since the Lorentz transformation is a $4 \times 4$ matrix, it depends on $16-10 = 6$ independent parameters, which will be interpreted later as three parameters for the boosts and three for rotations. 

For an infinitesimal transformation
\begin{equation}
    \tensor{\Lambda}{^\mu_\nu} \simeq \delta^\mu_\nu + \tensor{\omega}{^\mu_\nu} ,
\end{equation}
and using eq.~\eqref{eq:lorentz-transf-def-components}, we find
\begin{equation}\label{eq:parameters-lorentz}
    \omega_{\mu\nu} = -\omega_{\nu\mu} .
\end{equation}
\begin{proof}
    \begin{equation}
    \begin{split}
        \eta_{\mu\nu} &= \eta_{\alpha\beta} \tensor{\Lambda}{^\alpha_\mu} \tensor{\Lambda}{^\beta_\nu}
        \simeq \eta_{\alpha\beta} \left( \delta^\alpha_\mu + \tensor{\omega}{^\alpha_\mu} \right) \left(  \delta^\beta_\nu + \tensor{\omega}{^\beta_\nu} \right) 
        \\ &= \eta_{\alpha\beta} \delta^\alpha_\mu \delta^\beta_\nu + \eta_{\mu\beta} \tensor{\omega}{^\beta_\nu} + \eta_{\alpha\nu} \tensor{\omega}{^\alpha_\mu} + O(\omega^2)
        = \eta_{\mu\nu} + \omega_{\mu\nu} + \omega_{\nu\mu} + O(\omega^2) . \qedhere
    \end{split}
    \end{equation}
\end{proof}

The transformations of a space with coordinates $\{y_1, \dots y_n, x_1, \dots, x_m\}$ which leave the quadratic form $( {y_1}^2 + \dots + {y_n}^2 ) - ( {x_1}^2 + \dots + {x_m}^2 )$ invariant define the orthogonal group $O(n,m)$. Thus, the \emph{Lorentz group} is $O(1,3)$.

The group axioms~\ref{def:group-axioms} are satisfied, and in particular there exists a unit element $\1$ and each $\Lambda$ has an inverse since its determinant is different from zero. Further, using that the determinant of a product is the product of the determinants, and that the transpose of a matrix has the same determinant as the matrix, one can verify
\begin{subequations}
\begin{gather}
    \Lambda^T \eta \Lambda = \eta \implies (\det \Lambda)^2 = 1 \implies \det \Lambda = \pm 1 , \\
    \eta_{\mu\nu} \tensor{\Lambda}{^\mu_0} \tensor{\Lambda}{^\nu_0} = (\tensor{\Lambda}{^0_0})^2 - \sum_k {(\tensor{\Lambda}{^k_0})}^2 = 1 \implies (\tensor{\Lambda}{^0_0})^2 \geq 1 .
\end{gather}
\end{subequations}

Depending on the signs of $\det \Lambda$ and $\tensor{\Lambda}{^0_0}$, the Lorentz group has four disconnected components. The subgroup with $\det \Lambda = 1$ and $\tensor{\Lambda}{^0_0} \geq 1$ is called \emph{proper orthochronous} Lorentz group, $SO(1,3)^+$. The other components can be constructed from a given $\Lambda \in SO(1,3)^+$ combined with space and/or time reflection.


%%%%%%%%%%%%%%% LORENTZ ALGEBRA %%%%%%%%%%%%%%%%%
\section{Lorentz Algebra}\label{sec:lorentz-algebra}
Let's now detach for a moment from any representation, considering the abstract group $SO(1,3)^+$ to study its algebra.

We've seen that the Lorentz group is characterised by six independent parameters, which can be collected into the antisymmetric matrix $\omega_{\mu\nu}$ (see eq.~\eqref{eq:parameters-lorentz}). It is then convenient to label the generators as $M^{\mu\nu} = -M^{\nu\mu}$, where each pair $(\mu,\nu)$ identifies a particular generator. Then, using the exponential map\footnote{As discussed in sec.~\ref{sec:lie-groups-algebras} after eq.~\eqref{eq:exp-map}, recall we assume the exponential map is surjective.\color{red}true?\color{black}}, any element $\Lambda \in SO(1,3)^+$ can be written as
\begin{equation}\label{eq:abstract-lorentz-group-element}
   \Lambda = e^{-\frac{i}{2} \omega_{\mu\nu} M^{\mu\nu}},
\end{equation}
(\color{red} is it a problem if we have minus sign here and before plus sign?\color{black}) with conventional choice of constants. Then, given a finite dimensional representation $(\rho, V)$ of dimension $n$, the group element~\eqref{eq:abstract-lorentz-group-element} is represented by the $n \times n$ matrix
\begin{equation}
    \Lambda_\rho = e^{-\frac{i}{2} \omega_{\mu\nu} M^{\mu\nu}_\rho},
\end{equation}
which acts on $V$, where $M^{\mu\nu}_\rho$ are the Lorentz group generators in the representation $\rho$. Further, the elements of $V$ transform, under a Lorentz transformation, as
\begin{equation}
    \phi^i \to \tensor{\left[ e^{-\frac{i}{2} \omega_{\mu\nu} M^{\mu\nu}_\rho} \right]}{^i_j}  \, \phi^j .
\end{equation}

For an infinitesimal Lorentz transformation, with infinitesimal parameters $\omega_{\mu\nu}$, the variation of $\phi^i$ is
\begin{equation}\label{eq:action-lorentz-transf}
    \delta \phi^i = -\frac{i}{2} \omega_{\mu\nu} \tensor{\left(  M^{\mu\nu}_\rho \right)}{^i_j} \, \phi^j, 
\end{equation}
where in $\tensor{\left(  M^{\mu\nu}_\rho \right)}{^i_j}$, the indices $\mu,\nu$ identify the generator\footnote{Beware the index $\rho$! It isn't a Lorentz index, but it stands for “representation”.}, while the indices $i,j$ are the matrix indices of the representation which has been considered. We can then classify all physical quantities according to their transformation properties under the Lorentz group.

For clarity, we anticipate the Lie algebra generators' commutation relations, which are representation independent; we'll further compute them explicitly for the four-vector representation. The Lie algebra $\mathfrak{so}(1,3)$ is characterised by
\begin{equation}\label{eq:lorentz-algebra-relations}
    \comm{M^{\mu\nu}}{M^{\rho\sigma}} = i \left( \eta^{\nu\rho} M^{\mu\sigma} - \eta^{\mu\rho} M^{\nu\sigma} - \eta^{\sigma\mu} M^{\rho\nu} + \eta^{\sigma\nu} M^{\rho\mu} \right).
\end{equation}

It is convenient to rearrange the generators into two spatial vectors\footnote{To see how to invert expressions with Levi-Civita symbol, look at appendix~\ref{app:levi-civita}.},
\begin{equation}\label{eq:redef-lorentz-gen}
    J^i = \frac{1}{2} \epsilon^{ijk} M^{jk} \iff M^{ij} = \epsilon^{ijk} J^k, \quad K^i = M^{i0} .
\end{equation}

In terms of $\vec{J}$ and $\vec{K}$, the Lie algebra of the Lorentz group reads (\color{red}To be computed!\color{black})
\begin{subequations}
\begin{align}
    \comm{J^i}{J^j} &= i \epsilon^{ijk} J^k \label{eq:lorentz-algebra-rotation} \\ 
    \comm{J^i}{K^j} &= i \epsilon^{ijk} K^k \label{eq:lorentz-algebra-boost-vector} \\ 
    \comm{K^i}{K^j} &= -i \epsilon^{ijk} J^k .
\end{align}
\end{subequations}

Equation~\eqref{eq:lorentz-algebra-rotation} is the Lie algebra of $SU(2)$ and this shows that $\vec{J}$ can be interpreted as the angular momentum. Instead, eq.~\eqref{eq:lorentz-algebra-boost-vector} expresses the fact that $\vec{K}$ is a spatial vector, since it transforms accordingly under a rotation. 

Defining, further
\begin{equation}\label{eq:lorentz-redef-param}
    \theta^i = \frac{1}{2} \epsilon^{ijk} \omega^{jk} \iff \omega^{ij} = \epsilon^{ijk} \theta^k, \quad \eta^i = \omega^{i0} ,
\end{equation}
a generic Lorentz transformation can be written as
\begin{equation}
    \Lambda = e^{-i \vec{\theta} \cdot \vec{J} + i \vec{\eta} \cdot \vec{K}} .
\end{equation}
\begin{proof}
    \begin{equation*}
    \begin{split}
        \frac{1}{2} \omega_{\mu\nu} M^{\mu\nu} &= \frac{1}{2} \Bigl( \sum_{i=1}^3 \left( \omega_{i0} M^{i0} + \omega_{0i} M^{0i} \right) + \omega_{12} M^{12} + \omega_{21} M^{21} \\
        &\qquad \quad \; \; + \omega_{13} M^{13} + \omega_{31} M^{31} + \omega_{23} M^{23} + \omega_{32} M^{32} \Bigr) \\
        &= \omega^{12} M^{12} + \omega^{13} M^{13} + \omega^{23} M^{23} - \sum_{i=1}^3 \omega^{i0} M^{i0} \\
        &= \vec{\vec{\theta} \cdot \vec{J} - \vec{\eta} \cdot \vec{K}},
    \end{split}
    \end{equation*}
    where we've used at the same time the antisymmetries $\omega_{\mu\nu} = - \omega_{\nu\mu}$ and $M^{\mu\nu}=- M^{\nu\mu}$. Moreover, according to~\eqref{eq:lorentz-redef-param}, $\omega_{i0} = -\omega^{i0}=-\eta^i$, $\omega_{12} = \omega^{12} = \theta^3$, $\omega_{13} = \omega^{13} = - \theta^2$ and $\omega_{23} = \omega^{23} = \theta^1$, and according to~\eqref{eq:redef-lorentz-gen}, $M_{i0} = -M^{i0}=- K^i$, $M_{12} = M^{12} = J^3$, $M_{13} = M^{13} = - J^2$ and $M_{23} = M^{23} = J^1$.

    Substituting into eq.~\eqref{eq:abstract-lorentz-group-element} leads to the result.
\end{proof}

Let's now classify the representations of the Lorentz group.


%%%%%%%%%%%%%%%  TRIVIAL REPRESENTATION %%%%%%%%%%%%%%%%%
\subsection{Trivial Representation}
This is the one acting on a \emph{scalar} $\phi$, i.e., a quantity which is invariant under a Lorentz transformation, like the rest mass of a particle. By means of~\eqref{eq:action-lorentz-transf}, we have
\begin{equation}\label{eq:lorentz-transf-scalar}
    \delta \phi = 0,
\end{equation}
so that the generators, which are $1 \times 1$ matrices, identically vanish,
\begin{equation}
    \M^{\mu\nu} \equiv 0.
\end{equation}

The representation is called \emph{trivial}, since the algebra commutation relations~\eqref{eq:structure-contants} are trivially satisfied.

%%%%%%%%%%%%%%%  VECTOR REPRESENTATION %%%%%%%%%%%%%%%%%
\subsection{Vector Representation}
This is the \emph{defining representation} and acts on a \emph{four-vector} $V^\mu$, which has transformation law
\begin{equation}\label{eq:lorentz-transf-four-vector}
    V^\mu \to \tensor{\Lambda}{^\mu_\nu} V^\nu,
\end{equation}
where $\Lambda$ satisfy the condition~\eqref{eq:lorentz-transf-def-components}, i.e.,
\begin{equation*}
    \eta_{\mu\nu} = \eta_{\alpha\beta} \tensor{\Lambda}{^\alpha_\mu} \tensor{\Lambda}{^\beta_\nu}.
\end{equation*}

One could generically call the previous vector a \emph{contravariant} one and define a \emph{covariant} four-vector $V_\mu$ as one which transforms as $V_\mu \to \tensor{\Lambda}{_\mu^\nu} V_\nu$, with $\tensor{\Lambda}{_\mu^\nu} = \eta_{\mu\alpha} \eta^{\nu\beta} \tensor{\Lambda}{^\alpha_\beta} = \tensor{(\Lambda^{-1})}{^\mu_\nu}$. However, this distinction is not useful, since the two are related via index raising/lowering by the metric tensor, $V_\mu = \eta_{\mu\nu} V^\nu$. An example of four-vector is given by the four-momentum.

To find the explicit representation acting on $V^\mu$, let's first notice that the indices $i,j$ of eq.~\eqref{eq:action-lorentz-transf} are now Lorentz indices themselves, so that the representation matrix would read $\tensor{(M^{\mu\nu})}{^\alpha_\beta}$. Here, we drop any index referring to the representation. If not explicitly indicated, the context should make evident if we're talking about the abstract generator or a representation. 

Then, a four-vector transforms as
\begin{equation}\label{eq:variation-lorentz-transf-four-vector}
   \delta V^\alpha = - \frac{i}{2} \omega_{\mu\nu} \tensor{(M^{\mu\nu})}{^\alpha_\beta} V^\beta ,
\end{equation}
where
\begin{equation}\label{eq:lorentz-transf-matrix-vector-rep}
    \tensor{(M^{\mu\nu})}{^\alpha_\beta} = i \left( \eta^{\mu\alpha} \delta^\nu_\beta - \eta^{\nu\alpha} \delta^\mu_\beta \right) .
\end{equation}

\begin{proof}
    Considering the infinitesimal version of eq.~\eqref{eq:lorentz-transf-four-vector}, we get
    \begin{equation*}
        V^\alpha \to \tensor{\Lambda}{^\alpha_\beta} V^\beta \simeq (\delta^\alpha_\beta + \tensor{\omega}{^\alpha_\beta}) V^\beta,
    \end{equation*}
    which gives an infinitesimal variation
    \begin{equation*}
    \begin{split}
        \delta V^\alpha &= \tensor{\omega}{^\alpha_\beta} V^\beta = \frac{1}{2} \omega_{\mu\beta} \left( \eta^{\mu\alpha} V^\beta - \eta^{\beta\alpha} V^\mu \right) = \frac{1}{2} \omega_{\mu\nu} \delta^\nu_\beta \left( \eta^{\mu\alpha} V^\beta - \eta^{\beta\alpha} V^\mu \right) \\
        &= \frac{1}{2} \omega_{\mu\nu} \left( \delta^\nu_\beta \eta^{\mu\alpha} - \delta^\mu_\beta \eta^{\nu\alpha} \right) V^\beta \overset{!}{=} - \frac{i}{2} \omega_{\mu\nu} \tensor{(M^{\mu\nu})}{^\alpha_\beta} V^\beta ,
    \end{split}
    \end{equation*}
    where the second addend was added to ensure the piece within parenthesis is antisymmetric with respect to $\mu$--$\beta$, as should be by eq.~\eqref{eq:parameters-lorentz}. Then, in the second to last passage we relabelled a couple of dummy indices, to compare the expression with eq.~\eqref{eq:variation-lorentz-transf-four-vector}. We obtain the matrix~\eqref{eq:lorentz-transf-matrix-vector-rep}.
\end{proof}

It's now easy to compute the commutators of~\eqref{eq:lorentz-transf-matrix-vector-rep} to find Lorentz algebra~\eqref{eq:lorentz-algebra-relations}. As already stated, the structure constants don't depend on the representation, so what we find here are the abstract commutation relations.
\begin{proof}
    Using eq.~\eqref{eq:lorentz-transf-matrix-vector-rep}, we compute
    \begin{equation*}
    \begin{split}
        \tensor{\comm{M^{\mu\nu}}{M^{\rho\sigma}}}{^\alpha_\beta} &= \tensor{(M^{\mu\nu})}{^\alpha_\gamma} \tensor{(M^{\rho\sigma})}{^\gamma_\beta} - \tensor{(M^{\rho\sigma})}{^\alpha_\gamma} \tensor{(M^{\mu\nu})}{^\gamma_\beta}  \\
        &= - (\eta^{\mu\alpha} \delta^\nu_\gamma - \eta^{\nu\alpha} \delta^\mu_\gamma) (\eta^{\rho\gamma} \delta^\sigma_\beta - \eta^{\sigma\gamma} \delta^\rho_\beta) + (\rho \leftrightarrow \mu, \; \sigma \leftrightarrow \nu) \\
        & = -\eta^{\mu\alpha} \eta^{\rho\nu} \delta^\sigma_\beta + \eta^{\mu\alpha} \eta^{\sigma\nu} \delta^\rho_\beta + \eta^{\nu\alpha} \eta^{\rho\mu} \delta^\sigma_\beta - \eta^{\nu\alpha} \eta^{\sigma\mu} \delta^\rho_\beta \\
        &\quad + \eta^{\rho\alpha} \eta^{\mu\sigma} \delta^\nu_\beta - \eta^{\rho\alpha} \eta^{\nu\sigma} \delta^\mu_\beta - \eta^{\sigma\alpha} \eta^{\mu\rho} \delta^\nu_\beta + \eta^{\sigma\alpha} \eta^{\nu\rho} \delta^\mu_\beta \\
        &= - \eta^{\nu\rho} ( \eta^{\mu\alpha} \delta^\sigma_\beta - \eta^{\sigma\alpha} \delta^\mu_\beta ) + \eta^{\mu\rho} ( \eta^{\nu\alpha} \delta^\sigma_\beta - \eta^{\sigma\alpha} \delta^\nu_\beta ) \\
        &\quad + \eta^{\sigma\mu} ( \eta^{\rho\alpha} \delta^\nu_\beta - \eta^{\nu\alpha} \delta^\rho_\beta) - \eta^{\sigma\nu} (\eta^{\rho\alpha} \delta^\mu_\beta - \eta^{\mu\alpha} \delta^\rho_\beta) \\
        &= i  \eta^{\nu\rho} \tensor{(M^{\mu\sigma})}{^\alpha_\beta}  - i \eta^{\mu\rho} \tensor{(M^{\nu\sigma})}{^\alpha_\beta} - i \eta^{\sigma\mu}  \tensor{(M^{\rho\nu})}{^\alpha_\beta} +  i \eta^{\sigma\nu} \tensor{(M^{\rho\mu})}{^\alpha_\beta} . \qedhere
    \end{split}
    \end{equation*}
\end{proof}

As previously noticed, the Lie algebra $\mathfrak{so}(1,3)$ contains the Lie algebra $\mathfrak{so}(3)$. This is to be expected. Indeed, for Lorentz transformation of the form $\Lambda = \left(\begin{smallmatrix} 1 & 0 \\ 0 & R \end{smallmatrix} \right) \in SO(1,3)^+$, with $R$ a $3 \times 3$ matrix, the defining property~\eqref{eq:lorentz-transf-def-components} reduced to $R^TR = \1$, which is the defining property of $SO(3)$. Then, $SO(3)$ is a subgroup of $SO(1,3)^+$ and consequently the Lie algebras must reflect this property. From the explicit representation of the generators~\eqref{eq:lorentz-transf-matrix-vector-rep}, we can see that $M_{ij}$ are block diagonal matrices of the form $M \sim \left( \begin{smallmatrix} 1 & 0 \\ 0 & N \end{smallmatrix} \right) $, with $N$ a $3 \times 3$ matrix, so the exponentiation conserves this form and produces the expected rotation matrices inside $SO(1,3)^+$.


%%%%%%%%%%%%%%%  TENSOR REPRESENTATION %%%%%%%%%%%%%%%%%
\subsection{Tensor Representation}
Let's briefly recall what a tensor is. If $V$ is an $n$ dimensional real vector space and $V^*$ its dual space, the space of \emph{type $(p,q)$-tensors} is defined as
\begin{equation}
    T^{(p,q)}(V) = 
    \underbrace{V \otimes \dots \otimes V}_\text{$p$ times}
    \otimes
    \underbrace{V^* \otimes \dots \otimes V^*}_\text{$q$ times} .
\end{equation}

As we know from multilinear algebra, if $\{e_i\}$ is any basis for $V$ and $\{\epsilon^j\}$ the corresponding dual basis for $V^*$, a basis for $T^{(p,q)}(V)$ is given by
\begin{equation}
    \{ e_{i_1} \otimes \dots \otimes e_{i_p} \otimes \epsilon^{j_1} \otimes \dots \otimes \epsilon^{j_q} : 1 \leq i_1, \dots, i_p, j_1, \dots, j_q \leq n \},
\end{equation}
and the tensor can be written as
\begin{equation}
    T = \tensor{T}{^{i_1}^\dots^{i_p}_{j_1}_\dots_{j_q}} e_{i_1} \otimes \dots \otimes e_{i_p} \otimes \epsilon^{j_1} \otimes \dots \otimes \epsilon^{j_q}.
\end{equation}

We can uniquely identify a tensor via its coordinates, so we'll forget about the basis elements from now on. Further, we'll consider only \emph{contravariant} tensors, i.e., tensors with upper indices, since we can easily lower indices via the metric tensor and transform a \emph{covariant} index with the inverse of the metric, as already discussed for vectors, which are rank $1$ tensors.

Then, a Lorentz tensor of \emph{rank} $n$ is defined by the transformation law
\begin{equation}\label{eq:lorentz-transf-tensor}
    T^{\mu\nu\dots\tau} \to {T'}^{\mu\nu\dots\tau} = \underbrace{\tensor{\Lambda}{^\mu_\alpha} \tensor{\Lambda}{^\nu_\beta} \dots \tensor{\Lambda}{^\tau_\lambda}}_\text{$n$ times} T^{\alpha\beta\dots\lambda},
\end{equation}
so we can always construct the representation matrices $\tensor{\Lambda}{^\mu_\alpha} \tensor{\Lambda}{^\nu_\beta} \dots$ of the Lorentz transformation as the outer product $\vec{4} \otimes \vec{4} \otimes \dots$ of the $4$ dimensional defining representation $\Lambda$. However, these representations are not irreducible. To see why, let's study the simplest case of a $4 \times 4$ tensor $T^{\mu\nu}$, which has $16$ components.

Its trace, its antisymmetric component, and its symmetric and traceless part,
\begin{equation}
   S = \tensor{T}{^\alpha_\alpha} , \quad A^{\mu\nu} = \frac{1}{2} (T^{\mu\nu} - T^{\nu\mu}), \quad S^{\mu\nu} = \frac{1}{2} (T^{\mu\nu} + T^{\nu\mu}) - \frac{1}{4} \eta^{\mu\nu} S, 
\end{equation}
don't mix under Lorentz transformations, since a (anti-) symmetric tensor is still (anti-) symmetric after the transformation, and the trace is a Lorentz scalar.
\begin{proof}
    For the trace, using the property~\eqref{eq:lorentz-transf-def-components}, we can see
    \begin{equation*}
       S \coloneq \eta_{\mu\nu} S^{\mu\nu} \to \eta_{\mu\nu} \tensor{\Lambda}{^\mu_\alpha} \tensor{\Lambda}{^\nu_\beta} S^{\alpha\beta} = \eta_{\alpha\beta} S^{\alpha\beta} = S .
    \end{equation*}
    A faster way to reach this conclusion is to note that, due to the (absence of) index structure of $S$, it is a Lorentz scalar, and so transforms trivially by~\eqref{eq:lorentz-transf-scalar}.

    Concerning the antisymmetric part, a tensor has this property if $A^{\mu\nu} = - A^{\mu\nu}$. Let's verify this is preserved under a Lorentz transformation. A generic two tensor transforms according to eq.~\eqref{eq:lorentz-transf-tensor}, so
    \begin{equation*}
       A^{\mu\nu} \to A'^{\mu\nu} = \tensor{\Lambda}{^\mu_\alpha} \tensor{\Lambda}{^\nu_\beta} A^{\alpha\beta} ,
    \end{equation*}
    and using the antisymmetry of $A^{\mu\nu}$
    \begin{equation*}
        A'^{\nu\mu} = \tensor{\Lambda}{^\nu_\alpha} \tensor{\Lambda}{^\mu_\beta} A^{\alpha\beta} = - \tensor{\Lambda}{^\nu_\alpha} \tensor{\Lambda}{^\mu_\beta} A^{\beta\alpha} = -A'^{\mu\nu} .
    \end{equation*}
    Finally, for the symmetric traceless part,
    \begin{equation*}
        S^{\mu\nu} \to S'^{\mu\nu} = \tensor{\Lambda}{^\mu_\alpha} \tensor{\Lambda}{^\nu_\beta} S^{\alpha\beta} ,
    \end{equation*}
    and using its symmetry
    \begin{equation*}
        S'^{\nu\mu} = \tensor{\Lambda}{^\nu_\alpha} \tensor{\Lambda}{^\mu_\beta} S^{\alpha\beta} = \tensor{\Lambda}{^\nu_\alpha} \tensor{\Lambda}{^\mu_\beta} S^{\beta\alpha} = S'^{\mu\nu} ,
    \end{equation*}
    and the facts that it is traceless and the trace it's a scalar,
    \begin{equation*}
        S' = S = 0. \qedhere
    \end{equation*}
\end{proof}

Then, recalling the definition~\eqref{def:reducible-rep}, the original tensor $T^{\mu\nu}$ lived in a reducible representation, and there are three irreducible subspaces, a one-dimensional subspace, spanned by the trace, a $6$-dimensional one, spanned by the antisymmetric tensors, and a $9$-dimensional one, spanned by traceless symmetric tensors. Using the convention of denoting an irreducible representation with its dimensionality, written in boldface, we have
\begin{equation}
    \vec{4} \otimes \vec{4} = \vec{1} \oplus \vec{6} \oplus \vec{9} .
\end{equation}

A priori, it's not necessary true that those representations are irreducible. One should prove it. Without going into the details, let's just cite that the most general irreducible tensor representations of the Lorentz group are found starting from a generic tensor with an arbitrary number of indices, removing first all traces, and then symmetrizing or antisymmetrising over all pairs of indices.


%%%%%%%%%%%%%%%  DECOMPOSITION %%%%%%%%%%%%%%%%%
\color{red}
\subsection{Decomposition of Lorentz tensors under SO(3)}
Even if the vector representation is an irreducible representation of the Lorentz group, from the point of view of $SO(3)$, which is a subgroup, it is reducible. This will be related to spin.




%%%%%%%%%%%%%%%  SPINORIAL REPRESENTATION %%%%%%%%%%%%%%%%%
\subsection{Spinorial Representation}
\dots
\color{black}

%%%%%%%%%%%%%%%  FIELD REPRESENTATION %%%%%%%%%%%%%%%%%
\subsection{Field Representation and Action}
Up to know we've seen representations acting on spacetime-independent quantities, which generically transform as
\begin{equation}
    \phi'_i  = D_{ij}(\Lambda) \phi_j.
\end{equation}

A field $\phi_i$ is a function of the coordinates with some definite transformation properties under the Poincaré group. When we consider their transformation we must consider that it acts on their argument too, $x' = \Lambda x$:
\begin{equation}
    \phi'_i (x) = D_{ij}(\Lambda) \phi_j(\Lambda^{-1}x) \iff \phi'_i(x') = D_{ij}(\Lambda) \phi_j(x) .
\end{equation}

We can define two types of infinitesimal variation the field
\begin{itemize}
    \item \emph{A change in perspective}:
    \begin{equation}
        \delta \phi_i = \phi'_i(x') - \phi_i(x) = \frac{i}{2} \epsilon_{\mu\nu} (\hat{S}^{\mu\nu})_{ij} \phi_j(x) ,
    \end{equation}
    with $(\hat{S}^{\mu\nu})_{ij}$ a finite-dimensional matrix representation of the generator $M^{\mu\nu}$. The subscript will stand for \emph{spin}, as will be clear in the following.
    \item \emph{Functional change}:
    \begin{equation}\label{eq:total-functional-variation}
        \delta_0 \phi_i = \phi'_i(x) - \phi(x) = \phi'_i (x' - \delta x) - \phi_i(x) = \delta \phi_i - \delta x_\mu \de^\mu \phi_i ,
    \end{equation}
    where we evaluate the field changes at the same position $x$.
\end{itemize}

An \emph{infinitesimal Lorentz transformation} can be written as $\delta x_\mu = \omega_{\mu\nu} x^\nu$, so
\begin{equation*}
    -\delta x_\mu \de^\mu \phi_i = - \omega_{\mu\nu} x^\nu \de^\mu \phi_i = \frac{i}{2} \omega_{\mu\nu} \left[ -i (x^\mu \de^\nu - x^\nu \de^\mu) \right] \phi_i .
\end{equation*} 
Defining
\begin{equation}
    {\hat{L}}^{\mu\nu} \coloneq -i (x^\mu \de^\nu - x^\nu \de^\mu),
\end{equation}
we get
\begin{equation}
    -\delta x_\mu \de^\mu \phi_i = \frac{i}{2} \omega_{\mu\nu} {\hat{L}}^{\mu\nu} \phi_i.
\end{equation}

To generalise this to a Poincaré transformation, let's first consider a pure translation, after which
\begin{equation}
    \phi'_i(x) = \phi_i (x-a) \iff \phi'_i(x') = \phi_i(x) ,
\end{equation}
and hence $\delta \phi_i = 0$ and $\delta_0 \phi_i = - \epsilon_\mu \de^\mu \phi_i = i \epsilon_\mu \hat{P}^\mu \phi_i$, with $\hat{P}^\mu 0 i \de^\mu$. The total change of the field is then
\begin{equation}
    \phi'_i(x) = \phi_i(x) + \left[ \frac{i}{2} \omega_{\mu\nu} ({\hat{S}}^{\mu\nu} + {\hat{L}}^{\mu\nu}) + i \epsilon_\mu \hat{P}^\mu \right]_{ij} \phi_j(x) .
\end{equation}

Here, $\hat{L}^{\mu\nu}$ and $\hat{P}^\mu$ are differential operators that satisfy Poincaré algebra relations when applied to $\phi_i(x)$. There are diagonal in $i,j$, whereas the spin matrix $\hat{S}^{\mu\nu}$ depends on the representation of the field. Moreover, we can write $\hat{J}^{\mu\nu} = \hat{S}^{\mu\nu} + \hat{L}^{\mu\nu}$ and try to find out the explicit form of those operators in the representation under study. To do so, we use eq.~\eqref{eq:poincare-boosts-rotations} for the abstract generators, and find out that we can divide the generators of the angular momentum and of the boosts in a spin and an orbital angular part too, in particular
\begin{equation}
    \vec{L} = \vec{x} \wedge \vec{P}, \quad \vec{K}_L = \vec{x} P^0 - x^0 \vec{P}, \quad P^\mu = i \de^\mu .
\end{equation}


\paragraph{Action}
One way to study and quantise a theory is starting from the classical action $S$. In order for it to be invariant, it must be a Poincaré scalar. Further, by Noether's theorem, associated to the Poincaré invariance there are some conserved charges. In particular, using eq.~\eqref{eq:total-functional-variation}, we may write
\begin{equation}
    \delta S = \int \udq x \, \delta_0 \L + \int \udq x \, \de_\mu \L \delta x^\mu
\end{equation}



To construct a Poincaré invariant theory, we start from a \emph{classical} action which has the same property. We can then apply Noether theorem to find out the conserved charges related to the Poincaré symmetry. 
\dots
\dots

It turns out that the currents associated to the Poincaré symmetry of a classical action are
\begin{subequations}
\begin{gather}
T^{\mu\alpha} = -i \frac{\de \L}{\de(\de_\mu \phi_i)} P^\alpha \phi_j - g^{\mu\alpha} \L , \\
m^{\mu,\alpha\beta} = -i \frac{\de \L}{\de (\de_\mu \phi_i)} M^{\alpha\beta}_{ij} \phi_j + (x^\alpha g^{\mu\beta} - x^\beta g^{\mu\alpha}) \L .
\end{gather}
\end{subequations}

The corresponding charges are
\begin{subequations}\label{eq:const-motion}
\begin{align}
    \hat{P}^\alpha &= \int \udt x \, T^{0\alpha} , \\
   \hat{M}^{\alpha\beta} &= \int \udt x \, m^{0 ,\alpha\beta} .
\end{align}
\end{subequations}

Those, after quantization, will form a representation of the Poincaré group that acts on the state space.

Further, in the quantum setting, Wigner's theorem states that continuous symmetries must be implemented by unitary operators on the state space. The Lorentz group is not compact because it contains boosts, hence all unitary representations must be infinite-dimensional. 

This is realized in the quantum field theory: the fields $\phi_i(x)$ become operators on the Fock space, and the constants of motion in eq.~\eqref{eq:const-motion} are hermitian operators that define a unitary representation of the Poincaré algebra on the state space:
\begin{equation}
    U(\Lambda,a) = e^{\frac{i}{2} \omega_{\mu\nu} \hat{M}^{\mu\nu} + i \epsilon_\mu \hat{P}^\mu} \simeq 1 + \frac{i}{2} \omega_{\mu\nu} \hat{M}^{\mu\nu} + i \epsilon_\mu \hat{P}^\mu .
\end{equation}

The irreducible state space those operators act on is the one-particle state space we constructed above, using Casimir operators and so on.

\color{red} How, from this, we get to $U(\Lambda,a) \phi_i(x) U(\Lambda,a)^{-1} = D(\Lambda)^{-1}_ij \phi_j (\Lambda x + a)$
\color{black}

\chapter{Poincaré Group and its Representations}
%%%%%%%%%%%%%%% POINCARE GROUP %%%%%%%%%%%%%%%%%
\section{Poincaré Group}
The Lorentz transformations defined in sec.~\ref{sec:lorentz-group} guarantee that the norm $x^2 \coloneq x_\mu x^\mu$ of a four-vector is invariant under a transformation. However, the principles of special relativity require that the line element $(\ud x)^2 = \eta_{\mu\nu} \ud x^\mu \ud x^\nu = c^2 (\ud t)^2 - (\ud \vec{x})^2$ is invariant. This guarantees that the speed of light is the same in each inertial frame. 

In order to satisfy the above condition, we can enlarge the Lorentz transformations, adding constant translations too. We call the resulting group \emph{Poincaré group} or \emph{inhomogeneous Lorentz group}. For simplicity, we only take the connected component of the Lorentz group, which allows us to consider the connected component of the Poincaré group, given by
\begin{equation}
    ISO(1,3)^+ = \{ (a,\Lambda) : a \in \R^4, \Lambda \in SO(1,3)^+ \},
\end{equation}
with group multiplication defined by
\begin{equation}
    (\Lambda_1, a_1) (\Lambda_2, a_2) = (\Lambda_1 \Lambda_2, a_1 + \Lambda_1 a_2),
\end{equation}
as we'll show soon. In the \emph{defining representation}, the group elements are matrices acting on the Minkowski vector space $\R^{1,3}$. Not to be too pedant with the notation, we continue to denote group elements in this particular representation with $(\Lambda,a)$.

\begin{remark}
    Pay attention to the fact that $\R^{1,3}$ is \emph{not} the flat Minkowski spacetime $\M$. Rather, it can be thought as its tangent space at a point. Then, it's a case that we are able to identify this tangent space with the spacetime itself, if the latter is flat.
\end{remark}

Then, the action of $(\Lambda,a)$ on $R^{1,3}$ is given by
\begin{equation}\label{eq:poincare-definition}
    x^\mu \to {x'}^\mu = (\Lambda, a) x = \tensor{\Lambda}{^\mu_\nu} x^\nu + a^\mu ,
\end{equation}
where $\tensor{\Lambda}{^\mu_\nu}$ is a Lorentz transformation in the defining representation, as defined in sec.~\ref{sec:lorentz-group}, and $a^\mu \in \R^(1,3)$ is an arbitrary constant four-vector. Basically, we're adding four new parameters, having in total a $10$ parameter group, containing translations, rotations, and boosts.

We can check again the group axioms~\ref{def:group-axioms}, considering that by means of eq.~\eqref{eq:representation-property}, a representation inherits the composition rule from the group. We obtain
\begin{subequations}
\begin{align}
    (\Lambda_1, a_1) (\Lambda_2, a_2) &= (\Lambda_1 \Lambda_2, a_1 + \Lambda_1 a_2), \label{eq:poincare-group-multiplication} \\
    \left[(\Lambda_1,a_1)(\Lambda_2,a_2)\right] (\Lambda_3,a_3) &= (\Lambda_1, a_1) \left[ (\Lambda_2,a_2)(\Lambda_3,a_3) \right], \label{eq:poincare-associative} \\
    \1 &= (\1, 0) ,\label{eq:poincare-unit-element} \\
    (\Lambda, a)^{-1} &= (\Lambda^{-1}, -\Lambda^{-1}a). \label{eq:poincare-inverse-element}
\end{align}
\end{subequations}
\begin{proof}
    After two consecutive Poincaré transformations, using eq.~\eqref{eq:poincare-definition},
    \begin{equation*}
        x \to x' = (\Lambda_2,a_2)x = \Lambda_2 x + a_2 \to (\Lambda_1,a_1) x' = \Lambda_1 (\Lambda_2 x + a_2) + a_1 = (\Lambda_1 \Lambda_2) x + (\Lambda_1 a_2 + a_1) ,
    \end{equation*}
    showing that the composition of two Poincaré transformations gives another transformation, as appointed by eq.~\eqref{eq:poincare-group-multiplication}.

    Further, taking three transformations, $(\Lambda_1,a_1)$, $(\Lambda_2,a_2)$, $(\Lambda_3,a_3)$, and composing them using eq.~\eqref{eq:poincare-group-multiplication}, we get
    \begin{gather*}
        \left[ (\Lambda_1,a_1)(\Lambda_2,a_2) \right] (\Lambda_3,a_3) = (\Lambda_1\Lambda_2, a_1 + \Lambda_1 a_2) (\Lambda_3,a_3) = (\Lambda_1\Lambda_2\Lambda_3, a_1 + \Lambda_1 a_2 + (\Lambda_1 \Lambda_2)a_3) , \\
        (\Lambda_1,a_1)\left[(\Lambda_2,a_2)T(\Lambda_3,a_3)\right] = (\Lambda_1,a_1) (\Lambda_2\Lambda_3, a_2 + \Lambda_2 a_3) = (\Lambda_1\Lambda_2\Lambda_3, a_1+\Lambda_1(a_2 + \Lambda_2 a_3)) .
    \end{gather*}
 
    Since they lead to the same result, eq.~\eqref{eq:poincare-associative} is proved.

    The unit element is trivial, since by means of eq.~\eqref{eq:poincare-definition}, taking $(\1,0)$,
    \begin{equation*}
        x \to (\1,0) x = x.
    \end{equation*}

    Finally, to show the inverse element is given by eq.~\eqref{eq:poincare-inverse-element}, let's use eq.~\eqref{eq:poincare-group-multiplication} to compute
    \begin{equation*}
       (\Lambda,a) (\Lambda^{-1}, -\Lambda^{-1}a) = (\Lambda^{-1}\Lambda, a - \Lambda(\Lambda^{-1}a)) = (\1,0),
    \end{equation*}
    which is the group unit element, as expected.
\end{proof}

Basically, while $SO(1,3)$ relates orthonormal frames with coincident origins, elements of $ISO(1,3)$ also allow for the shift of origin.



%%%%%%%%%%%%%%% POINCARE ALGEBRA %%%%%%%%%%%%%%%%%
\section{Poincaré Algebra}
To work out the \emph{Poincaré algebra}, let's follow the approach of sec.~\ref{sec:lorentz-algebra}. Then, we consider the abstract group $ISO(1,3)^+$, which is now characterised by $10$ parameters. Without translations, we must recover Lorentz group. Therefore, we continue to denote the six generators referring to Lorentz transformations by $M^{\mu\nu} = -M^{\nu\mu}$, and their parameters by $\omega_{\mu\nu}=-\omega_{\nu\mu}$. Further, we call the four generators related to translations $P^\mu$, and their parameters $\epsilon_\mu$.

Here, $(\Lambda,a) \in ISO(1,3)^+$ will denote an element of the abstract Poincaré group, while $\Lambda \in SO(1,3)^+$ is a proper orthochronous Lorentz transformation and $a \in \R^4$ a translation.

Using the exponential map\footnote{\color{red}problem here, is it surjective?\color{black}}, any element $(\Lambda,a) \in ISO(1,3)^+$ can be written as
\begin{equation}
    (\Lambda,a) = e^{-\frac{i}{2} \omega_{\mu\nu} M^{\mu\nu} - i \epsilon_\sigma P^\sigma},
\end{equation}
with a conventional choice of constants.

Let's denote a generic representation of the Poincaré group with $U(\Lambda, a)$. The following properties are not necessarily referred to unitary representations, but at the end we'll be interested on them; from this the choice of notation. In order for it to satisfy the definition~\ref{def:representation}, it must inherit the transformation properties of the group, in particular eq.~\eqref{eq:poincare-group-multiplication} and eq.~\eqref{eq:poincare-inverse-element},
\begin{subequations}
\begin{align}
    U(\Lambda, a) U(\Lambda',a') &= U(\Lambda\Lambda', a + \Lambda a') , \label{eq:representation-poincare-group-multiplication} \\
    U^{-1} (\Lambda, a) &= U(\Lambda^{-1},-\Lambda^{-1}a) . \label{eq:representation-poincare-inverse-element}
\end{align}
\end{subequations}

Then, for infinitesimal transformations\footnote{See chapter $2$ of Weinberg~\cite{weinberg} to convince yourself this is possible. Here, we're assuming the exponential map is surjective. As discussed in sec.~\ref{sec:lie-groups-algebras} after eq.~\eqref{eq:exp-map}, it is a non-trivial fact to prove the surjectivity. However, we're not interested in such details, so we'll just ignore it. \color{red} Exponential vs product of exponentials, still to be solved \color{black}},
\begin{equation}\label{eq:infinitesimal-poicare-transformation}
    U(\Lambda, a) = e^{-\frac{i}{2} \omega_{\mu\nu} M^{\mu\nu} - i \epsilon_\sigma P^\sigma} \simeq \1 - \frac{i}{2} \omega_{\mu\nu} M^{\mu\nu} - i \epsilon_\sigma P^\sigma,
\end{equation}
where the explicit forms of $U(\Lambda,a)$ and the generators $M^{\mu\nu}$ and $P^\mu$ depend on the representation. Here, we're denoting the abstract generators and their representation with the same symbols, $M^{\mu\nu}$ and $P^\mu$, without further indices or specifications. From the context it should be straightforward to identify what we're talking about. At the end of the day, we'll be interested in representations, and further recall that the commutation relations defining the Lie algebra are representation independent.

The \emph{Poincaré algebra} is defined by
\begin{subequations}
    \begin{align}
        \comm{M^{\mu\nu}}{M^{\rho\sigma}} &= i ( \eta^{\nu\rho}M^{\mu\sigma} - \eta^{\mu\rho}M^{\nu\sigma} - \eta^{\mu\sigma}M^{\rho\nu} + \eta^{\sigma\nu}M^{\rho\mu}) \label{eq:comm-poincare-lorentz}\\
        i \comm{P^\mu}{M^{\rho\sigma}} &= \eta^{\mu\rho} P^\sigma - \eta^{\mu\sigma} P^\rho , \label{eq:comm-poincare-lorentzmomentum}\\
        \comm{P^\mu}{P^\nu} &= 0 \label{eq:comm-poincare-momentum},
\end{align}
\end{subequations}
%%%%%%% begin proof
\begin{proof}
    Combining eq.~\eqref{eq:representation-poincare-group-multiplication} and eq.~\eqref{eq:representation-poincare-inverse-element}, we can write
    \begin{equation*}
        U(\Lambda, a) U(\Lambda',a') U^{-1}(\Lambda,a) =  = U(\Lambda \Lambda' \Lambda^{-1}, a + \Lambda a' - \Lambda \Lambda' \Lambda^{-1}a) .
    \end{equation*}
    If we consider infinitesimal transformations~\eqref{eq:infinitesimal-poicare-transformation} about the identity $U(\1,0)$, we have
    \begin{align*}
        U(\Lambda,a) &\simeq \1 -\frac{i}{2} \omega_{\mu\nu} M^{\mu\nu} - i \epsilon_\sigma P^\sigma, \\ 
        {U^{-1}}(\Lambda,a) &\simeq \1 +\frac{i}{2} \omega_{\mu\nu} M^{\mu\nu} + i \epsilon_\sigma P^\sigma .
    \end{align*}
    Substituting and keeping only linear terms of the parameters $\omega_{\mu\nu}$, $\omega'_{\mu\nu}$, $\epsilon_\sigma$, $\epsilon'_\sigma$, the left-hand side of the starting relations reads
    \begin{align*}
        &\left[ \1 -\frac{i}{2} \omega_{\mu\nu} M^{\mu\nu} - i \epsilon_\alpha P^\alpha  \right] \left[ \1 -\frac{i}{2} \omega'_{\rho\sigma} M^{\rho\sigma} - i \epsilon'_\beta P^\beta  \right] \left[ \1 +\frac{i}{2} \omega_{\tau\xi} M^{\tau\xi} + i \epsilon_\gamma P^\gamma  \right] \\
        &=\left[ \1 - \frac{i}{2} \omega' M - i \epsilon' P - \frac{i}{2} \omega M - \frac{1}{4} \omega \omega' M^2 - \frac{1}{2} \omega \epsilon' M P - i \epsilon P - \frac{1}{2} \epsilon \omega' P M - \epsilon \epsilon' P^2 \right] \left[ \1 + \frac{i}{2} \omega M + i \epsilon P  \right] \\
        &= \1 - \frac{i}{2} \omega' M - i \epsilon' P - \frac{i}{2} \omega M - \frac{1}{4} \omega \omega' M^2 - \frac{1}{2} \omega \epsilon' M P - i \epsilon P - \frac{1}{2} \epsilon \omega' P M - \epsilon \epsilon' P^2 \\
        &+\frac{i}{2}\omega M + i \epsilon P \\
        &+\frac{1}{4} \omega' \omega M^2 + \frac{1}{2} \omega' \epsilon M P \\
        &+\frac{1}{2} \epsilon' \omega P M + \epsilon' \epsilon P^2 \\
        &+\frac{1}{4} \omega^2 M^2 + \frac{1}{2} \omega \epsilon M P \\
        &-\frac{i}{8} \omega \omega' \omega M^3 - \frac{i}{4} \omega \omega' \epsilon M^2 P \\
        &-\frac{i}{4} \omega \epsilon' \omega M P M - \frac{i}{2} \omega \epsilon' \epsilon M P^2 \\
        &+\frac{1}{2} \epsilon \omega P M + \epsilon^2 P^2 \\
        &-\frac{i}{4} \epsilon \omega' \omega P M^2 - \frac{i}{2} \epsilon \omega' \epsilon P M P \\
        &-\frac{i}{2} \epsilon \epsilon' \omega P^2 M - i \epsilon \epsilon' \epsilon P^3 \\
        &= \1 -\frac{i}{2} \omega' M - i \epsilon' P \\
        &- \frac{1}{4} \omega_{\mu\nu} \omega'_{\rho\sigma} \comm{M^{\mu\nu}}{M^{\rho\sigma}} \\
        &- \epsilon_\alpha \epsilon'_\beta \comm{P^\alpha}{P^\beta} \\
        &+\frac{1}{2} \omega_{\mu\nu} \epsilon_\alpha \{ M^{\mu\nu}, P^\alpha \} - \frac{1}{2} \omega'_{\rho\sigma} \epsilon_\alpha \comm{P^\alpha}{M^{\rho\sigma}} + \frac{1}{2} \omega_{\mu\nu} \epsilon'_\beta \comm{P^\beta}{M^{\mu\nu}}.
    \end{align*}

    For the right-hand side, we use $\tensor{\Lambda}{^\mu_\nu} \simeq \delta^\mu_\nu + \tensor{\omega}{^\mu_\nu}$ to compute, at infinitesimal level,
    \begin{align*}
        \tensor{(\Lambda\Lambda'\Lambda^{-1})}{^\mu_\nu} &= \tensor{\Lambda}{^\mu_\rho}\tensor{(\Lambda')}{^\rho_\kappa}\tensor{(\Lambda^{-1})}{^\kappa_\nu} \simeq (\delta^\mu_\rho + \tensor{\omega}{^\mu_\rho})(\delta^\rho_\kappa + \tensor{{\omega'}}{^\rho_\kappa})(\delta^\kappa_\nu - \tensor{\omega}{^\kappa_\nu}) \\
        &= \delta^\mu_\nu + \tensor{{\omega'}}{^\mu_\nu} + \tensor{\omega}{^\mu_\rho}\tensor{{\omega'}}{^\rho_\nu} - \tensor{{\omega'}}{^\mu_\kappa}\tensor{\omega}{^\kappa_\nu} + O(\omega^2) + O({\omega'}^2)
    \end{align*}
    Then, using also $a^\mu \simeq \epsilon^\mu$, we may compute
    \begin{align*}
        (\Lambda a')^\mu = \tensor{\Lambda}{^\mu_\nu} a'^\nu &\simeq (\delta^\mu_\nu + \tensor{\omega}{^\mu_\nu}) \epsilon'^\nu = \epsilon'^\mu + \tensor{\omega}{^\mu_\nu} \epsilon'^\nu , \\
        \tensor{(\Lambda\Lambda'\Lambda^{-1})}{^\mu_\nu} a^\nu &\simeq \epsilon^\mu + \tensor{{\omega'}}{^\mu_\nu} \epsilon^\nu + \tensor{\omega}{^\mu_\kappa}\tensor{{\omega'}}{^\kappa_\nu} \epsilon^\nu - \tensor{{\omega'}}{^\mu_\kappa}\tensor{\omega}{^\kappa_\nu} \epsilon^\nu ,
    \end{align*}
    from which
    \begin{equation*}
        (a + \Lambda a' - \Lambda \Lambda' \Lambda^{-1}a)^\mu \simeq \epsilon'^\mu + \tensor{\omega}{^\mu_\nu} \epsilon'^\nu - \tensor{{\omega'}}{^\mu_\nu} \epsilon^\nu + \tensor{{\omega'}}{^\mu_\kappa}\tensor{\omega}{^\kappa_\nu} \epsilon^\nu - \tensor{\omega}{^\mu_\kappa}\tensor{{\omega'}}{^\kappa_\nu} \epsilon^\nu
    \end{equation*}

    Therefore, using eq.~\eqref{eq:infinitesimal-poicare-transformation}, we can expand $U(\Lambda\Lambda'\Lambda^{-1},a + \Lambda a' - \Lambda\Lambda'\Lambda^{-1}a)$ about the identity $U(\1,0)$, obtaining
    \begin{align*}
        U(\Lambda\Lambda'\Lambda^{-1},a + \Lambda a' - \Lambda\Lambda'\Lambda^{-1}a) &\simeq \1 - \frac{i}{2} (\omega'_{\mu\nu} + \omega_{\mu\kappa}\tensor{{\omega'}}{^\kappa_\nu}-\omega'_{\mu\kappa} \tensor{\omega}{^\kappa_\nu})M^{\mu\nu} \\
        &- i \left( \epsilon'_\mu +\omega_{\mu\nu} \epsilon'^\nu - \omega'_{\mu\nu} \epsilon^\nu + \omega'_{\mu\kappa}\tensor{\omega}{^\kappa_\nu} \epsilon^\nu - \omega_{\mu\kappa} \tensor{{\omega'}}{^\kappa_\nu} \epsilon^\nu \right) P^\mu \\
        &\simeq \1 - \frac{i}{2} (\omega'_{\mu\nu} + \omega_{\mu\kappa}\tensor{{\omega'}}{^\kappa_\nu}-\omega'_{\mu\kappa} \tensor{\omega}{^\kappa_\nu})M^{\mu\nu} \\
        &- i \left( \epsilon'_\mu +\omega_{\mu\nu} \epsilon'^\nu - \omega'_{\mu\nu} \epsilon^\nu\right) P^\mu ,
    \end{align*}
    which we can compare to the previous expansion. Indeed, comparing the two we obtain
    \begin{align*}
        &\1 -\frac{i}{2} \omega' M - i \epsilon' P 
        - \frac{1}{4} \omega_{\mu\nu} \omega'_{\rho\sigma} \comm{M^{\mu\nu}}{M^{\rho\sigma}} 
        - \epsilon_\alpha \epsilon'_\beta \comm{P^\alpha}{P^\beta} \\
        &+\frac{1}{2} \omega_{\mu\nu} \epsilon_\alpha \{ M^{\mu\nu}, P^\alpha \} - \frac{1}{2} \omega'_{\rho\sigma} \epsilon_\alpha \comm{P^\alpha}{M^{\rho\sigma}} + \frac{1}{2} \omega_{\mu\nu} \epsilon'_\beta \comm{P^\beta}{M^{\mu\nu}} \\
        &= \1 - \frac{i}{2} \omega'_{\mu\nu} M^{\mu\nu} - \frac{i}{2} \omega_{\mu\kappa}\tensor{{\omega'}}{^\kappa_\nu} M^{\mu\nu} + \frac{i}{2} \omega'_{\mu\kappa} \tensor{\omega}{^\kappa_\nu} M^{\mu\nu} \\
        &  - i \epsilon'_\mu P^\mu - i \omega_{\mu\nu} \epsilon'^\nu P^\mu + i \omega'_{\mu\nu} \epsilon^\nu P^\mu   .
    \end{align*}
    \begin{itemize}
        \item From the $\omega\omega'$ term we can read the commutator $\comm{M^{\mu\nu}}{M^{\rho\sigma}}$,
    \begin{align*}
       - \frac{1}{4} \omega_{\mu\nu} \omega'_{\rho\sigma} \comm{M^{\mu\nu}}{M^{\rho\sigma}} = -\frac{i}{2} \omega_{\mu\kappa} \tensor{{\omega'}}{^\kappa_\nu} M^{\mu\nu} + \frac{i}{2} \omega'_{\mu\kappa} \tensor{\omega}{^\kappa_\nu} M^{\mu\nu} .
    \end{align*}
    Now, we can rewrite the right-hand side in the following way, subtracting a piece to make everything antisymmetric in $\mu\nu$:
    \begin{align*}
        -\frac{i}{2} \omega_{\mu\kappa} \tensor{{\omega'}}{^\kappa_\nu} M^{\mu\nu} + \frac{i}{2} \omega'_{\mu\kappa} \tensor{\omega}{^\kappa_\nu} M^{\mu\nu} = &-\frac{i}{4} \omega_{\mu\nu} \omega'_{\rho\sigma}( \eta^{\nu\rho}M^{\mu\sigma} - \eta^{\mu\rho}M^{\nu\sigma} ) \\
        &+ \frac{i}{4} \omega_{\mu\nu}\omega'_{\rho\sigma} ( \eta^{\mu\sigma}M^{\rho\nu} -\eta^{\nu\sigma}M^{\rho\mu})
    \end{align*}
    Therefore, by comparing, we obtain
    \begin{equation*}
        \comm{M^{\mu\nu}}{M^{\rho\sigma}} = i ( \eta^{\nu\rho}M^{\mu\sigma} - \eta^{\mu\rho}M^{\nu\sigma} - \eta^{\mu\sigma}M^{\rho\nu} + \eta^{\nu\sigma}M^{\rho\mu}),
    \end{equation*}
    which is exactly eq.~\eqref{eq:comm-poincare-lorentz}.
    \item Looking at the $\omega' \epsilon$ terms, we get
    \begin{equation*}
        \frac{1}{2} \omega'_{\rho\sigma} \epsilon_\alpha \comm{M^{\rho\sigma}}{P^\alpha} = -i \omega'_{\mu\nu} \epsilon^\nu P^\mu .
    \end{equation*}
    Relabelling and makind sure that the right-hand side is antisymmetric with respect to the indices of $\omega'$ (which are $\rho\sigma$ above), we get
    \begin{equation*}
        -\frac{1}{2} \epsilon_{\mu} \omega'_{\rho\sigma}\comm{P^\mu}{M^{\rho\sigma}} = -\frac{i}{2} \epsilon_\mu \omega'_{\rho\sigma}  ( \eta^{\mu\sigma} P^\rho - \eta^{\mu\rho}P^{\sigma} ) ,
    \end{equation*}
    which leads to
    \begin{equation*}
        \comm{P^\mu}{M^{\rho\sigma}} = i (\eta^{\mu\sigma} P^\rho - \eta^{\mu\rho} P^\sigma),
    \end{equation*}
    which is \color{red} quasi, cazzo\color{black} eq.~\eqref{eq:comm-poincare-lorentzmomentum}.
    \item Since there's no $\epsilon \epsilon'$ on the right-hand side, we immediatly have
    \begin{equation*}
        \comm{P^\mu}{P^\nu} = 0,
    \end{equation*}
    which is eq.~\eqref{eq:comm-poincare-momentum}.
\end{itemize}
\end{proof}
%%%%%%%% end proof
    

To derive the commutation relations of the generators, we can use the relation
\begin{equation}
    U(\Lambda, a) U(\Lambda',a') U^{-1}(\Lambda,a) = U(\Lambda \Lambda' \Lambda^{-1}, a + \Lambda a' - \Lambda \Lambda' \Lambda^{-1}a) ,
\end{equation}
which follows from eq.~\eqref{eq:representation-poincare-group-multiplication} and eq.~\eqref{eq:representation-poincare-inverse-element}. Inserting infinitesimal transformations~\eqref{eq:infinitesimal-poicare-transformation} for each $U(\Lambda = 1 + \epsilon, a)$, with $U^{-1} (\Lambda,a) = U(1-\epsilon,-a)$, keeping only linear terms in all group parameters $\epsilon$, $\epsilon'$, $a$ and $a'$, and comparing coefficients of the terms $\sim \epsilon\epsilon'$, $a\epsilon'$, $\epsilon a'$ and $aa'$, leads to the identities

which define the Poincaré algebra.

We can recast those relations in a less compact but more useful form. Since $M^{\mu\nu}$ contains six generators, we can define the generator of $SO(3)$ rotations $\vec{J}$ and the generator of the boosts $\vec{K}$ via\footnote{To see how to invert expressions with Levi-Civita symbol, look at apprendix~\ref{app:levi-civita}.}
\begin{equation}\label{eq:poincare-boosts-rotations}
    M^{ij} = - \epsilon_{ijk} J^k \iff J^i =- \frac{1}{2} \epsilon_{ijk} M^{jk}, \quad M^{0i} = K^i, 
\end{equation}
then the commutation relations takes the form

\begin{subequations}
\label{eq:poincare-commutation-boosts-rotations}
\noindent\centering
    \begin{minipage}{0.48\textwidth}
        \begin{align}
        \comm{J^i}{J^j} &= i \epsilon_{ijk} J^k, \\
        \comm{J^i}{K^j} &= i \epsilon_{ijk} K^k ,\\
        \comm{K^i}{K^j} &= -i \epsilon_{ijk} J^k, \\
        \comm{J^i}{P^j} &= i \epsilon_{ijk} P^k ,\\
        \comm{K^i}{P^j} &= i \delta_{ij} P_0,
        \end{align}
    \end{minipage}
    \hfill
    \begin{minipage}{0.48\textwidth}
    \label{eq:bell_states}
        \begin{align}
        \comm{K^i}{P_0} &= i P^i ,\\
        \comm{P^i}{P^j} &= 0, \\
        \comm{J^i}{P_0} &= 0, \\
        \comm{P^i}{P_0} &= 0 .
        \end{align}
    \end{minipage}\bigskip
    \end{subequations}

If we similarly define $\epsilon_{ij} = - \epsilon_{ijk} \phi^k$ and $\epsilon_{0i} = s^i$, we obtain
\begin{equation}
    \frac{i}{2} \epsilon_{\mu\nu} M^{\mu\nu} = i \vec{\phi} \cdot \vec{J} + i \vec{s} \cdot \vec{K}.
\end{equation}

$\vec{J}$ is hermitian but, because the Lorentz group is not compact, $\vec{K}$ is antihermitian for all finite-dimensional representations which prevents them from being unitary. From~\eqref{eq:poincare-commutation-boosts-rotations} we see that boosts and rotations generally do not commute unless the boost and rotation axes coincide. Moreover, $P_0$ (which becomes the Hamilton operator in the quantum theory) commutes with rotations and spatial translations but not with boosts and therefore the eigenvalues of K cannot be used for labelling physical states.


%%%%%%%%%%%%%%% CASIMIR OPERATORS %%%%%%%%%%%%%%%%%
\section{Casimir Operators}

%%%%%%%%%%%%%%% UNITARY REPRESENTATIONS %%%%%%%%%%%%%%%%%
\section{Unitary Representations of the Poincaré Group}
Moving to the quantum theory, as already discussed, \emph{Wigner's theorem} states that continuous symmetries must be implemented by unitary operators on the state space. Since Poincaré group is non-compact, due to the presence of the boosts, all unitary representations must be infinite dimensional. In quantum field theory, fields are promoted to operators which act on the Fock space. The classical action which describes the system is constructed to be Poincaré invariant, therefore, by Noether's theorem, there will be conserved charges, which will be functions of the fields. Those, after quantisation, become hermitian operators on the Fock space and furnish a unitary representation of the Poincaré algebra on the state space. In particular, calling $\hat{M}^{\mu\nu}$ the generators of Lorentz transformations and $\hat{P}^\mu$ those of the translations, we'll have, for an infinitesimal transformation (cfr.~\eqref{eq:infinitesimal-poicare-transformation})
\begin{equation}
    U(\Lambda,a) = e^{\frac{i}{2} \omega_{\mu\nu} \hat{M}^{\mu\nu} + i \epsilon_\mu \hat{P}^\mu} \simeq 1 + \frac{i}{2} \omega_{\mu\nu} \hat{M}^{\mu\nu} + i \epsilon_\mu \hat{P}^\mu .
\end{equation}

To find out the irreducible state space, we want to work with operators which commute with each other, so that states in different such states won't transform into each other. Since $\comm{\hat{P}^\mu}{\hat{P}^\nu} = 0$, it's natural to express physical state vectors in terms of eigenvectors of $\hat{P}^\mu$. We'll further denote with $\sigma$ all the other quantum numbers necessary to describe the state. We're interested in \emph{one-particle states}, which in QFT are defined, in a free theory, as eigenstates of the number operator of eigenvalue one. For such a state, the labels $\sigma$ must be \emph{discrete}. Therefore, we start from
\begin{equation}\label{eq:unitary-poincare-eigenstate-momentum}
    \hat{P}^\mu \ket{p,\sigma} = p^\mu \ket{p,\sigma} .
\end{equation}

Pure translations form an abelian subgroup of the Poincaré group and are represented, on the state space, by
\begin{equation}\label{eq:unitary-poincare-translations}
    U(1, a) = e^{-i \epsilon_\mu \hat{P}^\mu} .
\end{equation}
Combining~\eqref{eq:unitary-poincare-eigenstate-momentum} with~\eqref{eq:unitary-poincare-translations}, we can explicitly see how a state transforms under translations
\begin{equation}\label{eq:unitary-poincare-state-translation}
    U(1,a)\ket{p,\sigma} = e^{-i \epsilon_\mu \hat{P}^\mu} \ket{p,\sigma} = e^{-i \epsilon_\mu p^\mu} \ket{p,\sigma} .
\end{equation}

We must, then find, how such states transform under a Lorentz transformation. We can easily see that acting on $\ket{p,\sigma}$ with a pure Lorentz transformation $U(\Lambda,0) \coloneq U(\Lambda)$, produces an eigenstate of $\hat{P}^\mu$ of eigenvalue $\Lambda p$:
\begin{equation}
    \hat{P}^\mu U(\Lambda) \ket{p,\sigma} = \tensor{\Lambda}{^\mu_\rho} p^\rho U(\Lambda) \ket{p,\sigma}
\end{equation}
\begin{proof}
    \begin{equation*}
    \begin{split}
        \hat{P}^\mu U(\Lambda) \ket{p,\sigma} &= U(\Lambda) \left[U^{-1} \hat{P}^\mu U(\Lambda)\right] \ket{p,\sigma} = U(\Lambda) \left[ \tensor{(\Lambda^{-1})}{_\rho^\mu} \hat{P}^\rho \right] \ket{p,\sigma}  \\
        &= \tensor{\Lambda}{^\mu_\rho} p^\rho U(\Lambda) \ket{p,\sigma} .
    \end{split}
    \end{equation*}
    where we used $U^{-1}U=\1$ in the first equality, eq.\color{red}\dots\dots\color{black} in the second, and finally eq.~\eqref{eq:unitary-poincare-eigenstate-momentum}, together with the well-known relation $\tensor{(\Lambda^{-1})}{_\rho^\mu} = \tensor{\Lambda}{^\mu_\nu}$
\end{proof}

Hence, because of eq.~\eqref{eq:unitary-poincare-eigenstate-momentum}, $U(\Lambda)\ket{p,\sigma}$ must be a linear combination of $\ket{\Lambda p,\sigma}$:
\begin{equation}\label{eq:temp-1}
    U(\Lambda)\ket{p,\sigma} =  \sum_{\sigma'} C_{\sigma' \sigma} (\Lambda, p) \ket{\Lambda p,\sigma'}.
\end{equation}

In general, by using suitable linear combinations of the $\ket{p,\sigma}$, it may be possible to choose the $\sigma$ labels in such a way that the matrix $C_{\sigma' \sigma} (\Lambda, p)$ is block-diagonal. In this case, $C_{\sigma' \sigma} (\Lambda, p)$ would be a reducible representation, and the $\ket{p,\sigma}$ with $\sigma$ in any given block, will transform within the same block, i.e., according to an irreducible representation. 

Therefore, it's natural to identity the states of a \emph{specific particle} as those transforming under irreducible representations of the Poincaré group. This approach ensures that a Poincaré transformation can't change the type of particle. To construct such irreducible representation, we'll make use of the Casimir operators of the group. Roughly speaking, the eigenvalues of the Casimir operators labels different irreducible representations, which correspond to Poincaré transformations acting on \emph{different types} of particles. Since those operators commute with all the generators of the group (and so with each group element via the exponential map), their eigenvalues remain constant under any Poincaré transformation, preserving the identification of the particle type. Moreover, states that are transformed into one another within an irreducible state space are interpreted as different states of the same particle.

It's important to emphasise, however, that the eigenvalues of the Casimir operators do \emph{not} uniquely characterise an irreducible representation. For example, as we will see, the Casimirs vanish for all massless (helicity) representations. Since a comprehensive mathematical treatment would be quite technical and wouldn't provide additional physical insight, we'll stick to the interpretation of an irreducible representation as one that can't change the labels which characterise the type of particle. 

Additionally, let's note that physicists often employ an abuse of language by referring to irreducible representations as the states. Keep in mind, however, that the representation is actually the operator acting on the state.

Let's go back to~\eqref{eq:temp-1} and to the problem of working out the structure of the coefficients $C_{\sigma'\sigma}(\Lambda,p)$ in irreducible representations of the Poincaré group. In order to do that, note that the only functions of $p^\mu$ that are left invariant by all transformations $\tensor{\Lambda}{^\mu_\nu} \in SO(1,3)^+$ are $p^2 = \eta_{\mu\nu} p^\mu p^\nu$ and, for $p^2 \leq 0$, also the sign of $p^0$. Hence, for each value of $p^2$, and (for $p^2 \leq 0$) each sign of $p^0$, one can choose a standard four-momentum, say $q^\mu$, and express any $p^\mu$ of this class as
\begin{equation}
    p^\mu = \tensor{L}{^\mu_\nu}(p)q^\nu,
\end{equation}
where $\tensor{L}{^\mu_\nu}$ is some standard proper orthochronous Lorentz transformation that depends on $p^\mu$, and also implicitly on our choice of $q^\mu$. One can then define the states $\ket{p,\sigma}$ of momentum $p^\mu$ by
\begin{equation}\label{eq:temp-2}
    \ket{p,\sigma} \coloneq N(p) U(L(p)) \ket{q,\sigma},
\end{equation}
where $N(p)$ is a numerical normalisation factor. Now, acting on~\eqref{eq:temp-2} with a generic Lorentz transformation $U(\Lambda)$, we find
\begin{equation}\label{eq:temp-3}
    U(\Lambda) \ket{p,\sigma} = N(p) U(L(\Lambda p)) U(L^{-1}(\Lambda p) \Lambda L(p)) \ket{q,\sigma}
\end{equation}
\begin{proof}
    \begin{equation*}
    \begin{split}
        U(\Lambda) \ket{p,\sigma} &= N(p) U(\Lambda) U(L(p)) \ket{q,\sigma} = N(p) U(\Lambda L(p)) \ket{q,\sigma}\\
         &= N(p) U(L(\Lambda p)) U(L^{-1}(\Lambda p) \Lambda L(p)) \ket{q,\sigma} ,
    \end{split}
    \end{equation*}
    where we used the (representation) group composition rule~\eqref{eq:representation-poincare-group-multiplication} two times, together with $L L^{-1} = \1$.
\end{proof}

 The point of this last step is that the Lorentz transformation $L^{-1}(\Lambda p) \Lambda L(p)$ takes $k$ to the following values
\begin{equation}
    k \to p \to \Lambda p \to k .
\end{equation}
Therefore, $L^{-1}(\Lambda p) \Lambda L(p)$ belongs to the \emph{little  group} or \emph{stabiliser group}, i.e., the subgroup made of transformations which leave $q^\mu$ invariant,
\begin{equation}\label{eq:little-group}
    \tensor{W}{^\mu_\nu} q^\nu = q^\mu .
\end{equation}

Using eq.~\eqref{eq:temp-1}, for each $W$ in the little group satisfying~\eqref{eq:little-group}, we have
\begin{equation}
    U(W) \ket{k,\sigma} = \sum_{\sigma'} D_{\sigma'\sigma}(W) \ket{k,\sigma'} ,
\end{equation}
where the coefficients $D(W)$ furnish a representation of the little group. Then, for any $W,\bar{W}$ in the little group,
\begin{equation*}
\begin{split}
    \sum_{\sigma'} D_{\sigma'\sigma} (\bar{W}W) \ket{k,\sigma'} &= U(\bar{W}W)  \ket{k,\sigma} = U(\bar{W}) U(W) \ket{k,\sigma} = U(\bar{W}) \sum_{\sigma''} D_{\sigma''\sigma}(W) \ket{k,\sigma''} \\ &= \sum_{\sigma' \sigma''} D_{\sigma''\sigma}(W) D_{\sigma'\sigma''}(\bar{W}) \ket{k,\sigma'} ,
\end{split}
\end{equation*}
and so
\begin{equation}
    D_{\sigma'\sigma}(\bar{W}W) = \sum_{\sigma''} D_{\sigma'\sigma''} (\bar{W}) D_{\sigma'' \sigma}(W) .
\end{equation}

In particular, for $W(\Lambda,p) \equiv L^{-1}(\Lambda p) \Lambda L(p)$, eq.~\eqref{eq:temp-3} becomes
\begin{equation}
   U(\Lambda) \ket{p,\sigma} = N(p) \sum_{\sigma'} D_{\sigma'\sigma}(W(\Lambda,p))U(L(\Lambda p)) \ket{q,\sigma'} ,
\end{equation}
or, recalling the definition~\eqref{eq:temp-2},
\begin{equation}\label{eq:induced-representation}
    U(\Lambda) \ket{p,\sigma} = \frac{N(p)}{N(\Lambda p)} \sum_{\sigma'} D_{\sigma'\sigma} (W(\Lambda, p)) \ket{\Lambda p, \sigma'} .
\end{equation}

We won't go into the details of the normalisation here. We simply note that, by the usual orthonormalization procedure in quantum mechanics, we may choose states with standard momentum $q^\mu$ to be orthonormal, in the sense that
\begin{equation}
    \braket{k',\sigma'}{k,\sigma} = \delta^{(3)}(\vec{k}' - \vec{k}) \delta_{\sigma'\sigma} ,
\end{equation}
and by suitable choices of the coefficients in~\eqref{eq:induced-representation}, we can normalise states of arbitrary momentum as
\begin{equation}
    \braket{p',\sigma'}{p,\sigma} = \delta^{(3)}(\vec{p}' - \vec{p}) \delta_{\sigma'\sigma} .
\end{equation}

By eq.~\eqref{eq:induced-representation}, the problem of determining the coefficients $C_{\sigma'\sigma}$ in the transformation rule\color{red}\dots\color{black} has been reduced to the problem of determining the coefficients $D_{\sigma'\sigma}$. In other words, the problem of determining all possible irreducible representations of the Poincaré group has been reduced to the problem of finding all possible irreducible representations of the little group, depending on the class of momentum to which $q^\mu$ belongs. This approach of deriving the representations of a group from the representations of its little group is called \emph{method of induced representations}.



\chapter{Gigi Bellissimo}
\lipsum{10}
%\section{Still to be finished}
\paragraph{Field representations.}




\section{Questions}
\paragraph{Why do we study representations of the algebra and not directly of the group, whatever it means? Even more so, considering that not all irreps of the algebra fit to irreps of the group, if the latter is not simply connected.}
\begin{enumerate}
    \item First, we sometimes know the algebra. Indeed, in SUSY we started constructing the algebra adding some fermionic generators, with no reference to a group or manifold. So, at least it's useful to do so.
    \item Then, as we've seen at the end of section~\ref{sec:symmetries-qm}, in quantum mechanics we're interested in \emph{unitary projective representations} of a group, rather than linear representations, and theorem~\ref{th:unitary-rep} tells us that \emph{all} unitary projective representations of a group $G$ arises from a unitary \emph{linear} representation of the universal covering group $\tilde{G}$. \color{red} As far as I know this is true for finite dimensional representations, is it still true in infinite dimensions? \color{black}
\end{enumerate}

\paragraph{Is it correct to say that the eigenvalues of the Casimir operators do \emph{not} uniquely characterise an irreducible representation. For example, the Casimirs vanish for all massless (helicity) representations?}

Even if this is true, we can still make use of the usual considerations, by simply noticing that helicity is Lorentz invariant. So, we can consider states with different helicities as states of different particles type, since I can't go from one helicity to another using a Lorentz transformation.

\paragraph{I can see a field in classical theory as an infinite dimensional representation of Poicaré. Further, in the quantum theory, a field becomes an operator on the Fock space, and the representation of the group is given by the Noether charges associated to the Poincaré invariance of the action. How do these pictures reconcile?}

\paragraph{A generic Poincare transformation can be written how?}
Some references write 
\begin{equation}
    U(\Lambda,a) = e^{\frac{i}{2}\omega_{\mu\nu}M^{\mu\nu}-i\epsilon_{\sigma} P^{\sigma}},
\end{equation}
while others
\begin{equation}
    U(\Lambda,a) = e^{\frac{i}{2}\omega_{\mu\nu}M^{\mu\nu}}  e^{-i\epsilon_{\sigma} P^{\sigma}} .
\end{equation}
They should not  be equivalent, due to the BCH formula
\begin{equation}
    \exp(X) \circ \exp(Y) = \exp\left( X + Y + \frac{1}{2}\comm{X}{Y} + \frac{1}{12} \comm{X}{\comm{X}{Y}} - \frac{1}{12} \comm{Y}{\comm{X}{Y}} + \dots \right),
\end{equation}
and the commutator
\begin{equation}
    i \comm{P^\mu}{M^{\rho\sigma}} = \eta^{\mu\rho} P^\sigma - \eta^{\mu\sigma} P^\rho .
\end{equation}
Then? Further, is it surjective, as for Lorentz group, as discussed above?

\color{black}
%\color{blue}


\section{Group Theory Basics}

\subsection{Lie Groups and Algebra}

\begin{definition}[Group]
A group $(G, \circ)$ is a set $G$ equipped with a composition map, 
\[
\circ : G \times G \to G, \quad (g, h) \mapsto g \circ h \in G,
\]
called group multiplication, which satisfies the following axioms:
\begin{enumerate}
    \item Closure: $\forall g_1, g_2 \in G \Rightarrow g_1 \circ g_2 \in G$.
    \item Associativity: $\forall g_1, g_2, g_3 \in G \Rightarrow (g_1 \circ g_2) \circ g_3 = g_1 \circ (g_2 \circ g_3)$.
    \item Identity element: $\exists e \in G$ such that $\forall g \in G \Rightarrow g \circ e = e \circ g = g$.
    \item Inverse element: $\forall g \in G, \exists g^{-1} \in G$ such that $g \circ g^{-1} = g^{-1} \circ g = e$.
\end{enumerate}
If, in addition, $\forall g_1, g_2 \in G \Rightarrow g_1 \circ g_2 = g_2 \circ g_1$, then $G$ is said to be an \emph{abelian} group.
\end{definition}

\begin{definition}[Subgroup]
Given a group $(G, \circ)$, a \emph{subgroup} $(H, \circ)$ is a subset $H \subset G$ that itself satisfies the group axioms with the composition inherited from $G$.
\end{definition}

\noindent
\textbf{Remark.} For $H$ to be a subgroup, it must contain the same identity element $e \in G$. While an abelian group admits only abelian subgroups, a non-abelian group can have both abelian and non-abelian subgroups.

A group can be finite or infinite based on its number of elements. The \emph{dimension} of a group $G$, denoted by $\dim(G)$, represents the number of real parameters required to specify an element of $G$.

\begin{definition}[Group Homomorphism and Isomorphism]
A \emph{group homomorphism} is a function between two groups that preserves the group structure. For groups $(G_1, \circ)$ and $(G_2, \cdot)$ with $g_1, g_2 \in G_1$, a map $f : G_1 \to G_2$ is a homomorphism if
\[
f(g_1) \cdot f(g_2) = f(g_1 \circ g_2).
\]
A \emph{group isomorphism} is a bijective homomorphism.
\end{definition}

\begin{definition}[One-parameter Subgroup]
A \emph{one-parameter subgroup} is a continuous group homomorphism
\[
\varphi : \mathbb{R} \to G,
\]
where $\mathbb{R}$ is considered as an additive group. If $\varphi$ is injective, its image $\varphi(\mathbb{R})$ forms a subgroup of $G$, isomorphic to $\mathbb{R}$.
\end{definition}

\begin{definition}[Lie Group]
A \emph{Lie group} is a group whose elements depend continuously and differentiably on a set of real parameters $\theta^a$, where $a = 1, \dots, N$. Thus, a Lie group is both a group and a differentiable manifold.
\end{definition}

\noindent
\textbf{Remark.} While a Lie group could be defined as a Hausdorff topological group that behaves like a transformation group near the identity, we will focus on its manifold structure, which is more relevant to our purposes.

\noindent
Each element of the Lie group can be represented as a point on its manifold, and the dimension of the group matches that of the manifold. We denote a generic element as $g(\theta)$ and choose the coordinates $\theta^a$ such that the identity element $e$ of the group corresponds to $\theta^a = 0$, i.e., $g(0) = e$. This structure allows us to expand group elements in a Taylor-like series and consider elements infinitesimally close to the identity. The set of these “infinitesimal” elements forms the tangent space at the identity $e$, which is the basis of the Lie algebra. 

To clarify, we introduce the following definitions.

\begin{definition}[Tangent Space]
Let $M$ be a manifold. For each point $x \in M$, the \emph{tangent space} at $x$, $T_xM$, is the space of tangent vectors
\[
v = \left. \frac{d}{dt} \right|_{t=0} \gamma(t) \in T_xM,
\]
where $\gamma : \mathbb{R} \to M$ is any curve on the manifold passing through $x$. The dimension of $T_xM$ matches that of $M$.
\end{definition}

\begin{definition}[Lie Algebra]
For a Lie group $G$ with identity $e \in G$, the \emph{Lie algebra} $\mathfrak{g}$ is defined as the tangent space at the identity, i.e., $\mathfrak{g} = T_eG$.
\end{definition}

Moving forward, we will introduce additional definitions and theorems that will be essential in our exploration of the Poincaré group and its properties. Where possible, we will focus on the manifold structure of Lie groups and specify relevant topological aspects.

\begin{definition}[Connected Space]
A topological space is said to be \emph{connected} if it cannot be divided into two or more disjoint non-empty open subsets.
\end{definition}

\begin{definition}[Connected Component]
For a group $G$, the \emph{connected component} (also known as the identity component) is the largest connected subgroup of $G$ containing the identity element.
\end{definition}

\begin{definition}[Simply Connected]
A Lie group $G$ is \emph{simply connected} if any two paths $\gamma(t)$ and $\gamma'(t)$, which share the same endpoints, can be continuously deformed into one another.
\end{definition}

\noindent
The Lorentz group, for instance, has four connected components, of which we choose the proper orthochronous part. Since it is not simply connected, the following theorem becomes relevant.

\begin{theorem}
If a Lie group $G$ is not simply connected, there exists another Lie group $\tilde{G}$ with an isomorphic Lie algebra, $\tilde{\mathfrak{g}} \cong \mathfrak{g}$, that is simply connected. This $\tilde{G}$ is known as the \emph{universal cover} of $G$, and there exists a surjective group homomorphism $\pi : \tilde{G} \to G$.
\end{theorem}

Thus, the algebras of $\tilde{G}$ and $G$ coincide, i.e., $\tilde{\mathfrak{g}} \cong \mathfrak{g}$. For each point $g \in \tilde{G}$, there exists an open neighborhood $g \in U \subset \tilde{G}$ such that the restriction $\pi|_U : U \to \pi(U)$ is a diffeomorphism. Consequently, the tangent space at any $g \in \tilde{G}$ is isomorphic to the tangent space at $\pi(g) \in G$, including the tangent space at the identity.

For the proper orthochronous Lorentz group, this universal cover is the spin group, which is isomorphic to $SL(2; \mathbb{C})$.

\begin{definition}[Compact Space]
A topological space $X$ is \emph{compact} if every open cover of $X$ has a finite subcover. Roughly speaking, a Lie group is compact if its parameter space is bounded.
\end{definition}

\begin{theorem}
Every compact Lie group is isomorphic to a matrix group.
\end{theorem}

Due to the presence of boosts, the Lorentz group is non-compact, which implies that its unitary representations must be infinite-dimensional. This property directs us towards studying representations on one-particle Hilbert spaces.

We now introduce the \emph{exponential map}, a key tool for obtaining elements of $G$ from its Lie algebra $\mathfrak{g}$.

\begin{definition}[Exponential Map]
Let $G$ be a Lie group and $\mathfrak{g}$ its Lie algebra. The exponential map is defined as
\[
\exp: \mathfrak{g} \to G,
\]
where for $X \in \mathfrak{g}$, $\exp(X) = \gamma(1)$, with $\gamma : \mathbb{R} \to G$ as the unique one-parameter subgroup of $G$ such that the tangent vector at the identity equals $X$.
\end{definition}

The exponential map allows us to explore the properties of $G$ via its Lie algebra. Although generally the image $\operatorname{Im}(\exp) \subset G$ is only a neighborhood of the identity and not necessarily surjective, for connected and compact groups $G$ it is indeed surjective. However, due to the boosts, even the proper orthochronous Lorentz group is non-compact, so exponentiating the Lie algebra elements may not span all elements of $G$. For our purposes, we will assume that, in the cases of interest, each element of the connected component of $G$ can be obtained from its Lie algebra $\mathfrak{g}$.

The Lie algebra multiplication, called the \emph{Lie bracket} or \emph{commutator}, is given by
\[
[ \cdot, \cdot ] : \mathfrak{g} \times \mathfrak{g} \to \mathfrak{g}, \quad (X, Y) \mapsto [X, Y],
\]
which is antisymmetric,
\[
[X, Y] = -[Y, X],
\]
and satisfies the Jacobi identity,
\[
[X, [Y, Z]] + [Y, [Z, X]] + [Z, [X, Y]] = 0.
\]
For connected groups, the Baker-Campbell-Hausdorff (BCH) formula relates group composition to Lie algebra multiplication:
\[
\exp(X) \circ \exp(Y) = \exp \left( X + Y + \frac{1}{2}[X, Y] + \frac{1}{12}[X, [X, Y]] - \frac{1}{12}[Y, [X, Y]] + \dots \right).
\]

Given a basis $\{ T_a \}$ for the Lie algebra $\mathfrak{g}$, the Lie bracket structure can be encoded in terms of structure constants $f^c_{ab}$, where
\[
[T_a, T_b] = i f^c_{ab} T_c.
\]
These structure constants uniquely define the Lie algebra. Through the exponential map and the BCH formula, we express elements of the Lie group in terms of exponentials of linear combinations of the basis elements $\{T_a\}$, called \emph{generators} of the Lie algebra. For any $g \in G$, we have
\[
g = e^{i \theta^a T_a}.
\]


\subsection{Group Representations}

So far, we have treated groups and algebras as abstract entities. To apply these concepts and interpret them as symmetry transformations, we need to “represent” them concretely.

\begin{definition}[Group Representation]
A \emph{(linear) representation} of a group $G$ is a group homomorphism (see Definition 3) $\rho : G \to \operatorname{Aut}(V)$, mapping $G$ to the group of automorphisms\footnote{An automorphism is an isomorphism from an object to itself. In this context, the group of automorphisms is the set of invertible linear maps on a vector space.} on a vector space $V$. Formally, 
\[
\rho(e) = \operatorname{id}_V, \quad \rho(g \circ h) = \rho(g)\rho(h), \quad \rho(g^{-1}) = \rho(g)^{-1}, \quad \forall g, h \in G.
\]
For finite-dimensional vector spaces where $\dim(V) = n < \infty$, we have $\operatorname{Aut}(V) \cong GL(n; K)$, so elements of the group are represented by $n \times n$ matrices with multiplication defined by matrix group law. The \emph{dimension} of a representation $(\rho, V)$ is the same as that of $V$.
\end{definition}

\noindent
\textbf{Remark.} The dimension of a representation differs from the dimension of the group itself.

From now on, we focus on finite-dimensional representations. For a given representation $(\rho, V)$, the group acts on vectors in $V$ as linear transformations. For each $g \in G$ and $v \in V$, we have
\[
v \mapsto \rho(g)v.
\]

\begin{definition}[Reducible Representation]
A representation is called \emph{reducible} if there exists a non-zero subspace $U \subset V$ such that
\[
\rho(g)u \in U, \quad \forall u \in U, \; g \in G.
\]
If no such subspace exists, $\rho$ is called an \emph{irreducible representation}.
\end{definition}

Given a reducible representation $\rho$, we can always find a basis for $V$ such that
\[
\rho(g) = \begin{pmatrix} \rho'(g) & \beta(g) \\ 0 & \rho''(g) \end{pmatrix},
\]
where the invariant subspace is $U = \{(u, 0)^T \in V\}$. This structure gives rise to a representation $\rho'$ of smaller dimension.

\begin{definition}[Completely Reducible Representation]
A reducible representation is called \emph{completely reducible} if $\beta(g) = 0$. In this case, $\rho$ decomposes into the direct sum of two representations:
\[
\rho \cong \rho' \oplus \rho''.
\]
In other words, in a completely reducible representation, the basis vectors of $V$ can be chosen to split into subsets that remain independent under transformation.
\end{definition}

\begin{definition}[Equivalent Representations]
Two representations $\rho_1$ and $\rho_2$ of the same dimension $n$ are called \emph{equivalent} if there exists an invertible $n \times n$ matrix $S$ such that
\[
\rho_2(g) = S^{-1} \rho_1(g) S, \quad \forall g \in G.
\]
Thus, if there exists a change of basis $S$ on $V$ relating the representations, they are equivalent.
\end{definition}

\begin{definition}[Faithful Representation]
A representation $\rho$ is called \emph{faithful} if
\[
g_1 \neq g_2 \Rightarrow \rho(g_1) \neq \rho(g_2).
\]
For a non-faithful representation, there exists a subset $H \subset G$ such that $\rho(h) = 1$ for all $h \in H$, where $H$ is a subgroup of $G$.
\end{definition}

\begin{definition}[Unitary Representation]
A \emph{unitary representation} is a complex representation, $\rho : G \to GL(n; \mathbb{C})$, where each $\rho(g)$ is a unitary matrix, meaning
\[
\rho(g)^{-1} = \rho(g)^{\dagger}.
\]
\end{definition}

To conclude this section, we introduce specific types of representations that will be used in the following.

\begin{definition}[Matrix Representation]
Let $(\rho, V)$ be a representation. In a \emph{matrix representation}, $V$ is a finite-dimensional vector space (dim $V = n$), and each group element $g \in G$ is represented by an $n \times n$ matrix $\rho(g)_{ij}$, where $i, j = 1, \dots, n$. For any vector $v = (v_1, \dots, v_n) \in V$, the action of $g$ on $V$ is given by
\[
v_i \mapsto \sum_{j=1}^n \rho(g)_{ij} v_j.
\]
\end{definition}

\begin{definition}[Fundamental Representation]
The \emph{fundamental representation} of a group $G$ is the representation $D$ such that, for any $v \in V$,
\[
D(g)v = gv, \quad D(T^a) = T^a.
\]
\end{definition}

\begin{definition}[Conjugate Representation]
The \emph{conjugate representation} $\overline{D}$ is defined by
\[
\overline{D}(g)v = g^* v, \quad g^* = \left( e^{i \theta^a T^a} \right)^* = e^{-i \theta^a (T^a)^*}.
\]
In this representation, $\overline{D}(T^a) = -(T^a)^*$.
\end{definition}

\subsection{Algebra Representations}

We can similarly define a representation for a Lie algebra $\mathfrak{g}$. Note that a Lie algebra can be defined independently of any associated Lie group, simply as a vector space $\mathfrak{g}$ with an antisymmetric product, the Lie bracket $[ \cdot, \cdot ]$.

\begin{definition}[Algebra Representation]
Given a Lie algebra $\mathfrak{g}$, a \emph{representation} of $\mathfrak{g}$ is a vector space $V$ along with an algebra homomorphism (see Definition 13)
\[
\rho_\mathfrak{g} : \mathfrak{g} \to \operatorname{End}(V),
\]
where $\operatorname{End}(V)$ denotes the set of endomorphisms of $V$.
\end{definition}

The space of endomorphisms, $\operatorname{End}(V)$, has a natural vector space structure defined by the addition of linear maps and a non-necessarily invertible product defined by composition. With a chosen basis for $V$, $\operatorname{End}(V)$ becomes a space of matrices, with matrix multiplication as the product.

For compatibility with the Lie algebra structure, the representation $\rho_\mathfrak{g}$ must satisfy
\[
\rho_\mathfrak{g}([X, Y]) = \rho_\mathfrak{g}(X)\rho_\mathfrak{g}(Y) - \rho_\mathfrak{g}(Y)\rho_\mathfrak{g}(X),
\]
where the product on the right-hand side represents matrix multiplication. Given a set of generators $\{ T^a \}$ with structure constants $f^c_{ab}$, we have
\[
\rho_\mathfrak{g}([T^a, T^b]) = f^c_{ab} \rho_\mathfrak{g}(T^c).
\]

We can now demonstrate that any representation $(\rho, V)$ of a Lie group $G$ induces a representation of its Lie algebra $\mathfrak{g}$. For any $g = \exp(tX) \in G$, with $t \in \mathbb{R}$ and $X \in \mathfrak{g}$, the map $\rho(\exp(tX))$ defines a continuous “path” of transformations on the representation space $V$. This leads to a Lie algebra representation $\rho_\mathfrak{g}(X)$ on $V$, defined by
\[
\rho_\mathfrak{g}(X)v := \left. \frac{d}{dt} \rho(\exp(tX))v \right|_{t=0}, \quad \forall v \in V.
\]
Thus, $\rho_\mathfrak{g}(X)$ is a matrix of the same size as $\rho(g)$ and also acts on $V$. By applying the Baker-Campbell-Hausdorff (BCH) formula, one can show that $\rho_\mathfrak{g}$ respects the Lie bracket structure, making $(\rho_\mathfrak{g}, V)$ a valid representation of the Lie algebra.

Consider a parameterization $\rho(g(\theta)) := \rho(\theta)$ for the Lie group $G$, with generators in the representation $\rho$ defined as $T^a_\rho := \rho_\mathfrak{g}(T^a)$. In the neighborhood of the identity, we have
\[
\rho(\theta) \approx \mathbb{1} + i \theta^a T^a_\rho,
\]
with
\[
T^a_\rho \equiv -i \frac{\partial \rho}{\partial \theta^a} \bigg|_{\theta=0}.
\]
For the identity-connected component of the group, any group element $g(\theta)$ can be represented by
\[
\rho(g(\theta)) = \exp(i \theta^a T^a_\rho).
\]

\noindent
\textbf{Remark.} Although the explicit form of the generators $T^a_\rho$ depends on the specific representation, the structure constants $f^c_{ab}$, defined by $[T^a, T^b] = i f^c_{ab} T^c$, remain independent of the representation.

Conversely, not all Lie algebra representations $\rho_\mathfrak{g}$ necessarily extend to representations of the corresponding group $G$. This discrepancy arises from the global topology of $G$. In particular, all representations of the Lie algebra extend to group representations if $G$ is simply connected (see Definition 10). According to Theorem T.1, if $G$ is not simply connected, there exists a universal cover $\tilde{G}$ that is simply connected and has an isomorphic algebra, $\tilde{\mathfrak{g}} \cong \mathfrak{g}$. Consequently, we can construct representations of the algebras, which extend to representations of the covering group $\tilde{G}$.

As an example, the Lorentz group is not simply connected, and its universal cover is $SL(2; \mathbb{C})$. Therefore, not all representations of its Lie algebra extend to representations of the Lorentz group itself.

We now state two important theorems that will be essential in identifying physical observables in quantum mechanics.

\begin{theorem}
All unitary projective representations of a group $G$ originate from unitary linear representations of the universal covering group $\tilde{G}$, which in turn come from representations of the Lie algebra.
\end{theorem}

\begin{theorem}
Non-compact groups have no finite-dimensional unitary representations, except for cases where non-compact generators act trivially, i.e., as zero.
\end{theorem}

\noindent
The physical significance of this second theorem is due to the fact that in a unitary representation, the generators are Hermitian operators. According to the principles of quantum mechanics, only Hermitian operators correspond to observables. Therefore, for a non-compact group, we require an infinite-dimensional representation to identify its generators with physical observables. This requirement leads us to consider representations on the Hilbert space of one-particle states, as we will explore.

\subsection{Casimir Operators}

Casimir operators play a significant role in the study of representations. These operators are constructed from the generators $T^a$ of a Lie algebra and commute with all generators themselves. In each irreducible representation, Casimir operators are proportional to the identity matrix, with the proportionality constant serving to label the representation.

As an example, consider the algebra $\mathfrak{su}(2)$, defined by the commutation relations
\[
[J_i, J_j] = i \epsilon_{ijk} J_k.
\]
The Casimir operator for this algebra is
\[
J^2 = J_1^2 + J_2^2 + J_3^2.
\]
In an irreducible representation, $J^2$ takes the value $j(j + 1)$ times the identity matrix, where $j$ can take values $0, \frac{1}{2}, 1, \dots$.

Casimir operators are essential for characterizing representations because they allow us to classify the different possible representations of a given Lie algebra by their eigenvalues. For example, in the case of $\mathfrak{su}(2)$, the value of $j$ distinguishes different representations. 

For the Poincaré group, which we will discuss in detail later, the Casimir operators will similarly provide essential information on how one-particle states can be classified under the symmetry transformations of spacetime.

\section{Symmetries in Quantum Mechanics}

In this section, we review the concepts of states and observables in quantum mechanics, with a focus on the role of projective representations as opposed to standard representations. 

Let us begin by outlining some fundamental aspects of quantum mechanics.

\begin{itemize}
    \item \textbf{States.} Physical states are represented by rays in a Hilbert space $\mathcal{H}$, which is a complex vector space with a scalar product $\langle \cdot, \cdot \rangle$ defined by
    \begin{subequations}
        \begin{align}
            \langle \phi, \psi \rangle &= \langle \psi, \phi \rangle^*, \\
            \langle \phi, \xi_1 \psi_1 + \xi_2 \psi_2 \rangle &= \xi_1 \langle \phi, \psi_1 \rangle + \xi_2 \langle \phi, \psi_2 \rangle, \\
            \langle \eta_1 \phi_1 + \eta_2 \phi_2, \psi \rangle &= \eta_1^* \langle \phi_1, \psi \rangle + \eta_2^* \langle \phi_2, \psi \rangle,
        \end{align}
    \end{subequations}
    where $\phi, \psi \in \mathcal{H}$. The norm is defined by $\langle \psi, \psi \rangle = \langle \psi | \psi \rangle \geq 0$, and it vanishes only if $\psi \equiv 0$. A ray $R$ is a set of normalized vectors (satisfying $\langle \psi | \psi \rangle = 1$) where $\psi, \psi' \in R$ if $\psi' = \xi \psi$ for some $\xi \in \mathbb{C}$ with $|\xi| = 1$.
    
    \item \textbf{Observables.} Observables are represented by Hermitian operators $A$ on $\mathcal{H}$, defined by the properties
    \begin{subequations}
        \begin{align}
            A(\xi \psi + \eta \phi) &= \xi A \psi + \eta A \phi, \\
            A &= A^\dagger,
        \end{align}
    \end{subequations}
    where $A^\dagger$ denotes the adjoint of $A$, given by
    \[
    \langle \phi, A^\dagger \psi \rangle = \langle A \phi, \psi \rangle = \langle \psi, A \phi \rangle^*.
    \]
    If a state is represented by a ray $R$ and an observable by a Hermitian operator $A$, then the state has a definite value $\alpha$ for the observable if vectors $\psi \in R$ are eigenstates of $A$ with eigenvalue $\alpha$:
    \[
    A \psi = \alpha \psi, \quad \text{for } \psi \in R.
    \]
    If $A$ is Hermitian, $\alpha \in \mathbb{R}$ and eigenstates with different eigenvalues are orthogonal.
    
    \item \textbf{Probabilities.} If a system is in a state represented by a ray $R$, and an experiment tests if it is in one of the mutually orthogonal rays $R_1, R_2, \dots$, then the probability of finding the system in $R_n$ is
    \[
    P(R \to R_n) = |\langle \psi, \psi_n \rangle|^2,
    \]
    where $\psi \in R$ and $\psi_n \in R_n$. If the $\psi_n$ form a complete set, then $\sum_n P(R \to R_n) = 1$.
\end{itemize}

A transformation acting on rays is a \emph{symmetry} if it preserves probabilities. If an observer $\mathcal{O}$ sees a state represented by a ray $R$, or one of several states represented by rays $R_1, R_2, \dots$, then another observer $\mathcal{O}'$ sees the same system in states represented by rays $R', R_1', R_2', \dots$, with matching probabilities:
\[
P(R \to R_n) = P(R' \to R_n').
\]

\noindent
A fundamental result, \emph{Wigner's theorem}, states that for any such ray transformation $T : R \to R'$, there exists a unitary operator $U$ on the Hilbert space such that if $\psi \in R$, then $U \psi \in R'$, with $U$ satisfying
\begin{subequations}
    \begin{align}
        \langle U \psi, U \phi \rangle &= \langle \psi, \phi \rangle, \\
        U(\xi \psi + \eta \phi) &= \xi U \psi + \eta U \phi,
    \end{align}
\end{subequations}
for all $\psi, \phi \in \mathcal{H}$. Using the definition of an adjoint operator, this unitarity condition can also be expressed as
\[
U^\dagger = U^{-1}.
\]

The identity operator $U = 1$ trivially represents a symmetry, preserving the ray $R$. An infinitesimal symmetry transformation close to the identity is represented by a unitary operator near $U = 1$:
\[
U = 1 + i \epsilon T,
\]
where $\epsilon$ is a real infinitesimal parameter. For $U$ to be unitary, $T$ must be both linear and Hermitian, qualifying it as an observable. In physics, most observables arise in this way from symmetry transformations.

It is straightforward to verify that symmetry transformations from one ray to another satisfy the group axioms. Given transformations $T_1 : R_n \to R_n'$ and $T_2 : R_n' \to R_n''$, the transformation $T_2 T_1 : R_n \to R_n''$ is also a symmetry transformation. The inverse transformation $T^{-1} : R_n' \to R_n$ exists, and the identity transformation leaves the ray unchanged.

The operators $U(T)$ corresponding to these symmetry transformations form a group, but unlike the symmetry transformations themselves, the operators $U(T)$ act on vectors in Hilbert space rather than on rays. If $T_1 : R_n \to R_n'$, then $U(T_1)$ acts on $\psi_n \in R_n$ and gives $U(T_1) \psi_n \in R_n'$. Similarly, if $T_2 : R_n' \to R_n''$, then $U(T_2)U(T_1) \psi_n \in R_n''$. Since $U(T_2 T_1)$ is also in the same ray, the vectors $U(T_2) U(T_1) \psi_n$ and $U(T_2 T_1) \psi_n$ differ only by a phase factor $\phi(T_2, T_1)$:
\[
U(T_2) U(T_1) \psi_n = e^{i \phi(T_2, T_1)} U(T_2 T_1) \psi_n.
\]

This type of representation is called a \emph{projective representation}. Unlike a linear representation (Definition 15), it includes a phase factor. Referring back to Section 2.2, a symmetry group $G$ acts on the Hilbert space $\mathcal{H}$ via a unitary operator $\rho(g) : \mathcal{H} \to \mathcal{H}$ for $g \in G$, which defines a projective representation:
\[
\rho(g) \rho(h) = e^{i \phi(g, h)} \rho(g \circ h), \quad \phi(g, h) \in \mathbb{R}.
\]

Therefore, in quantum theory, we are interested in unitary projective representations of a symmetry group $G$. According to Theorem T.3, all unitary projective representations of $G$ can be derived from unitary linear representations of the universal covering group $\tilde{G}$, which, in turn, arise from representations of the Lie algebra. Furthermore, by Theorem T.4, since the Poincaré group is non-compact, we must study infinite-dimensional representations to obtain unitary ones, which can then be associated with physical observables.

\section{The Groups SO(3) and SU(2) and their Representations}

In this section, we review the essential properties of the Lie groups $SO(3)$ and $SU(2)$ and their respective algebras, as these will be useful in analyzing the representations of the Lorentz group.

\subsection{The Group SO(3)}

In its fundamental representation, the special orthogonal group $SO(n)$ consists of $n \times n$ orthogonal matrices satisfying
\[
R \in SO(n) : R^T R = R R^T = 1, \quad \det(R) = 1.
\]
It is straightforward to show that the dimension of $SO(n)$ is $\dim(SO(n)) = \frac{n(n-1)}{2}$. We are specifically interested in the case $n = 3$, giving the three-dimensional rotation group $SO(3)$.

To determine the Lie algebra $\mathfrak{so}(3)$, observe that any $R \in SO(3)$ can be parameterized by three Euler angles, or equivalently, by a rotation axis and an angle about this axis. For a unit vector $\mathbf{n} \in \mathbb{R}^3$ (normalized such that $\mathbf{n} \cdot \mathbf{n} = 1$) and an angle $\theta$ about it, the identity element is given by $R(\theta = 0, \mathbf{n}) = 1$. An infinitesimal rotation by a small angle $\delta \theta$ acts on an arbitrary vector $\mathbf{x} \in \mathbb{R}^3$ as
\[
R(\delta \theta, \mathbf{n}) \mathbf{x} = \mathbf{x} + \delta \theta \, \mathbf{n} \times \mathbf{x} = \mathbf{x} + \delta \theta \begin{pmatrix} n_2 x_3 - n_3 x_2 \\ n_3 x_1 - n_1 x_3 \\ n_1 x_2 - n_2 x_1 \end{pmatrix}.
\]
Choosing $\mathbf{n}$ along each coordinate axis, we obtain
\begin{subequations}
    \begin{align}
        R(\delta \theta, \mathbf{n} = (1, 0, 0)^T) \mathbf{x} &= \mathbf{x} + \delta \theta \begin{pmatrix} 0 \\ -x_3 \\ x_2 \end{pmatrix} = \mathbf{x} + \delta \theta L_1 \mathbf{x}, \\
        R(\delta \theta, \mathbf{n} = (0, 1, 0)^T) \mathbf{x} &= \mathbf{x} + \delta \theta \begin{pmatrix} x_3 \\ 0 \\ -x_1 \end{pmatrix} = \mathbf{x} + \delta \theta L_2 \mathbf{x}, \\
        R(\delta \theta, \mathbf{n} = (0, 0, 1)^T) \mathbf{x} &= \mathbf{x} + \delta \theta \begin{pmatrix} -x_2 \\ x_1 \\ 0 \end{pmatrix} = \mathbf{x} + \delta \theta L_3 \mathbf{x}.
    \end{align}
\end{subequations}
The matrices $L_i$ defined above correspond to a basis for the Lie algebra $\mathfrak{so}(3)$. Since $SO(3)$ is the connected component of $O(3)$ and is compact, each element of the group can be written as an exponential of a linear combination of these generators, as shown in Equation (2.32).

Using the orthogonality condition $R^T R = 1$ and Equation (2.32), together with the Baker-Campbell-Hausdorff (BCH) formula, we find that
\[
1 = R(\theta, \mathbf{n})^T R(\theta, \mathbf{n}) = \exp \left( i \theta \sum_i n_i (L_i^T + L_i) + \dots \right).
\]
For an infinitesimal transformation, this condition requires $L_i^T + L_i = 0$, implying that $L_i$ are antisymmetric matrices. Furthermore, the condition $\det(R) = 1$ implies that $L_i$ are traceless. Thus, in the fundamental representation, the algebra $\mathfrak{so}(3)$ is given by
\[
\mathfrak{so}(3) = \{ \text{antisymmetric, traceless } 3 \times 3 \text{ matrices} \}.
\]
With this basis $\{L_i\}$, we can compute the commutation relations, finding
\[
[J_i, J_j] = i \sum_k \epsilon_{ijk} J_k.
\]

\subsection{The Group SU(2)}

The group $SU(2)$ can be similarly defined in its fundamental representation, where we show that its Lie algebra has the same commutation relations as in Equation (4.5).

\subsection{Representation of the Algebra $\mathfrak{so}(3) \cong \mathfrak{su}(2)$}

We now study the finite-dimensional representations of the algebra $\mathfrak{so}(3) \cong \mathfrak{su}(2)$, defined by the commutation relations
\[
[J_i, J_j] = i \epsilon_{ijk} J_k.
\]
In particular, the physically relevant representations are unitary representations, which act on a vector space $V$ that is also a Hilbert space, equipped with a scalar product $\langle \cdot, \cdot \rangle$ as in Equation (3.1). To find the finite-dimensional representation spaces $V$, we follow the standard quantum mechanical approach.

Define the ladder operators
\[
J_\pm := J_1 \pm i J_2,
\]
which satisfy
\[
[J_3, J_\pm] = \pm J_\pm, \quad [J_+, J_-] = 2 J_3.
\]
Also define the Casimir operator
\[
J^2 := J_1^2 + J_2^2 + J_3^2.
\]
Since the operators $J_i$ are Hermitian, we have
\[
(J_\pm)^\dagger = J_\mp, \quad (J^2)^\dagger = J^2.
\]
One can show that the Casimir operator commutes with all generators:
\[
[J^2, J_i] = [J^2, J_\pm] = 0,
\]
indicating that $J^2$ serves as a Casimir operator. This operator represents the total angular momentum.

The spectral theorem guarantees the existence of an orthonormal basis $\{ |j, m\rangle \}$ of $V$, where $\langle j, m | j', m' \rangle = \delta_{mm'} \delta_{jj'}$, such that $|j, m \rangle$ are simultaneous eigenvectors of $J_3$ and $J^2$:
\[
J_3 |j, m \rangle = m |j, m \rangle, \quad J^2 |j, m \rangle = j(j+1) |j, m \rangle.
\]
Assuming without loss of generality that each eigenspace is one-dimensional, we find that each representation space $V$ has an orthonormal basis $|j, m \rangle$, with $j \in \frac{\mathbb{N}_0}{2}$ and $m \in \{-j, -j+1, \dots, j-1, j\}$.

The action of the ladder operators on these basis states is given by
\[
J_\pm |j, m \rangle = \sqrt{(j \mp m)(j \pm m + 1)} |j, m \pm 1 \rangle.
\]
This shows that different values of $j$ correspond to different representations, each with dimension $2j + 1$. These representations are irreducible and are referred to as \emph{spin-$j$} irreducible representations of $\mathfrak{so}(3) \cong \mathfrak{su}(2)$.

For $j = 1$, we obtain the three-dimensional representation corresponding to the Lie algebra representation induced by the defining representation of $SO(3)$. Representations with integer spin $j > 1$ can be obtained from tensor products of the $j = 1$ representation, while half-integer spin representations correspond to representations of the covering group $SU(2)$, as we discuss next.

\subsection{Half-integer Spin Irreducible Representation}

Let us consider a specific value $j = \tilde{j} \in \frac{\mathbb{N}_0}{2}$ and denote the basis elements of the $(2\tilde{j} + 1)$-dimensional representation space $V$ as $\{ |\tilde{j}, m \rangle \}$ with $m \in \{-\tilde{j}, \dots, \tilde{j}\}$. For an element $X \in \mathfrak{so}(3) \cong \mathfrak{su}(2)$, the corresponding representation matrix on $V$ is given by
\[
M = \rho_{\tilde{j}}(X), \quad M_{ab} = \langle \tilde{j}, a | M | \tilde{j}, b \rangle, \quad M | \tilde{j}, m \rangle = \sum_k | \tilde{j}, k \rangle M_{km}.
\]
Through the exponential map, we obtain a representation of the covering group.

For a $2\pi$ rotation, we find
\[
\rho_{\tilde{j}}[R(\pi)R(\pi)]_{ab} = (-1)^{2\tilde{j}} \delta_{ab}.
\]
This shows that not all irreducible representations of the algebra extend to irreducible representations of $SO(3)$, though they do extend to representations of the universal covering group, $SU(2)$. In quantum mechanics, we are primarily interested in the group $SU(2)$.

\subsection{Relation between SO(3) and SU(2) and Proof of Double Cover}

To conclude, $SU(2)$ is the double cover of $SO(3)$, which means there is a two-to-one mapping from $SU(2)$ to $SO(3)$. This double cover structure allows half-integer spin representations in $SU(2)$ that do not have counterparts in $SO(3)$.

\section{Poincaré Group and Algebra and Their Representations}

To analyze the properties of the Lorentz and Poincaré groups, we begin with their defining representations, keeping in mind which properties are representation-independent. We will work in a four-dimensional Minkowski spacetime with metric tensor $\eta = (\eta_{\mu \nu}) = \operatorname{diag}(+1, -1, -1, -1)$ and scalar product defined by
\[
x \cdot y := x^T \eta y = x^0 y^0 - \mathbf{x} \cdot \mathbf{y} = \eta_{\mu \nu} x^\mu y^\nu = x_\mu y^\mu.
\]

\subsection{Lorentz Group}

Lorentz transformations are those transformations $x \to x' = \Lambda x$ that leave the scalar product invariant, i.e.,
\[
(\Lambda x) \cdot (\Lambda y) = x \cdot y \Rightarrow x^T \Lambda^T \eta \Lambda y = x^T \eta y \Rightarrow \Lambda^T \eta \Lambda = \eta.
\]
In component form, this condition becomes
\[
\eta_{\mu \nu} = \eta_{\alpha \beta} \Lambda^\alpha_{\ \mu} \Lambda^\beta_{\ \nu}.
\]
Since $\eta_{\mu \nu}$ is symmetric, this condition yields 10 constraints. Given that a Lorentz transformation is represented by a $4 \times 4$ matrix, it depends on $16 - 10 = 6$ independent parameters, which can be interpreted as three parameters for boosts and three for rotations.

For an infinitesimal transformation
\[
\Lambda^\mu_{\ \nu} \approx \delta^\mu_\nu + \omega^\mu_{\ \nu},
\]
and using Eq. (5.3), we find that
\[
\omega_{\mu \nu} = -\omega_{\nu \mu}.
\]

\begin{proof}
Starting from the invariance condition $\eta_{\mu \nu} = \eta_{\alpha \beta} \Lambda^\alpha_{\ \mu} \Lambda^\beta_{\ \nu}$ and expanding $\Lambda^\mu_{\ \nu} \approx \delta^\mu_\nu + \omega^\mu_{\ \nu}$, we obtain
\[
\eta_{\mu \nu} \approx \eta_{\mu \nu} + \omega_{\mu \nu} + \omega_{\nu \mu}.
\]
Since $\eta_{\mu \nu} = \eta_{\mu \nu}$, we must have $\omega_{\mu \nu} = -\omega_{\nu \mu}$, as desired.
\end{proof}

The transformations of a space with coordinates $\{ y_1, \dots, y_n, x_1, \dots, x_m \}$ that leave the quadratic form $(y_1^2 + \cdots + y_n^2) - (x_1^2 + \cdots + x_m^2)$ invariant define the orthogonal group $O(n, m)$. Thus, the Lorentz group is $O(1, 3)$.

The group axioms are satisfied, and, in particular, each $\Lambda$ has an inverse. Using the fact that the determinant of a product is the product of the determinants, and that the transpose of a matrix has the same determinant, we find
\begin{align}
\Lambda^T \eta \Lambda &= \eta \Rightarrow (\det \Lambda)^2 = 1 \Rightarrow \det \Lambda = \pm 1, \\
\eta_{\mu \nu} \Lambda^\mu_{\ 0} \Lambda^\nu_{\ 0} &= (\Lambda^0_{\ 0})^2 - \sum_{k=1}^3 (\Lambda^k_{\ 0})^2 = 1 \Rightarrow (\Lambda^0_{\ 0})^2 \geq 1.
\end{align}
The Lorentz group thus has four disconnected components, with the subgroup satisfying $\det \Lambda = 1$ and $\Lambda^0_{\ 0} \geq 1$ being the \emph{proper orthochronous Lorentz group}, denoted $SO(1, 3)^+$. The remaining components can be obtained by combining an element of $SO(1, 3)^+$ with space or time reflection.

\subsection{Lorentz Algebra}

We have seen that the Lorentz group has six independent parameters, which we can collect into the antisymmetric matrix $\omega_{\mu \nu}$ as in Eq. (5.5). It is convenient to label the generators as $M_{\mu \nu} = -M_{\nu \mu}$, where each pair $(\mu, \nu)$ identifies a particular generator. Using the exponential map, any element $\Lambda \in SO(1, 3)^+$ can be written as
\[
\Lambda = \exp \left( -\frac{i}{2} \omega_{\mu \nu} M^{\mu \nu} \right),
\]
with a choice of constants. Given a finite-dimensional representation $(\rho, V)$ of dimension $n$, this group element is represented by the $n \times n$ matrix
\[
\Lambda_\rho = \exp \left( -\frac{i}{2} \omega_{\mu \nu} M^{\mu \nu}_\rho \right),
\]
which acts on $V$, where $M^{\mu \nu}_\rho$ are the generators in the representation $\rho$. The elements of $V$ transform under a Lorentz transformation as
\[
\varphi^i \to \left[ e^{-\frac{i}{2} \omega_{\mu \nu} M^{\mu \nu}_\rho} \right]^i_j \varphi^j.
\]
For an infinitesimal Lorentz transformation with small parameters $\omega_{\mu \nu}$, the variation of $\varphi^i$ is
\[
\delta \varphi^i = -\frac{i}{2} \omega_{\mu \nu} \left( M^{\mu \nu}_\rho \right)^i_j \varphi^j,
\]
where $M^{\mu \nu}_\rho$ are matrices labeled by $\mu$ and $\nu$, with indices $i$ and $j$ indicating the matrix representation.

The commutation relations of the Lorentz algebra generators, which are representation-independent, are given by
\[
[M^{\mu \nu}, M^{\rho \sigma}] = i \left( \eta^{\nu \rho} M^{\mu \sigma} - \eta^{\mu \rho} M^{\nu \sigma} - \eta^{\sigma \mu} M^{\rho \nu} + \eta^{\sigma \nu} M^{\rho \mu} \right).
\]
It is useful to decompose these generators into two spatial vectors:
\[
J^i = \frac{1}{2} \epsilon^{ijk} M^{jk}, \quad K^i = M^{i0},
\]
so that the Lie algebra of the Lorentz group can be written as
\begin{subequations}
    \begin{align}
        [J^i, J^j] &= i \epsilon^{ijk} J^k, \\
        [J^i, K^j] &= i \epsilon^{ijk} K^k, \\
        [K^i, K^j] &= -i \epsilon^{ijk} J^k.
    \end{align}
\end{subequations}
Equation (5.13a) is the Lie algebra of $SU(2)$, where $J$ can be interpreted as the angular momentum operator. The remaining commutation relations indicate that $K$ behaves as a spatial vector under rotation.

Using the rotation vector $\theta^i = \frac{1}{2} \epsilon^{ijk} \omega_{jk}$ and boost vector $\eta^i = \omega_{i0}$, any Lorentz transformation can be expressed as
\[
\Lambda = e^{-i \theta \cdot J + i \eta \cdot K}.
\]

\subsubsection{Trivial Representation}

In the trivial representation, a scalar field $\varphi$ is invariant under Lorentz transformations, so
\[
\delta \varphi = 0,
\]
and the generators, represented by $1 \times 1$ matrices, vanish identically:
\[
M^{\mu \nu} \equiv 0.
\]

\subsubsection{Vector Representation}

The defining (or vector) representation acts on four-vectors $V^\mu$, which transform according to
\[
V^\mu \to \Lambda^\mu_{\ \nu} V^\nu,
\]
where $\Lambda$ satisfies the Lorentz condition $\Lambda^T \eta \Lambda = \eta$. For a four-vector $V^\mu$, we find that the representation matrices are
\[
(M^{\mu \nu})^\alpha_{\ \beta} = i \left( \eta^{\mu \alpha} \delta^\nu_\beta - \eta^{\nu \alpha} \delta^\mu_\beta \right).
\]
This representation satisfies the commutation relations of the Lorentz algebra as expected, and provides a concrete example of the abstract commutation relations in Eq. (5.11).

\subsubsection{Spinor Representation}

To define the spinor representation, we use the fact that the group $SL(2; \mathbb{C})$ is the universal cover of the proper orthochronous Lorentz group $SO(1,3)^+$. Elements $\Lambda \in SO(1,3)^+$ are associated with elements $S \in SL(2; \mathbb{C})$, such that $\Lambda$ acts on four-vectors via
\[
x'^\mu = \Lambda^\mu_{\ \nu} x^\nu,
\]
while $S$ acts on spinors via
\[
\psi \to S \psi,
\]
where $\psi$ is a two-component object. The specific form of this transformation depends on the Pauli matrices and the relation between spinor and vector representations.

Spinor representations are crucial in quantum field theory, as they enable the description of particles with half-integer spin.

\section{One-particle State Representations}

In quantum field theory, one-particle states are described as irreducible representations of the Poincaré group. These representations provide insight into the properties of particles in terms of mass, spin, and energy-momentum.

\subsection{The Hilbert Space of One-particle States}

Consider the Hilbert space $\mathcal{H}$ of one-particle states. A one-particle state is specified by its four-momentum $p^\mu = (p^0, \mathbf{p})$ and additional quantum numbers $\sigma$ (e.g., spin) that distinguish states with the same four-momentum. We denote these states by $|p, \sigma \rangle$, with the inner product
\[
\langle p, \sigma | p', \sigma' \rangle = 2 p^0 (2\pi)^3 \delta^3(\mathbf{p} - \mathbf{p}') \delta_{\sigma \sigma'}.
\]

The action of the Poincaré group on $\mathcal{H}$ is specified by a unitary representation $U(\Lambda, a)$, where $\Lambda$ is a Lorentz transformation and $a$ is a spacetime translation. For a state $|p, \sigma \rangle$, we have
\[
U(\Lambda, a) |p, \sigma \rangle = e^{-i \Lambda p \cdot a} \sum_{\sigma'} C_{\sigma \sigma'}(\Lambda, p) | \Lambda p, \sigma' \rangle,
\]
where $C_{\sigma \sigma'}(\Lambda, p)$ is a unitary matrix that acts on the spin indices.

\subsection{Classification by Casimir Operators}

To classify the irreducible representations of the Poincaré group, we use its Casimir operators. These operators are constructed from the generators of translations $P_\mu$ and Lorentz transformations $M_{\mu \nu}$, which satisfy the Poincaré algebra. The two Casimir operators are:
\begin{itemize}
    \item The squared four-momentum operator
    \[
    P^2 = P_\mu P^\mu.
    \]
    This operator represents the squared mass of the particle, as it commutes with all other generators in the algebra. In an irreducible representation, its eigenvalue is $m^2$, where $m$ is the particle's mass.
    
    \item The Pauli-Lubanski vector
    \[
    W^\mu = -\frac{1}{2} \epsilon^{\mu \nu \rho \sigma} P_\nu M_{\rho \sigma},
    \]
    which satisfies
    \[
    [P_\mu, W^\nu] = 0.
    \]
    The squared Pauli-Lubanski operator, $W^2 = W_\mu W^\mu$, commutes with all generators and, in an irreducible representation, takes the eigenvalue $-m^2 s(s+1)$, where $s$ is the particle’s spin.
\end{itemize}

Thus, irreducible representations are characterized by the eigenvalues of $P^2$ and $W^2$, which correspond to the mass and spin of the particle.

\subsection{Massive and Massless Representations}

\begin{itemize}
    \item \textbf{Massive case} ($m > 0$): For particles with non-zero mass, we can move to the rest frame where $p^\mu = (m, 0, 0, 0)$. In this frame, the Pauli-Lubanski vector has components
    \[
    W^0 = 0, \quad W^i = m J^i,
    \]
    where $J^i$ are the generators of rotations, which correspond to the spin of the particle. The spin $s$ is quantized as $s \in \{0, \frac{1}{2}, 1, \dots \}$, where half-integer values arise when considering the covering group $SL(2; \mathbb{C})$ instead of $SO(1,3)^+$.
    
    \item \textbf{Massless case} ($m = 0$): For massless particles, there is no rest frame, as they always travel at the speed of light. The relevant subgroup that leaves the four-momentum $p^\mu = (E, 0, 0, E)$ invariant is the little group $ISO(2)$, which includes rotations around and translations along the direction of $p^\mu$. The Pauli-Lubanski vector $W^\mu$ becomes proportional to $p^\mu$ itself:
    \[
    W^\mu = \lambda p^\mu,
    \]
    where $\lambda$ is interpreted as the \emph{helicity} of the particle, taking values in $\{\pm s\}$.
\end{itemize}

In summary:
\begin{itemize}
    \item \textbf{Massive representations} are characterized by two parameters: the mass $m$ and the spin $s$.
    \item \textbf{Massless representations} are characterized by the helicity $\lambda$.
\end{itemize}

The study of these representations in the context of one-particle states provides insight into the properties of elementary particles, including their classification under the Poincaré symmetry group.


\color{black}

\appendix{ciao}
\chapter{Conventions}
\section{Levi-Civita Symbol}\label{app:levi-civita}
In this appendix we fix the conventions for the \emph{Levi-Civita} symbol. We mainly refer to the Wikipedia page~\cite{wikipedia}. In these notes we're interested only in the Levi-Civita symbol in three dimensions, therefore we'll stick to it. We define the Levi-Civita symbol
\begin{equation}
    \epsilon_{ijk} = 
    \begin{cases}
        +1, \textup{if $(i,j,k)$ is an even permutation of $(1,2,3)$}, \\
        -1, \textup{if $(i,j,k)$ is an odd permutation of $(1,2,3)$},\\
        0, \textup{otherwise} ,
    \end{cases}
\end{equation}
and adopt the convention
\begin{equation*}
    \epsilon^{ijk} = \epsilon_{ijk}.
\end{equation*}
Pay attention to what convention the sources you're referring use. Indeed, this is not always the most convenient choice, in particular when dealing with the Levi-Civita tensor. Without dealing these details, in our case, just recall that when to contiguous indices are permuted, we get a minus sign.

Using Einstein's summation convention, some useful relations are
\begin{subequations}
\begin{align}
    \epsilon_{ijk} \epsilon^{pqk} &= \delta^p_i \delta^q_j - \delta^q_i \delta^p_j \label{appeq:levi-civita-delta} \\
    \epsilon_{jmn} \epsilon^{imn} &= 2 \delta^i_j \\
    \epsilon_{ijk} \epsilon^{ijk} &= 6 .
\end{align}
\end{subequations}

Let's then show that
\begin{equation*}
    J^i = \frac{1}{2} \epsilon^{ijk} M^{jk} \iff M^{ij} = \epsilon^{ijk} J^k, \quad \textup{if $M^{ij} = -M^{ji}$.}
\end{equation*}
\begin{proof}
    Multiplying both sides by $\epsilon_{iln}$ and using eq.~\eqref{appeq:levi-civita-delta} and the antisymmetry:
    \begin{equation*}
        \epsilon_{iln} J^i = \frac{1}{2} \epsilon_{iln} \epsilon^{ijk} \omega^{jk} = \frac{1}{2} ( \delta^j_l \delta^k_n - \delta^k_l \delta^j_n ) \omega^{jk} = \frac{1}{2} (\omega^{ln} - \omega^{nl}) = \omega^{ln} .
    \end{equation*}
    Relabelling dummy indices and using $\epsilon_{ijk} = \epsilon_{kij}$, we finally obtain
    \begin{equation*}
        \omega^{ij} = \epsilon_{ijk} \theta^k = \epsilon^{ijk} \theta^k \qedhere
    \end{equation*}

\end{proof}
\bibliography{bibliography.bib}

\end{document}
